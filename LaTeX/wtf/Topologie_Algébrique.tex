\documentclass{report}
% Maths Packages
\usepackage{mathtools, amsthm, amssymb, mathrsfs, interval, stmaryrd, centernot, esvect, cancel, commath, blkarray, empheq}
\usepackage{tabularx}
\usepackage{booktabs}
\usepackage{cellspace}
\setlength{\cellspacetoplimit}{5pt}
\setlength{\cellspacebottomlimit}{5pt}

% Sagemaths Formating Packages
\usepackage{listings}
\lstdefinelanguage{Sage}[]{Python}
{morekeywords={False,sage,True},sensitive=true}
\lstset{
  frame=none,
  showtabs=False,
  showspaces=False,
  showstringspaces=False,
  commentstyle={\ttfamily\color{dgreencolor}},
  keywordstyle={\ttfamily\color{dbluecolor}\bfseries},
  stringstyle={\ttfamily\color{dgraycolor}\bfseries},
  language=Sage,
  basicstyle={\fontsize{10pt}{10pt}\ttfamily},
  aboveskip=0.4em,
  belowskip=0.4em,
}

% TOC Packages
\usepackage{tocloft, titletoc, hyperref, bookmark}
% Formatting / Style Packages
\usepackage[T1]{fontenc}
\usepackage{geometry, subcaption, graphicx, fix-cm, accents, float, varwidth, soul, ulem, contour, multicol, enumitem}    
\usepackage[bottom]{footmisc}
\usepackage[x11names, table]{xcolor}
\usepackage[most, skins]{tcolorbox}
\usepackage{adjustbox}
\DeclareMathAlphabet{\mathmybb}{U}{bbold}{m}{n} % Indicatrices
\newcommand{\1}{\mathmybb{1}}

% Tikz
\usepackage{tikz, tkz-fct, tkz-euclide, tikz-cd, tkz-fct, pgfplots}
\pgfplotsset{compat=1.18}
\usetikzlibrary{
  angles, quotes, 3d, positioning,
  shapes,fit, arrows, arrows.meta, calc, 
  matrix, calligraphy, intersections, 
  quotes, patterns, patterns.meta, 
  decorations.pathreplacing, decorations.markings,decorations.pathmorphing,
}
\usepgfplotslibrary{fillbetween}
\tikzset{
  withparens/.style = {draw, outer sep=0pt,
    left delimiter= (, right delimiter=),
    above delimiter= (, below delimiter=),
    align=center},
  withbraces/.style = {draw, outer sep=0pt,
    left delimiter=\{, right delimiter=\},
    above delimiter=\{, below delimiter=\},
    align=center}
}
\tikzcdset{
  arrow style=tikz,
  diagrams={>={Straight Barb[scale=1]}},
}

% PAGE SETTINGS

\geometry{
  left=25mm, right=25mm, top= 15mm, bottom= 15mm,
  footskip=30pt
  }
\setlength{\parindent}{0cm}
\setlength{\parskip}{0cm}
\setlist[itemize]{itemsep=0pt, leftmargin=25pt}

\setlength{\cftbeforetoctitleskip}{0pt}
\setlength{\cftaftertoctitleskip}{0pt}


\begin{document}
\chapter*{\chapterstyle{Homotopie}}
Dans toute la suite, on considèrera \(X, Y\) trois espaces topologiques quelconques, on cherche alors classifier les invariants algébriques des espaces topologiques. Mais aussi on cherchera à comprendre le comportement topologique des applications continues entre ces espaces, c'est ceci que nous appelerons \textbf{homotopie} et qui sera la pierre de voute de la topologie algébrique élémentaire car elle permettra de définir des \textbf{groupes} "classifiants" sur les espaces topologiques.\<

\subsection*{\subsecstyle{Homotopie{:}}}
On considère deux applications continues \(f, g : X \longrightarrow Y\), on considère alors les applications de la forme:
\[
   \begin{aligned}
      H : \icc{0}{1} &\longrightarrow \mathscr{F}(X, Y)\\
      t &\longmapsto H(t)
   \end{aligned}
\]
Ces applications sont clairement des chemins dans l'espace fonctionnel \(\mathscr{F}(X, Y)\), on veut alors que la contrainte suivante soit vérifiée:
\[
   \begin{cases}
      H(0) &= f\\
      H(1) &= g
   \end{cases}
\]
En d'autres termes, ce chemin \textbf{relie} \(f\) et \(g\). Mais on veut aussi que cette application soit \textbf{continue en ses deux variables}, ie que l'application \textbf{décurryfiée} suivant soit continue:
\[
   \begin{aligned}
      H : \icc{0}{1} \times X &\longrightarrow Y\\
      (t, x) &\longmapsto H(t)(x) = H(t, x)
   \end{aligned}
\]
On dira alors que \(H\) est une \textbf{homotopie} entre \(f, g\) et que ces deux fonctions sont homotopes. Cela signifera alors moralement qu'on peut passer d'une de ses fonctions à l'autre continument par un chemin fonctionnel. On verra par la suite des exemples d'applications homotopes facilement illustrables.

\subsection*{\subsecstyle{Homotopie des lacets{:}}}
On définit alors un cas particulier d'homotopie trés important, il s'agit de l'homotopie des lacets de \(X\) de point base \(x_0 \in X\).

\chapter*{\chapterstyle{Homologie Simpliciale}}
Dans ce chapitre nous cherchons à nouveau à trouver des invariants algébriques sur les espaces topologiques, mais on se propose de construire une théorie plus simple que celle de l'homotopie, en particulier, on sait que le \(n\)-ième groupe d'homotopie encapsule le nombre de "trous" de dimensions \(n\) dans un espace topologique, et on va définir un nouveau groupe appelé \textbf{groupe d'homologie} (simpliciale), la présentation devra définit beaucoup de nouveaux concepts, pour donner un aperçu:
\begin{itemize}
   \item Nous devrons définir le concept de \textbf{simplexe standard} qui sera l'objet géométrique de base, une enveloppe convexe primitive.
   \item Nous définirons ensuite le concept de \textbf{face} d'un simplexe qui sera utile pour la suite.
   \item Nous définirons ensuite le premier concept fondamental, celui de \textbf{complexe simplicial} qui est une structure que l'on donnera à notre espace topologique, d'une certaine manière on le "triangule".
   \item Nous définirons enfin le deuxième concept fondamental, celui de \textbf{chaine} qui nous permet d'effectuer des opérations sur le complexe simplicial.
   \item Nous définirons finalement le concept final qui menera au groupe d'homologie, l'opérateur de \textbf{bord} qui à une chaîne de dimension \(n\) nous donner une chaîne de dimension \(n-1\), son bord.
\end{itemize}
LET'S GO MTHERFUCKER
\subsection*{\subsecstyle{Simplexes{:}}}
On appelle \textbf{n-simplexe}, l'ensemble de \(\R^n\) suivant défini par:
\[
   \Bigl\{ \sum\alpha_iv_i\; ; \; (v_i) \in \R^n, \sum\alpha_i = 1 \Bigl\}
\]
En fait, ce qu'on considère ici, c'est l'enveloppe convexe de \(n\) points qu'on notera \(\), c'est en fait une généralisation de l'idée de triangle à toutes dimensions, la surface convexe la plus simple, en particulier on a:
\begin{itemize}
   \item Un \(0\)-simplexe est un point.
   \item Un \(1\)-simplexe est une droite.
   \item Un \(2\)-simplexe est un triangle.
   \item Un \(3\)-simplexe est une pyramide.
\end{itemize}ont 
On peut maintenant définir le concept de \textbf{simplexe standard} qui est simplement un simplexe générique défini par:
\[
   \Delta^n := \Bigl\{(v_0, \ldots, v_n) \in \R^{n+1} \; ; \; \sum t_i = 1, t_i \geq 0\Bigl\}
\]
Cela correspond à placer tout les simplexes dans le quadrant positif, à l'origine. Il sont universels et donc par la suite quand on parlera de simplexe, on considérera implicitement un simplexe standard.
\subsection*{\subsecstyle{Faces d'un simplexe{:}}}
On se donne un \(n\)-simplexe \(\Delta^n\) et on cherche à définir le concept d'une \textbf{face} du simplexe, et en fait ce sont eux aussi des simplexes de dimension inférieure, on se donne un nombre \(m \leq n\), alors une \(m\)-face du simplexe \(\Delta^n\) est n'importe quel simplexe obtenu en "retirant" \(n - m\) points, ie on a A PRECISER:
\[
   F_i = \langle v_1, \ldots v_{i_1}, \ldots, v{i_{n-m}} \ldots, v_n \rangle 
\]
C'est l'enveloppe convexe de l'ensemble des points initiaux privés de \(n - m\) points, c'est bien un simplexe de dimension \(m\) par construction. Par exemple si on considère \(\Delta^2 = \langle v_1, v_2, v_3 \rangle\), alors on a troise \(1\)-faces:
\[
   \begin{cases}
      F_1 = \langle v_2, v_3 \rangle\\
      F_2 = \langle v_1, v_3 \rangle\\
      F_3 = \langle v_1, v_2 \rangle
   \end{cases}
\]

\subsection*{\subsecstyle{Complexe simplicial{:}}}
Soit \(X\) un espace topologique, une \textbf{structure de complexe simplicial} sur \(X\) est une collection d'application continues de la forme:
\[
   \sigma_\alpha : \Delta^n \longrightarrow X
\]
Ces applications doivent vérifier trois propriétés:
\begin{itemize}
   \item \(\sigma_\alpha\vert_{\text{int}({\Delta^n})}\) est \textbf{injective} et tout point \(x \in X\) appartient à un unique \(\Im({\sigma_\alpha\vert_{\text{int}({\Delta^n})}})\)
   \item Toute restriction de \(\sigma_\alpha\) sur une \textbf{face} de \(\Delta^n\) est une autre \(\sigma_\beta\)
   \item Pour toute partie \(A \subseteq X\), \(A\) est ouvert ssi toutes \(\sigma_\alpha^{-1}(A)\) est ouvert.
\end{itemize}
Interprétation ....

\subsection*{\subsecstyle{Chaînes{:}}}
On se donne maintenant \((X, (\sigma_\alpha)\) un espace topologique muni d'une structure de complexe simplicial, alors on définit le groupe des \textbf{k-chaînes} comme l'ensemble des sommes formelles de \textbf{k-simplexes},ie on a:
\[
   C_k(X) := \Bigl\{ \sum_{i} m_i\sigma_i\; ; \; m_i \in \Z\Bigl\}
\]
En fait, on peut montrer facilement que ce sont des \textbf{groupes abéliens}, c'est sur ces groupes que l'on définira \textbf{l'opérateur de bord} dans la section suivante.

\subsection*{\subsecstyle{Opérateur de bord{:}}}
On note un n-simplexe \(\sigma\) de \(X\) par \((\sigma[0], \ldots, \sigma[n])\), on peut alors définir enfin l'opérateur suivant sur une k-chaîne:
\[
   \begin{aligned}
      \partial_n : \Delta_n &\longrightarrow \Delta_{n-1} \\
      \partial_n(\sigma) &\longmapsto \sum_{i = 0}^{n} (-1)^i(\sigma[0], \ldots, \sigma[i-1], \sigma[i+1], \ldots, \sigma[n])
   \end{aligned}
\]
Moralement, c'est \textbf{une somme alternée de tout les bords du simplexe initial}. En particulier l'alternance nous donne une propriété algébrique profonde et fondamentale pour définir l'homologie:
\[
   \partial_{n-1} \circ \partial_{n} = 0
\]
\subsection*{\subsecstyle{Groupes d'homologies{:}}}
Finalement, on a d'aprés les propriétés de l'opérateur de bord que:
\[
   \Im{\partial_{n+1}} \subseteq \Ker{\partial_{n}}
\]
On définit alors le \textbf{i-ème groupe d'homologie de X} par:
\[
   H^\Delta_i = \Ker{\partial_{n}} / \Im{\partial_{n+1}}
\]
Ce sont moralement les simplexes qui n'ont pas de bords et qui ne sont pas le bord d'un simplexe plus grand.

\end{document}
