\documentclass{report}
% Maths Packages
\usepackage{mathtools, amsthm, amssymb, mathrsfs, interval, stmaryrd, centernot, esvect, cancel, commath, blkarray, empheq}
\usepackage{tabularx}
\usepackage{booktabs}
\usepackage{cellspace}
\setlength{\cellspacetoplimit}{5pt}
\setlength{\cellspacebottomlimit}{5pt}

% Sagemaths Formating Packages
\usepackage{listings}
\lstdefinelanguage{Sage}[]{Python}
{morekeywords={False,sage,True},sensitive=true}
\lstset{
  frame=none,
  showtabs=False,
  showspaces=False,
  showstringspaces=False,
  commentstyle={\ttfamily\color{dgreencolor}},
  keywordstyle={\ttfamily\color{dbluecolor}\bfseries},
  stringstyle={\ttfamily\color{dgraycolor}\bfseries},
  language=Sage,
  basicstyle={\fontsize{10pt}{10pt}\ttfamily},
  aboveskip=0.4em,
  belowskip=0.4em,
}

% TOC Packages
\usepackage{tocloft, titletoc, hyperref, bookmark}
% Formatting / Style Packages
\usepackage[T1]{fontenc}
\usepackage{geometry, subcaption, graphicx, fix-cm, accents, float, varwidth, soul, ulem, contour, multicol, enumitem}    
\usepackage[bottom]{footmisc}
\usepackage[x11names, table]{xcolor}
\usepackage[most, skins]{tcolorbox}
\usepackage{adjustbox}
\DeclareMathAlphabet{\mathmybb}{U}{bbold}{m}{n} % Indicatrices
\newcommand{\1}{\mathmybb{1}}

% Tikz
\usepackage{tikz, tkz-fct, tkz-euclide, tikz-cd, tkz-fct, pgfplots}
\pgfplotsset{compat=1.18}
\usetikzlibrary{
  angles, quotes, 3d, positioning,
  shapes,fit, arrows, arrows.meta, calc, 
  matrix, calligraphy, intersections, 
  quotes, patterns, patterns.meta, 
  decorations.pathreplacing, decorations.markings,decorations.pathmorphing,
}
\usepgfplotslibrary{fillbetween}
\tikzset{
  withparens/.style = {draw, outer sep=0pt,
    left delimiter= (, right delimiter=),
    above delimiter= (, below delimiter=),
    align=center},
  withbraces/.style = {draw, outer sep=0pt,
    left delimiter=\{, right delimiter=\},
    above delimiter=\{, below delimiter=\},
    align=center}
}
\tikzcdset{
  arrow style=tikz,
  diagrams={>={Straight Barb[scale=1]}},
}

% PAGE SETTINGS

\geometry{
  left=25mm, right=25mm, top= 15mm, bottom= 15mm,
  footskip=30pt
  }
\setlength{\parindent}{0cm}
\setlength{\parskip}{0cm}
\setlist[itemize]{itemsep=0pt, leftmargin=25pt}

\setlength{\cftbeforetoctitleskip}{0pt}
\setlength{\cftaftertoctitleskip}{0pt}


\begin{document}
\chapter*{\chapterstyle{Ordre}}
\subsection*{\subsecstyle{Filtres \& Ultrafiltres{:}}}
Soit \(E\) un ensemble on appelle \textbf{filtre} sur \(E\) tout ensemble de parties \(\mathscr{F}\) tel que:
\begin{itemize}
   \item \textbf{Non-trivialité :} Il ne contient pas l'ensemble vide.
   \item \textbf{Hérédité :} Si \(A \in \mathscr{F}\), alors toute partie qui contient \(A\) est dans \(\mathscr{F}\).
   \item \textbf{Stabilité par intersection finie}
\end{itemize}
Par exemple considérons l'ensemble \(E = \{ 1, 2, 3, 4\}\), alors on pose \(\mathscr{F} = \{\{1, 2\}, \{1, 2, 3\}, \{1, 2, 4\}, \{1, 2, 3, 4\}\}\) et on vérifie facilement que c'est un \textbf{filtre} sur \(E\), et on appelle alors \(E\) espace filtré.\<

On peut définir alors le concept de \textbf{filtre engendré} par des parties, c'est le plus petit filtre qui contient ces parties, par exemple le filtre précédent est le filtre engendré par \(\{1, 2\}\).\<

On définit alors aussi le concept \textbf{d'ultrafiltre}, un ultrafiltre étant un filtre maximal pour l'inclusion, ie un filtre qui n'est pas contenu dans un filtre plus grand.

\subsection*{\subsecstyle{Limite d'un filtre{:}}}
Soit \(x\) un élément d'un espace filtré \((X, \mathscr{F})\) alors on dira que \(x\) est (une) \textbf{limite du filtre} et on dira que le filtre \textbf{converge vers} \(x\) si et seulement si \(\mathscr{F}\) contient le filtre engendré par les voisinages de \(x\).

\subsection*{\subsecstyle{Equivalence des limites{:}}}
On peut alors montrer que le concept de filtre généralise le concept de limite d'une suite ou d'une fonction. Soit \((u_n)\) une suite à valeur dans un espace topologique \(X\), on définit alors le \textbf{filtre de Fréchet} par:
\[
   \mathscr{F}_{Frechet} := \{ A \subseteq \N \; ; \; A^c \text{ est fini }\}
\]
\begin{center}
   \textbf{Alors la limite du filtre de Frechet est exactement la limite classique de cette suite.}
\end{center}
En particulier se donner un filtre sur un ensemble est moralement équivalent à se donner une "définition de limite" sur cet ensemble.

\end{document}