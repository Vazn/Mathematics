\documentclass{report}
% Maths Packages
\usepackage{mathtools, amsthm, amssymb, mathrsfs, interval, stmaryrd, centernot, esvect, cancel, commath, blkarray, empheq}
\usepackage{tabularx}
\usepackage{booktabs}
\usepackage{cellspace}
\setlength{\cellspacetoplimit}{5pt}
\setlength{\cellspacebottomlimit}{5pt}

% Sagemaths Formating Packages
\usepackage{listings}
\lstdefinelanguage{Sage}[]{Python}
{morekeywords={False,sage,True},sensitive=true}
\lstset{
  frame=none,
  showtabs=False,
  showspaces=False,
  showstringspaces=False,
  commentstyle={\ttfamily\color{dgreencolor}},
  keywordstyle={\ttfamily\color{dbluecolor}\bfseries},
  stringstyle={\ttfamily\color{dgraycolor}\bfseries},
  language=Sage,
  basicstyle={\fontsize{10pt}{10pt}\ttfamily},
  aboveskip=0.4em,
  belowskip=0.4em,
}

% TOC Packages
\usepackage{tocloft, titletoc, hyperref, bookmark}
% Formatting / Style Packages
\usepackage[T1]{fontenc}
\usepackage{geometry, subcaption, graphicx, fix-cm, accents, float, varwidth, soul, ulem, contour, multicol, enumitem}    
\usepackage[bottom]{footmisc}
\usepackage[x11names, table]{xcolor}
\usepackage[most, skins]{tcolorbox}
\usepackage{adjustbox}
\DeclareMathAlphabet{\mathmybb}{U}{bbold}{m}{n} % Indicatrices
\newcommand{\1}{\mathmybb{1}}

% Tikz
\usepackage{tikz, tkz-fct, tkz-euclide, tikz-cd, tkz-fct, pgfplots}
\pgfplotsset{compat=1.18}
\usetikzlibrary{
  angles, quotes, 3d, positioning,
  shapes,fit, arrows, arrows.meta, calc, 
  matrix, calligraphy, intersections, 
  quotes, patterns, patterns.meta, 
  decorations.pathreplacing, decorations.markings,decorations.pathmorphing,
}
\usepgfplotslibrary{fillbetween}
\tikzset{
  withparens/.style = {draw, outer sep=0pt,
    left delimiter= (, right delimiter=),
    above delimiter= (, below delimiter=),
    align=center},
  withbraces/.style = {draw, outer sep=0pt,
    left delimiter=\{, right delimiter=\},
    above delimiter=\{, below delimiter=\},
    align=center}
}
\tikzcdset{
  arrow style=tikz,
  diagrams={>={Straight Barb[scale=1]}},
}

% PAGE SETTINGS

\geometry{
  left=25mm, right=25mm, top= 15mm, bottom= 15mm,
  footskip=30pt
  }
\setlength{\parindent}{0cm}
\setlength{\parskip}{0cm}
\setlist[itemize]{itemsep=0pt, leftmargin=25pt}

\setlength{\cftbeforetoctitleskip}{0pt}
\setlength{\cftaftertoctitleskip}{0pt}


\begin{document}
\chapter*{\chapterstyle{Calcul variationnel}}
Le calcul variationnel est en analyse fonctionnelle, un ensemble de méthodes permettant de minimiser une fonctionnelle. Celle-ci, qui est à valeurs réelles, dépend d'une fonction qui est l'inconnue du problème. Il s'agit donc d'un problème de minimisation dans un espace fonctionnel de dimension infinie. Par exemple:
\begin{itemize}
   \item \textbf{Problème géodésique :}Rechercher le \textbf{chemin} tel qu'une distance soit minimisée.
   \item \textbf{Problème isopérimétrique :} Rechercher la \textbf{surface} telle que pour un périmètre donné, une aire soit maximisée.
   \item \textbf{Principe de moindre action :} Rechercher un chemin qui minimise une fonctionelle qui représente l'énergie du système en physique.
\end{itemize}

Dans la majorité des situations, la fonctionnelle à minimiser sera de la forme:
\[
   S(\gamma) = \int_a^b \mathcal{L}(x, \gamma(x), \gamma'(x)) dx 
\]
On appelle alors une telle fonctionelle \textbf{action} et l'intégrande est apellée \textbf{Lagrangien} du système. Par exemple, le lagrangien du problème géodésique dans \(R^n\) est simplement:
\[
   \mathcal{L}(x, \gamma(x), \gamma'(x)) = ||\gamma'(x)||
\]
Minimiser l'action correspondante correspond alors à trouver le plus court chemin entre \(a\) et \(b\).

 

\subsection*{\subsecstyle{Filtres \& Ultrafiltres{:}}}

\end{document}