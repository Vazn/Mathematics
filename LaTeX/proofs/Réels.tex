\documentclass{report}
% Maths Packages
\usepackage{mathtools, amsthm, amssymb, mathrsfs, interval, stmaryrd, centernot, esvect, cancel, commath, blkarray, empheq}
\usepackage{tabularx}
\usepackage{booktabs}
\usepackage{cellspace}
\setlength{\cellspacetoplimit}{5pt}
\setlength{\cellspacebottomlimit}{5pt}

% Sagemaths Formating Packages
\usepackage{listings}
\lstdefinelanguage{Sage}[]{Python}
{morekeywords={False,sage,True},sensitive=true}
\lstset{
  frame=none,
  showtabs=False,
  showspaces=False,
  showstringspaces=False,
  commentstyle={\ttfamily\color{dgreencolor}},
  keywordstyle={\ttfamily\color{dbluecolor}\bfseries},
  stringstyle={\ttfamily\color{dgraycolor}\bfseries},
  language=Sage,
  basicstyle={\fontsize{10pt}{10pt}\ttfamily},
  aboveskip=0.4em,
  belowskip=0.4em,
}

% TOC Packages
\usepackage{tocloft, titletoc, hyperref, bookmark}
% Formatting / Style Packages
\usepackage[T1]{fontenc}
\usepackage{geometry, subcaption, graphicx, fix-cm, accents, float, varwidth, soul, ulem, contour, multicol, enumitem}    
\usepackage[bottom]{footmisc}
\usepackage[x11names, table]{xcolor}
\usepackage[most, skins]{tcolorbox}
\usepackage{adjustbox}
\DeclareMathAlphabet{\mathmybb}{U}{bbold}{m}{n} % Indicatrices
\newcommand{\1}{\mathmybb{1}}

% Tikz
\usepackage{tikz, tkz-fct, tkz-euclide, tikz-cd, tkz-fct, pgfplots}
\pgfplotsset{compat=1.18}
\usetikzlibrary{
  angles, quotes, 3d, positioning,
  shapes,fit, arrows, arrows.meta, calc, 
  matrix, calligraphy, intersections, 
  quotes, patterns, patterns.meta, 
  decorations.pathreplacing, decorations.markings,decorations.pathmorphing,
}
\usepgfplotslibrary{fillbetween}
\tikzset{
  withparens/.style = {draw, outer sep=0pt,
    left delimiter= (, right delimiter=),
    above delimiter= (, below delimiter=),
    align=center},
  withbraces/.style = {draw, outer sep=0pt,
    left delimiter=\{, right delimiter=\},
    above delimiter=\{, below delimiter=\},
    align=center}
}
\tikzcdset{
  arrow style=tikz,
  diagrams={>={Straight Barb[scale=1]}},
}

% PAGE SETTINGS

\geometry{
  left=25mm, right=25mm, top= 15mm, bottom= 15mm,
  footskip=30pt
  }
\setlength{\parindent}{0cm}
\setlength{\parskip}{0cm}
\setlist[itemize]{itemsep=0pt, leftmargin=25pt}

\setlength{\cftbeforetoctitleskip}{0pt}
\setlength{\cftaftertoctitleskip}{0pt}


\begin{document}
   \subsection*{\subsecstyle{Caractérisation de la borne supérieure}}
   Soit \(m\) un majorant d'une partie \(A \subseteq \R\), montrons que:
   \[
      m = \sup(A) \Longleftrightarrow \forall \epsilon > 0 \; \exists a \in A \; ; \; m - \epsilon < a  
   \]
   Soit \(\epsilon > 0\), supposons que \(m\) soit la borne supérieure de \(A\) et montrons qu'il existe bien un élément de \(A\) tel que \(m - \epsilon\) ne soit pas un majorant.
   On sait que \(m\) est le plus petit majorant, donc en particulier \(m - \epsilon\) est plus petit que \(m\) et n'est donc pas un majorant, ce qui signifie par définition qu'il existe \(a \in A\) tel que \(m - \epsilon < a\).\<

   Réciproquement supposons qu'il existe \(a \in A\) tel que \(m - \epsilon < a\), montrons alors par l'absurde que \(m\) est le plus petit des majorants. Supposons qu'il ne soit pas le plus petit, alors il existerait un \(m' < m\) tel que \(m'\) soit le plus petit des majorants\footnote{\label{Pourquoi1} Pourquoi ?}.\+
   Mais alors \(m' = m - (m - m')\), ie \(m' = m - \epsilon_0\) ce qui est absurde car notre hypothèse nous assure de l'existence d'un \(a \in A\) tel que \(m' = m - \epsilon_0 < a\) ce qui contredit le fait que \(m'\) soit un majorant.

   \subsection*{\subsecstyle{Le corps des réels est Archimédien}}
   Soit \(x, y \in \R^{+*}\), montrons par l'absurde que :
   \[
      \exists n \in \N  \; ; \; nx > y  
   \]
   \begin{proof}[\unskip\nopunct]
   On a donc \(\forall n \in \N \; ; \; nx \leq y\), en particulier la partie \(E := \bigl\{ nx \; ; \; n \in \N \bigl\}\) est majorée par \(y\) et elle admet donc une borne supérieure \(\sup(E)\).\<

   On sait que \(x > 0\) donc on sait que \(\sup(E) - x\) n'est pas un majorant\footnoteref{Pourquoi1}, ce qui par définition signifie que qu'il existe \(m \in \N\) tel que:
   \[
      \sup(E) - x < mx \Longleftrightarrow \sup(E) < (m + 1)x  
   \] 
   Or \((m+1)x \in E\) et donc \(\sup(E)\) ne serait pas un majorant, ce qui est absurde.
   \end{proof}
   \subsection*{\subsecstyle{Existence de la partie entière}}
   \begin{proof}[\unskip\nopunct]
      Soit \(y \in \R\), on veut montrer qu'il existe un entier relatif noté \(\lfloor y \rfloor\) tel que:
      \[
         \lfloor y \rfloor \leq y < \lfloor y \rfloor + 1
      \]
      Considérons l'ensemble \(E := \bigl\{ n \in \Z \; ; \; n \leq y\bigl\}\), on va montrer que la partie entière est le maximum de cet ensemble.\<
      
      Supposons que cette ensemble soit vide, alors pour tout entier relatif \(n \in \Z\), on aurait \(y < n\) et donc \(\Z\) serait minoré, ce qui est absurde. Donc \(E\) est non-vide.\<

      Aussi, la propriété d'Archimède nous donne l'existence\footnoteref{Pourquoi1} d'un \(m \in \N\) tel que \(m > y\). Donc \(E\) est majoré par \(m\). \<

      On définit alors \(\lfloor y \rfloor = \max(E)\).
   \end{proof}
   \pagebreak
   \subsection*{\subsecstyle{Densité de \(\Q\) dans \(\R\)}}
   \begin{proof}[\unskip\nopunct]
      Soit \(x, y \in \R\), supposons sans perte de généralité que \(y > x\). Alors la propriété d'Archimède nous donne l'existence d'un \(n \in \N\) tel que:
      \[
         n(y - x) > 1
      \]
      Alors en partant de cette inégalité\footnote[1]{Pourquoi choisit-on \(1\) en particulier ? Interprétation géométrique ?} et de cet entier \(n\) , on obtient:
      \begin{flalign*}
         ny & > nx + 1\\
            & \geq \lfloor nx + 1 \rfloor \shorteqnote{(Par définition de la partie entière)} \\
            & = \lfloor nx \rfloor + 1 \shorteqnote{(Propriété élémentaire de la partie entière)} \\
            & > nx \shorteqnote{(Par définition de la partie entière)}
      \end{flalign*}
      En conclusion on a:
      \[
         ny > \lfloor nx \rfloor + 1 > nx
      \]
      La division par \(n\) conclut et exhibe un rationnel qui convient\footnote[2]{Pourquoi la densité de \(\Q\) dans \(\R\) rend-elle la démonstration de la densité de \(\R \backslash \Q\) dans \(\R\) "évidente" ?}.
   \end{proof}  
\end{document}