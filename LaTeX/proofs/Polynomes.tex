\documentclass{report}
% Maths Packages
\usepackage{mathtools, amsthm, amssymb, mathrsfs, interval, stmaryrd, centernot, esvect, cancel, commath, blkarray, empheq}
\usepackage{tabularx}
\usepackage{booktabs}
\usepackage{cellspace}
\setlength{\cellspacetoplimit}{5pt}
\setlength{\cellspacebottomlimit}{5pt}


% Sagemaths Formating Packages
\usepackage{listings}
\lstdefinelanguage{Sage}[]{Python}
{morekeywords={False,sage,True},sensitive=true}
\lstset{
  frame=none,
  showtabs=False,
  showspaces=False,
  showstringspaces=False,
  commentstyle={\ttfamily\color{dgreencolor}},
  keywordstyle={\ttfamily\color{dbluecolor}\bfseries},
  stringstyle={\ttfamily\color{dgraycolor}\bfseries},
  language=Sage,
  basicstyle={\fontsize{10pt}{10pt}\ttfamily},
  aboveskip=0.4em,
  belowskip=0.4em,
}

% TOC Packages
\usepackage{tocloft, titletoc, hyperref, bookmark}
% Formatting / Style Packages
\usepackage[T1]{fontenc}
\usepackage{geometry, subcaption, graphicx, fix-cm, accents, float, varwidth, soul, ulem, contour, multicol, enumitem}    
\usepackage[bottom]{footmisc}
\usepackage[x11names, table]{xcolor}
\usepackage[most, skins]{tcolorbox}
\usepackage{adjustbox}
\DeclareMathAlphabet{\mathmybb}{U}{bbold}{m}{n} % Indicatrices
\newcommand{\1}{\mathmybb{1}}

% Tikz
\usepackage{tikz, tkz-fct, tkz-euclide, tikz-cd, tkz-fct, pgfplots}
\pgfplotsset{compat=1.18}
\usetikzlibrary{
  angles, quotes, 3d, positioning,
  shapes,fit, arrows, arrows.meta, calc, 
  matrix, calligraphy, intersections, 
  quotes, patterns, patterns.meta, 
  decorations.pathreplacing, decorations.markings,decorations.pathmorphing,
}
\usepgfplotslibrary{fillbetween}
\tikzset{
  withparens/.style = {draw, outer sep=0pt,
    left delimiter= (, right delimiter=),
    above delimiter= (, below delimiter=),
    align=center},
  withbraces/.style = {draw, outer sep=0pt,
    left delimiter=\{, right delimiter=\},
    above delimiter=\{, below delimiter=\},
    align=center}
}
\tikzcdset{
  arrow style=tikz,
  diagrams={>={Straight Barb[scale=1]}},
}

% PAGE SETTINGS

\geometry{
  left=25mm, right=25mm, top= 15mm, bottom= 15mm,
  footskip=30pt
  }
\setlength{\parindent}{0cm}
\setlength{\parskip}{0cm}
\setlist[itemize]{itemsep=0pt, leftmargin=25pt}

\setlength{\cftbeforetoctitleskip}{0pt}
\setlength{\cftaftertoctitleskip}{0pt}

\setcounter{secnumdepth}{-1}

% STYLE
\definecolor{BrightBlue1}{RGB}{95, 150, 210}

\definecolor{BrightRed1}{RGB}{210, 95, 95}
\definecolor{BrightRed2}{RGB}{210, 115, 115}

\definecolor{DarkBlueX}{RGB}{43, 68, 92}
\definecolor{DarkBlue0}{RGB}{53, 78, 102}
\definecolor{DarkBlue1}{RGB}{83, 108, 132}
\definecolor{DarkBlue2}{RGB}{58, 94, 132}
\definecolor{DarkBlue3}{RGB}{90, 126, 162}

\definecolor{DarkGreen3}{RGB}{83, 132, 108}
\definecolor{DarkGreen2}{RGB}{58, 132, 94}
\definecolor{DarkGreen1}{RGB}{90, 162, 126}
\tcbset{shield externalize, enhanced, sharp corners, halign=center, center}

\definecolor{dblackcolor}{rgb}{0.0,0.0,0.0}
\definecolor{dbluecolor}{rgb}{0.01,0.02,0.7}
\definecolor{dgreencolor}{rgb}{0.2,0.4,0.0}
\definecolor{dgraycolor}{rgb}{0.30,0.3,0.30}
\newcommand{\dblue}{\color{dbluecolor}\bf}
\newcommand{\dred}{\color{dredcolor}\bf}
\newcommand{\dblack}{\color{dblackcolor}\bf}

%Underline settings
\setlength{\ULdepth}{2pt}
\contourlength{0.8pt}
\renewcommand{\underline}[1]{
  \uline{\phantom{#1}}%
  \llap{\contour{white}{#1}}%
}

%drop shadow southwest=black!100!black
\newcommand{\secstyle}[1]{\color{DarkBlue1}\fbox{#1}}
\newcommand{\subsecstyle}[1]{\color{DarkBlue2}\underline{#1}}
\newcommand{\subsubsecstyle}[2]{\color{DarkBlue3}\underline{#1}}

\newcommand{\chapterstyle}[1]{
    \setlength{\fboxsep}{0.3em}
    \setlength{\fboxrule}{3pt}
    \centering\vspace{-70pt}
    
    \color{DarkBlue1}\huge\fbox{\textbf{\textsc{#1}}}
}

\newcommand{\customBox}[2]{
    \tcbset{boxrule=1.5pt, boxsep=-0.2mm, colframe=DarkBlue1, colback=BrightBlue1!05}
    \begin{tcolorbox}[#1]
        \abovedisplayskip=0pt % remove vertical space above align
        #2
    \end{tcolorbox}
}

\makeatletter % Crée une trés grosse taille de police pour la page de garde
\newcommand\HUGE{\@setfontsize\Huge{40}{60}}
\makeatother   

\makeatletter
\newcommand\footnoteref[1]{\protected@xdef\@thefnmark{\ref{#1}}\@footnotemark}
\makeatother

% COMMANDS
% TOC
\renewcommand{\cftchapfont}{\large \bfseries \scshape}
\renewcommand{\cftsecfont}{}
\renewcommand{\contentsname}{\hfill
\setlength{\fboxsep}{0.3em}\setlength{\fboxrule}{3pt}\vspace{20pt}
   \color{DarkBlue1}\Huge
   \fbox{\textbf{\textsc{Table des matières}}}
   \hfill
}

% MATHS
\newcommand{\C}{\mathbb{C}}
\newcommand{\R}{\mathbb{R}}
\newcommand{\Q}{\mathbb{Q}}
\newcommand{\Z}{\mathbb{Z}}
\newcommand{\N}{\mathbb{N}}
\newcommand{\U}{\mathbb{U}}
\newcommand{\K}{\mathbb{K}}

\newcommand{\A}{\mathbf{\mathscr{A}}}
\newcommand{\B}{\mathbf{\mathscr{B}}}
\newcommand{\Fam}{\mathbf{\mathscr{F}}}
\renewcommand{\P}{\mathbf{\mathscr{P}}}

\renewcommand{\epsilon}{\varepsilon}
\renewcommand{\rho}{\varrho}

\newcommand{\E}{\mathbf{\mathcal{E}}}
\newcommand{\F}{\mathbf{\mathcal{F}}}
\newcommand{\Pow}{\mathbf{\mathcal{P}}}
\newcommand{\G}{\mathbf{\mathfrak{G}}}

\newcommand{\<}{\bigskip}
\newcommand{\+}{\par}

% Notation equality

\newcommand\notationEq{\stackrel{\mbox{
    \begin{tiny}  
        notation
    \end{tiny}    
}}{=}}

% INTERVALS

\intervalconfig{separator symbol =  \,; \,}

\newcommand{\ioo}[2]{\interval[open]{#1}{#2}}
\newcommand{\ioc}[2]{\interval[open left]{#1}{#2}}
\newcommand{\ico}[2]{\interval[open right]{#1}{#2}}
\newcommand{\icc}[2]{\interval{#1}{#2}}

\newcommand{\intioo}[2]{\left\rrbracket{#1}\;;\;{#2}\right\llbracket}
\newcommand{\intioc}[2]{\left\rrbracket{#1}\;;\;{#2}\right\rrbracket}
\newcommand{\intico}[2]{\left\llbracket{#1}\;;\;{#2}\right\llbracket}
\newcommand{\inticc}[2]{\left\llbracket{#1}\;;\;{#2}\right\rrbracket}

% EQUATIONS NOTES

\newcommand{\shorteqnote}[1]{ &  & \text{\small\llap{#1}}}
\newcommand{\longeqnote}[1]{& & \\ \notag&  &  &  &  & \text{\small\llap{#1}}}

% FUNCTIONS NOTATIONS

\newcommand{\inject}{\hookrightarrow} 
\newcommand{\surject}{\twoheadrightarrow}

% MOD NOTATION

\newcommand{\eqmod}[1]{\underset{#1}{\equiv}} 

% LINEAR ALGEBRA

\newcommand{\dotproduct}[2]{\left\langle\;\! #1 \;\! | \;\! #2 \;\! \right\rangle}
\newcommand{\vectNorm}[1]{\left\Vert#1 \right\Vert}

\newcommand{\Ker}[1]{\text{Ker}#1}
\newcommand{\Sp}[1]{\text{Sp}(#1)}
\renewcommand{\Im}[1]{\text{Im}#1}

\NewDocumentCommand{\opNorm}{sO{}m}{%
  \IfBooleanTF{#1}{% automatic scaling, use with care
    \left|\opnormkern\left|\opnormkern\left|
    #3
    \right|\opnormkern\right|\opnormkern\right|
  }{
    \mathopen{#2|\opnormkern #2|\opnormkern #2|}
    #3
    \mathclose{#2|\opnormkern #2|\opnormkern #2|}
  }%
}
\newcommand{\opnormkern}{\mkern-1.5mu\relax}% adjust for the font

% TOPOLOGY
\newcommand{\ball}{\mathscr{B}}

% CALCULUS
\newcommand{\partialD}[2]{\frac{\partial #1}{\partial #2}}

% GEOMETRY
\newcommand{\RightAgnle}[4][5pt]
{%
    \draw($#3!#1!#2$)-- ($#3!2! ($ ($#3!#1!#2$)!.5! ($#3!#1!#4$)$)$)-- ($#3!#1!#4$);
}

% PROBABILITIES
\newcommand{\probability}[1]{\mathbb{E} (#1)}
\newcommand{\expectancy}[1]{\mathbb{E} (#1)}
\newcommand{\variance}[1]{\mathbb{V} (#1)}
\newcommand{\covariance}[1]{\mathbb{C} (#1)}


\begin{document}
   Soit \(P \in K[X]\), \(\alpha \in \K^*\) et \(m \in \N\), on notera \(\tilde{P}\) la fonction polynomiale associée à \(P\).
   \subsection*{\subsecstyle{15 - Degré de la somme}}
   Soit \(P, Q\) des polynomes de degrés respectifs \(p, q\) et qu'on note respectivement \((a_n), (b_n)\), alors:
   \[
      P+Q = (c_n) \text{ telle que } c_n = \sum_{k=0}^n a_k + b_{k}
   \]
   Si \(p = q\), on a \(c_n = 0\) a partir du rang \(p + 1\) par définition du degré et donc les indices des termes non nuls sont tous inférieurs à \(p\) (et donc à \(q\)) et on a \(\deg(P+Q) \leq \max\{\deg(P), \deg(Q)\}\).\<

   Si \(p \neq q\), alors on peut considèrer sans perte de généralité que \(p < q\).\+
   Tout d'abord, on a \(c_n = 0\) a partir du rang \(q + 1\), et on a \(c_q = b_q \neq 0 \) donc \(\deg(P+Q) = \max\{\deg(P), \deg(Q)\}\)

   \subsection*{\subsecstyle{16 - Degré du produit}}
   \begin{proof}[\unskip\nopunct]
      Soit \(P, Q\) des polynomes de degrés respectifs \(p, q\) et qu'on note respectivement \((a_n), (b_n)\), alors:
      \[
         PQ = (c_n) \text{ telle que } c_n = \sum_{k=0}^n a_kb_{n-k}
      \]
      On a alors:
      \begin{flalign*}
         c_{p+q} &= \sum_{k=0}^{p+q}a_kb_{p+q-k} \\
         &= \sum_{k=0}^{p-1}a_kb_{p+q-k} + a_pb_q + \sum_{k=p+1}^{q}a_kb_{p+q-k} \shorteqnote{(On sépare la somme en trois.)}\\
         &= 0 + a_pb_q + 0 \shorteqnote{(Dans la première somme \(p+q-k > q\) et dans la seconde \(k > p\))}\\
         &= 1 \shorteqnote{(Car par définition du degré \(a_n, b_n \neq 0)\)}
      \end{flalign*}
      On a donc montré que \(\deg(PQ) \geq p+q\).
      Réciproquement, si \(k > p+q\):
      \begin{flalign*}
         c_{k} &= \sum_{i=0}^{k}a_ib_{k-i} \\
         &= \sum_{i=0}^{p}a_ib_{k-i} + \sum_{i=p+1}^{k}a_ib_{k-i}\\
         &= 0 \shorteqnote{(Car le second terme de la première somme est toujours nul et inversement.)}
      \end{flalign*}
      En effet, pour la première somme \(k - i > q\) et pour la seconde \(i > p\).

      Donc \(\deg(PQ) \leq p+q\) et donc on a bien montré l'égalité.
   \end{proof}
   
   \subsection*{\subsecstyle{17 - Caractérisation d'une racine}}
   \begin{proof}[\unskip\nopunct]
      Montrons la propriété suivante par double implication: 
      \[
         \tilde{P}(\alpha) = 0 \Longleftrightarrow (X - \alpha) \mid P
      \]

      Supposons que \(\tilde{P}(\alpha) = 0\), on sait par le théorème de la division euclidienne qu'il existe des uniques \(Q, R \in K[X]\) tels que: 
      \[
         P = (X - \alpha)Q + R \text{ avec } \deg(R) < \deg(X- \alpha) = 1
      \]
      Donc \(R\) est un polynome constant, or si on évalue le polynome en \(\alpha\), on obtient:
      \begin{flalign*}
         \tilde{P}(\alpha) 
         &= (\alpha - \alpha)\tilde{Q}(\alpha) + \tilde{R}(\alpha)\\ 
         &= \tilde{R}(\alpha)\\
         &= 0 \shorteqnote{(Par hypothèse car \(\tilde{P}(\alpha) = 0\))}
      \end{flalign*}
      Donc le reste est bien le polynome nul et on a la première implication. \<

      Réciproquement si \((X - \alpha) \mid P\), par définition, on a \(P = (X - \alpha)Q\) dont la fonction polynomiale associée s'annulera bien en \(\alpha\).
   \end{proof}
   
   \subsection*{\subsecstyle{18 - Nombre maximal de racines}}
   \begin{proof}[\unskip\nopunct]
      On considère ici que \(\deg(P) = n \geq 1\), et on veut montrer que \(P\) admet au plus \(n\) racines distinctes.\<

      Supposons par l'absurde qu'il admette \(p > n\) racines distinctes, qu'on note \(\alpha_1, \ldots, \alpha_p\). D'aprés le théorème fondamental, on sait donc que: 
      \[
         \prod_{k=1}^{p}(X - \alpha_k) \mid P
      \]
      On a donc \(P = \prod_{k=1}^{p}(X - \alpha_k)Q\) pour un certain \(Q \in \K[X] \, \backslash \, \{0\}\) (non-nul car \(P\) non-nul) qui est absurde.\<
      
      En effet le degré du membre de gauche est par hypothèse \(n\) et celui du membre de droite est d'au moins \(p\) (on a un produit de \(p\) monomes de degré 1, donc on applique la formule du degré d'un produit).
   \end{proof}
   
   \subsection*{\subsecstyle{19 - Multiplicité au moins \(m\)}}
   \begin{proof}[\unskip\nopunct]
      Montrons la propriété suivante par double implication:
      \[
         \text{mult}(\alpha) \geq m \Longleftrightarrow (X-\alpha)^m \mid P
      \]
      Supposons que \(\text{mult}(\alpha) = n \geq m\), alors on a:
      \[
         (X - \alpha)^{m} \mid  (X - \alpha)^{n}
      \]
      Or par définition de la multiplicité on a aussi:
      \[
         (X - \alpha)^n \mid  P \quad\quad \text{ ET } \quad\quad (X - \alpha)^{n+1} \nmid P   
      \]
      Finalement en combinant ces informations on a:
      \[
         (X - \alpha)^{m} \mid  (X - \alpha)^{n} \mid P
      \]
      Et on conclut par transitivé de la relation de divisibilité.\<
   
      Réciproquement par l'absurde, si on a \((X - \alpha)^m \mid P\) et \(n < m\), alors \(n + 1 \leq m\), alors on a:
      \[
         (X - \alpha)^{n+1} \mid (X - \alpha)^{m} \mid P
      \]
      Et donc par transitivité:
      \[
         (X - \alpha)^{n+1} \mid P
      \]
      Ce qui est absurde par définition de la multiplicité.
   \end{proof}
   \pagebreak

   \subsection*{\subsecstyle{20 - Caractérisation de la multiplicité I}}
   \begin{proof}[\unskip\nopunct]
      Montrons la propriété suivante pas double implication:
      \begin{align*}
         \text{mult}(\alpha) = m &\Longleftrightarrow P = (X - \alpha)^mQ \quad \text{ ET } \quad Q(\alpha) \neq 0\\
         &\Longleftrightarrow (X - \alpha)^m \mid P \quad \text{ ET } \quad Q(\alpha) \neq 0
      \end{align*}

      Si mult(\(\alpha\)) = m, alors par définition cela signifie que: 
      \[
         (X - \alpha)^m \mid P \quad \quad \text{ ET } \quad\quad (X - \alpha)^{m+1} \centernot \mid P
      \]
      Supposons par l'absurde que \(Q(\alpha) = 0\), on a alors:
      \[
         P = (X - \alpha)^mQ \text{ avec } Q = (X-\alpha)Q'
      \]
      Et donc \((X-\alpha)^{m+1} \mid P\) ce qui est absurde, donc \(Q(\alpha \neq 0)\)\<

      Réciproquement si il existe \(Q \in \K[X] \backslash \, \{0\}\) tel que \(P = (X-\alpha)^mQ\), alors on voit directement que \((X-\alpha)^m\) divise \(P\) et si \((X-\alpha)^{m+1}\) divisait \(P\), on aurait pour un certain \(Q' \in \K[X]\):
      \[
         P = (X - \alpha)^m (X-\alpha)Q'   
      \]
      Et donc \(Q = (X-\alpha)Q' \) et en particulier \(\alpha\) serait une racine de \(Q\) ce qui est absurde.
   \end{proof}
   
   \subsection*{\subsecstyle{21 - Caractérisation de la multiplicité II}}   

   \subsection*{\subsecstyle{22 - Les polynomes complexes sont scindés}}
   \begin{proof}[\unskip\nopunct]
      Soit \(P \in \C[X]\) un polynome complexe, montrons par réccurence sur le degré de \(P\) qu'il est scindé, ie montrons que:
      \[
         \forall n \in \N ; \exists a  \in \C ; \exists z_1, \ldots z_n \in \C^n; P_n = a(X-z_1)\ldots (X-z_n)
      \]
      \underline{Initialisation:} 
      
      On considère que \(P\) est de degré 1, alors d'aprés le théorème de d'Alemenbert-Gauss, il admet une racine \(z_1\) et donc \(P = a(X-z_1)\) pour un certain \(a\) non-nul.\<

      \underline{Hérédité:} 
      
      Supposons \(P_k\) vraie pour \(k \geq 0\), ie on suppose que tout polynome de degré \(k\) s'écrit sous la forme \(a(X-z_1)\ldots(X-z_k)\).\<

      Soit \(P_{k+1}\) un polynome de degré \(k+1\), d'aprés le théorème de d'Alembert-Gauss, il admet une racine qu'on notera \(z_{k+1}\), et on a alors:
      \[
         P_{k+1} = (X - z_{k+1})Q
      \]
      Avec \(Q\) un polynome de degré \(k\), alors d'aprés l'hypothèse de récurrence, on a:
      \[
         Q = a(X-z_1)\ldots(X-z_k) 
      \]
      Et donc par suite:
      \[
         P_{k+1} = a(X-z_1)\ldots(X-z_{k+1}) 
      \]
   \end{proof}

\end{document}