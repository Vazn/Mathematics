\documentclass{report}
% Maths Packages
\usepackage{mathtools, amsthm, amssymb, mathrsfs, interval, stmaryrd, centernot, esvect, cancel, commath, blkarray, empheq}
\usepackage{tabularx}
\usepackage{booktabs}
\usepackage{cellspace}
\setlength{\cellspacetoplimit}{5pt}
\setlength{\cellspacebottomlimit}{5pt}


% Sagemaths Formating Packages
\usepackage{listings}
\lstdefinelanguage{Sage}[]{Python}
{morekeywords={False,sage,True},sensitive=true}
\lstset{
  frame=none,
  showtabs=False,
  showspaces=False,
  showstringspaces=False,
  commentstyle={\ttfamily\color{dgreencolor}},
  keywordstyle={\ttfamily\color{dbluecolor}\bfseries},
  stringstyle={\ttfamily\color{dgraycolor}\bfseries},
  language=Sage,
  basicstyle={\fontsize{10pt}{10pt}\ttfamily},
  aboveskip=0.4em,
  belowskip=0.4em,
}

% TOC Packages
\usepackage{tocloft, titletoc, hyperref, bookmark}
% Formatting / Style Packages
\usepackage[T1]{fontenc}
\usepackage{geometry, subcaption, graphicx, fix-cm, accents, float, varwidth, soul, ulem, contour, multicol, enumitem}    
\usepackage[bottom]{footmisc}
\usepackage[x11names, table]{xcolor}
\usepackage[most, skins]{tcolorbox}
\usepackage{adjustbox}
\DeclareMathAlphabet{\mathmybb}{U}{bbold}{m}{n} % Indicatrices
\newcommand{\1}{\mathmybb{1}}

% Tikz
\usepackage{tikz, tkz-fct, tkz-euclide, tikz-cd, tkz-fct, pgfplots}
\pgfplotsset{compat=1.18}
\usetikzlibrary{
  angles, quotes, 3d, positioning,
  shapes,fit, arrows, arrows.meta, calc, 
  matrix, calligraphy, intersections, 
  quotes, patterns, patterns.meta, 
  decorations.pathreplacing, decorations.markings,decorations.pathmorphing,
}
\usepgfplotslibrary{fillbetween}
\tikzset{
  withparens/.style = {draw, outer sep=0pt,
    left delimiter= (, right delimiter=),
    above delimiter= (, below delimiter=),
    align=center},
  withbraces/.style = {draw, outer sep=0pt,
    left delimiter=\{, right delimiter=\},
    above delimiter=\{, below delimiter=\},
    align=center}
}
\tikzcdset{
  arrow style=tikz,
  diagrams={>={Straight Barb[scale=1]}},
}

% PAGE SETTINGS

\geometry{
  left=25mm, right=25mm, top= 15mm, bottom= 15mm,
  footskip=30pt
  }
\setlength{\parindent}{0cm}
\setlength{\parskip}{0cm}
\setlist[itemize]{itemsep=0pt, leftmargin=25pt}

\setlength{\cftbeforetoctitleskip}{0pt}
\setlength{\cftaftertoctitleskip}{0pt}

\setcounter{secnumdepth}{-1}

% STYLE
\definecolor{BrightBlue1}{RGB}{95, 150, 210}

\definecolor{BrightRed1}{RGB}{210, 95, 95}
\definecolor{BrightRed2}{RGB}{210, 115, 115}

\definecolor{DarkBlueX}{RGB}{43, 68, 92}
\definecolor{DarkBlue0}{RGB}{53, 78, 102}
\definecolor{DarkBlue1}{RGB}{83, 108, 132}
\definecolor{DarkBlue2}{RGB}{58, 94, 132}
\definecolor{DarkBlue3}{RGB}{90, 126, 162}

\definecolor{DarkGreen3}{RGB}{83, 132, 108}
\definecolor{DarkGreen2}{RGB}{58, 132, 94}
\definecolor{DarkGreen1}{RGB}{90, 162, 126}
\tcbset{shield externalize, enhanced, sharp corners, halign=center, center}

\definecolor{dblackcolor}{rgb}{0.0,0.0,0.0}
\definecolor{dbluecolor}{rgb}{0.01,0.02,0.7}
\definecolor{dgreencolor}{rgb}{0.2,0.4,0.0}
\definecolor{dgraycolor}{rgb}{0.30,0.3,0.30}
\newcommand{\dblue}{\color{dbluecolor}\bf}
\newcommand{\dred}{\color{dredcolor}\bf}
\newcommand{\dblack}{\color{dblackcolor}\bf}

%Underline settings
\setlength{\ULdepth}{2pt}
\contourlength{0.8pt}
\renewcommand{\underline}[1]{
  \uline{\phantom{#1}}%
  \llap{\contour{white}{#1}}%
}

%drop shadow southwest=black!100!black
\newcommand{\secstyle}[1]{\color{DarkBlue1}\fbox{#1}}
\newcommand{\subsecstyle}[1]{\color{DarkBlue2}\underline{#1}}
\newcommand{\subsubsecstyle}[2]{\color{DarkBlue3}\underline{#1}}

\newcommand{\chapterstyle}[1]{
    \setlength{\fboxsep}{0.3em}
    \setlength{\fboxrule}{3pt}
    \centering\vspace{-70pt}
    
    \color{DarkBlue1}\huge\fbox{\textbf{\textsc{#1}}}
}

\newcommand{\customBox}[2]{
    \tcbset{boxrule=1.5pt, boxsep=-0.2mm, colframe=DarkBlue1, colback=BrightBlue1!05}
    \begin{tcolorbox}[#1]
        \abovedisplayskip=0pt % remove vertical space above align
        #2
    \end{tcolorbox}
}

\makeatletter % Crée une trés grosse taille de police pour la page de garde
\newcommand\HUGE{\@setfontsize\Huge{40}{60}}
\makeatother   

\makeatletter
\newcommand\footnoteref[1]{\protected@xdef\@thefnmark{\ref{#1}}\@footnotemark}
\makeatother

% COMMANDS
% TOC
\renewcommand{\cftchapfont}{\large \bfseries \scshape}
\renewcommand{\cftsecfont}{}
\renewcommand{\contentsname}{\hfill
\setlength{\fboxsep}{0.3em}\setlength{\fboxrule}{3pt}\vspace{20pt}
   \color{DarkBlue1}\Huge
   \fbox{\textbf{\textsc{Table des matières}}}
   \hfill
}

% MATHS
\newcommand{\C}{\mathbb{C}}
\newcommand{\R}{\mathbb{R}}
\newcommand{\Q}{\mathbb{Q}}
\newcommand{\Z}{\mathbb{Z}}
\newcommand{\N}{\mathbb{N}}
\newcommand{\U}{\mathbb{U}}
\newcommand{\K}{\mathbb{K}}

\newcommand{\A}{\mathbf{\mathscr{A}}}
\newcommand{\B}{\mathbf{\mathscr{B}}}
\newcommand{\Fam}{\mathbf{\mathscr{F}}}
\renewcommand{\P}{\mathbf{\mathscr{P}}}

\renewcommand{\epsilon}{\varepsilon}
\renewcommand{\rho}{\varrho}

\newcommand{\E}{\mathbf{\mathcal{E}}}
\newcommand{\F}{\mathbf{\mathcal{F}}}
\newcommand{\Pow}{\mathbf{\mathcal{P}}}
\newcommand{\G}{\mathbf{\mathfrak{G}}}

\newcommand{\<}{\bigskip}
\newcommand{\+}{\par}

% Notation equality

\newcommand\notationEq{\stackrel{\mbox{
    \begin{tiny}  
        notation
    \end{tiny}    
}}{=}}

% INTERVALS

\intervalconfig{separator symbol =  \,; \,}

\newcommand{\ioo}[2]{\interval[open]{#1}{#2}}
\newcommand{\ioc}[2]{\interval[open left]{#1}{#2}}
\newcommand{\ico}[2]{\interval[open right]{#1}{#2}}
\newcommand{\icc}[2]{\interval{#1}{#2}}

\newcommand{\intioo}[2]{\left\rrbracket{#1}\;;\;{#2}\right\llbracket}
\newcommand{\intioc}[2]{\left\rrbracket{#1}\;;\;{#2}\right\rrbracket}
\newcommand{\intico}[2]{\left\llbracket{#1}\;;\;{#2}\right\llbracket}
\newcommand{\inticc}[2]{\left\llbracket{#1}\;;\;{#2}\right\rrbracket}

% EQUATIONS NOTES

\newcommand{\shorteqnote}[1]{ &  & \text{\small\llap{#1}}}
\newcommand{\longeqnote}[1]{& & \\ \notag&  &  &  &  & \text{\small\llap{#1}}}

% FUNCTIONS NOTATIONS

\newcommand{\inject}{\hookrightarrow} 
\newcommand{\surject}{\twoheadrightarrow}

% MOD NOTATION

\newcommand{\eqmod}[1]{\underset{#1}{\equiv}} 

% LINEAR ALGEBRA

\newcommand{\dotproduct}[2]{\left\langle\;\! #1 \;\! | \;\! #2 \;\! \right\rangle}
\newcommand{\vectNorm}[1]{\left\Vert#1 \right\Vert}

\newcommand{\Ker}[1]{\text{Ker}#1}
\newcommand{\Sp}[1]{\text{Sp}(#1)}
\renewcommand{\Im}[1]{\text{Im}#1}

\NewDocumentCommand{\opNorm}{sO{}m}{%
  \IfBooleanTF{#1}{% automatic scaling, use with care
    \left|\opnormkern\left|\opnormkern\left|
    #3
    \right|\opnormkern\right|\opnormkern\right|
  }{
    \mathopen{#2|\opnormkern #2|\opnormkern #2|}
    #3
    \mathclose{#2|\opnormkern #2|\opnormkern #2|}
  }%
}
\newcommand{\opnormkern}{\mkern-1.5mu\relax}% adjust for the font

% TOPOLOGY
\newcommand{\ball}{\mathscr{B}}

% CALCULUS
\newcommand{\partialD}[2]{\frac{\partial #1}{\partial #2}}

% GEOMETRY
\newcommand{\RightAgnle}[4][5pt]
{%
    \draw($#3!#1!#2$)-- ($#3!2! ($ ($#3!#1!#2$)!.5! ($#3!#1!#4$)$)$)-- ($#3!#1!#4$);
}

% PROBABILITIES
\newcommand{\probability}[1]{\mathbb{E} (#1)}
\newcommand{\expectancy}[1]{\mathbb{E} (#1)}
\newcommand{\variance}[1]{\mathbb{V} (#1)}
\newcommand{\covariance}[1]{\mathbb{C} (#1)}


\begin{document}
   \chapter*{\chapterstyle{Théorie des groupes}}
      \subsection*{\subsecstyle{Sous-groupes engendré:}}
         On se propose de montrer que le sous-groupe de \(G\) engendré par une partie \(P\) défini ci-dessous est bien \textbf{le plus petit sous-groupe qui contient \(P\)}:
         \[
            H = \langle P \rangle := \Bigl\{ h_1^{k_1}h_2^{k_2} \ldots h_n^{k_n}\; ; \; n \in \N \; , \; h_i \in H \; , \; k_i \in \Z \Bigl\}
         \]
         On montre facilement que le plus petit sous-groupe qui contient \(P\) est l'intersection ci-dessous, reste à montrer l'égalité suivant:
         \[
            H = \bigcap_{\substack{P \subseteq Q\\Q < G}} Q
         \]
         On appelle \(\widetilde{H}\) l'interesection ci-dessus puis on procède par double inclusion:
         \begin{itemize}
            \item Le sous-groupe \(H\) est un sous-groupe qui contient \(P\) trivialement, donc en particulier, on a par les propriétés de l'intersection que \(\widetilde{H} \subseteq H\).
            \item Le sous-groupe \(\widetilde{H}\) contient tout les élément de \(P\), donc par stabilité, il contient nécéssairement toutes les combinaisons trouvées dans \(H\), ie on a bien \(H \subseteq \widetilde{H}\).
         \end{itemize}
      \subsection*{\subsecstyle{Sous-groupes monogène d'un groupe fini:}}
      On se donne \(G\) un groupe fini, et \(g \in G\), alors il existe un entier positif tel que:
      \[
         g^k = 1
      \]
      En effet l'ensemble \(\{g^k, k \in \N\}\) ne prends qu'un nombre fini de valeurs, donc il existe nécessairement deux indices \(
      n > p\) tel que:
      \[
         g^{n} = g^{p}
      \]
      Et donc \(g^{n-p} = 1\)
      \subsection*{\subsecstyle{Image d'une partie génératrice:}}
      On se donne un groupe \(G = \langle P \rangle \) pour \(P\) une partie de \(G\), ainsi qu'un morphisme \(\phi : G \longrightarrow H\), on veut alors montrer que l'image d'une partie génératrice est génératrice, ie que:
      \[
         f(G) = \langle f(P) \rangle
      \] 
      On se donne un élément \(f(g) \in  f(G)\), alors on a que \(g\) se décompose en produit \textbf{fini} de puissances d'éléments de \(P\) et donc on a:
      \[
         f(g) = f\left(\prod_{p_{i} \in P} p_i^{k_i}\right) = \prod_{p_{i} \in P} f(p_i)^{k_i}
      \]
      Fininalement on a bien que \(f(g)\) s'écrit comme produit \textbf{fini} de puissances d'éléments de \(f(P)\) donc on a bien le résultat.
      \subsection*{\subsecstyle{Théorème de Cayley:}}
         On se propose de démontrer que tout groupe \(G\) est \textbf{isomorphe à un groupe d'automorphismes}, on définit l'application suivante:
         \[
            \begin{aligned}
               \Phi : G &\longrightarrow \text{Aut}(G, G)\\
               g &\longmapsto \phi_g
            \end{aligned}
         \]
         Où \(\phi_g(x) = gx\), ie on associe à chaque élément de \(G\) la fonction qui translate tout les éléments de \(G\) par celui-ci, alors:
         \begin{itemize}
            \item Cette application est facilement un \textbf{morphisme de groupes}.
            \item Cette application est facilement \textbf{injective}.
         \end{itemize}
         Finalement, on a bien que \(G \cong \text{Im}(\Phi) < \text{Aut}(G, G)\) comme souhaité.
      \subsection*{\subsecstyle{Sous-groupes de \(\Z\):}}
         On se propose de montrer la propriété suivante:
         \[
            H < \Z \Longleftrightarrow \exists n \in \Z \; ; \; H = n\Z
         \]
         On se donne un sous groupe \(H < \Z\) et on distingue deux cas:
         \begin{itemize}
            \item Si \(H\) est trivial, alors \(H = 0\Z\) qui convient.
            \item Sinon \(H\) contient un élément non-nul, donc un élément \textbf{positif} \(h\).
         \end{itemize}
         Dans ce cas, la partie \(H \cap \N^*\) est non-vide, et donc elle contient un plus petit élément qu'on notera \(n\). On va alors montrer que tout élement du groupe est un multiple de cet élément.\<

         En effet soit \(h \in H\), alors on effectue la division euclidienne de \(h\) par \(n\) et on obtient:
         \[
            h = nq + r \; ; \; r \in \inticc{0}{n-1}
         \]
         Mais alors par construction on a \(r < n\) et \(r \in H \cap \N^*\) par opérations, donc nécéssairement \(r = 0\), car sinon, \(n\) ne serait pas le plus petit élément de la partie considérée.
      \subsection*{\subsecstyle{Classificiation des groupes cycliques:}}
         On se propose de classifier les groupes cycliques \(G\), engendrés pas un élément \(g \in G\), alors on définit:
         \[
            \begin{aligned}
               \phi : \Z &\longrightarrow G\\
               n &\longmapsto g^n
            \end{aligned}
         \]
         Alors cette application est \textbf{un morphisme de groupe surjectif}, aussi le noyau de \(\phi\) est un sous-groupe normal, et on sait que les sous-groupes de \(\Z\) sont de la forme \(n\Z\), on distingue alors deux cas:
         \begin{itemize}
            \item Soit \(\Ker\phi\) est trivial et alors \(\phi\) est un isomorphisme: \(G \cong \Z\).
            \item Soit \(\Ker\phi = n\Z\) et alors \(\phi\) induit un isomorphisme par passage au quotient: \(G \cong \Z/n\Z\)
         \end{itemize} 
         Finalement on a bien montré qu'il n'y a bien qu'un seul groupe cyclique d'ordre \(n\) qui est \(\Z/n\Z\) et qu'un seul groupe cyclique d'ordre infini qui est \(\Z\).

   \chapter*{\chapterstyle{Groupe Symétrique}}
      On note $S_n$ l'ensemble des bijections sur $\{1, \ldots, n\}$ et on montre facilement que c'est un groupe pour la composition.
      \subsection*{\subsecstyle{Cardinal de $S_n$}}
         Montrons par récurrence sur $n$ la propriété suivante:
         $$
            |A| = |B| = n \implies \left|\text{Bij}(A, B)\right| = n!
         $$
         L'initialisation est triviale, en effet si les deux ensembles sont vides, il n'existe qu'une seule application bijective du vide dans lui-même.\\
      
         Soit $\phi$ une bijection de $A'$ dans $B'$ ensembles de cardinal \(n+1\), alors il existe $n + 1$ choix possibles pour l'image de l'élément $a_1$, on considère alors la restriction:
         \[
            \begin{aligned}
               \widetilde{\phi}: A' \backslash \{a_1\} &\longrightarrow B'\\
               x &\longmapsto \phi(x) 
            \end{aligned}
         \]
         Alors $\phi(a_1)$ n'est pas dans l'image de cette fonction car sinon cela contredirait l'injectivité de $\phi$, on peut donc restreindre à l'arrivée en une fonction bijective:
         \[
            \begin{aligned}
               \widetilde{\phi}: A' \backslash \{a_1\} &\longrightarrow B'\backslash \{\phi(a_1)\}\\
               x &\longmapsto \phi(x) 
            \end{aligned}
         \]
         C'est une fonction bijective entre deux ensembles de cardinal $n$, il ya donc par hypothèse de récurrence $n!$ choix possibles pour une telle fonction. Finalement par dénombrement, on a bien qu'il y a $(n +1)n! = (n+1)!$ choix possibles de fonctions bijectives entre ensembles de taille $n+1$.\\
      
         On conclut sur le groupe symétrique en prenant $A = B = \{1, \ldots, n\}$
      \subsection*{\subsecstyle{Isomorphisme sur les ensembles de même cardinal}}
         On se donne $X$ un ensemble de cardinal $n$, montrons que $S(X) \cong S_n$.
         On sait qu'il y a une bijection $i$ entre $X$ et $\{1, \ldots, n\}$ donnée par:
         $$
            i : k \in \{1, \ldots, n\} \mapsto x_k \in X
         $$
         On pose $\sigma \in S_n$ et $\tau \in S(X)$ et on considère le diagramme commutatif suivant:
         $$
            \begin{tikzcd}
               \{1, \ldots, n\} \arrow[r, "\sigma"] \arrow[d, "i"]
               & \{1, \ldots, n\}\\
               X \arrow[r, "\tau"]
               & X \arrow[u, "i^{-1}"]
            \end{tikzcd}
         $$
         Ceci nous donne directement l'isomorphisme, en effet on pose:
         $$
            \phi : \tau \in S(X) \mapsto i^{-1} \circ \tau \circ i \in S_n
         $$
         Alors $\phi$ est bien une bijection de $S(X)$ sur $S_n$, et on montre facilement que c'est un morphisme car:
         $$
            \phi(\tau_1\tau_2) = i^{-1}(\tau_1\tau_2)i = i^{-1}\tau_1 ii^{-1}\tau_2i = \phi(\tau_1)\phi(\tau_2)
         $$
         \pagebreak 
      \subsection*{\subsecstyle{Non-commutativité}}
         On se propose de montrer que $S_n$ n'est pas commutatif pour $n \geq 3$, on montre pour cela que son centre est trivial ie:
         $$
            Z(S_n) = \{Id\}
         $$
         Supposons par l'absurde qu'il soit non-trivial, alors il existe une permutation $\sigma \neq Id$ qui commute avec toutes les autres. Elle admet au moins un point $x$ tel que $\sigma(x) \neq x$, mais alors il y a 3 éléments distincts dans $S_n$ donc il existe $y \in \{1, \ldots, n\}$ tel que:
         \begin{align*}
            \begin{cases}
               y \neq x\\
               y \neq \sigma(x)
            \end{cases}
         \end{align*}
         On définit alors $\sigma'$ comme une permutation telle que:
         $$
            \begin{cases}
               \sigma'(x) = x\\
               \sigma'(\phi(x)) = y
            \end{cases}
         $$
         Alors on a que nécessairement:
         $$
            \begin{cases}
               \sigma\sigma'(x) = \sigma(x)\\
               \sigma'\sigma(x) = y\\
            \end{cases}
         $$
         Ce qui est absurde car $\sigma$ devrai commuter, donc le centre est bien trivial.    
      \subsection*{\subsecstyle{Propriétés du support}}
         \begin{enumerate}
            \item On considère $\sigma, \tau$ deux permutations de $S_n$, montrons que:
            $$
               \text{Supp}(\sigma\tau) \subseteq \text{Supp}(\sigma) \cup \text{Supp}(\tau)
            $$
            Passons au complémentaire, on obtient que la propriété est équivalente à:
            $$
               \text{Fix}(\sigma) \cap \text{Fix}(\tau) \subseteq \text{Fix}(\sigma\tau)
            $$
            Qui est trivialement vérifiée.
            \item Un lemme utile pour la suite:
            $$
               x \in \text{Supp}(\sigma) \implies \sigma(x) \in \text{Supp}(\sigma)
            $$
            C'est évident car sinon \(\sigma\) ne serait pas injective.
            \item Supposons maintenant que:
            $$
               \text{Supp}(\sigma) \cap \text{Supp}(\tau) = \emptyset
            $$
            Alors on a égalité, en effet si $x$ est fixé par $\tau$ mais pas par $\sigma$, alors on a:
            $$
               \sigma\tau(x) = \sigma(x) \neq x
            $$
            Donc il est bien dans le support du produit, et si $x$ est fixé par $\sigma$ mais pas par $\tau$, alors on a:
            $$
               \sigma\tau(x) = \tau(x) \neq x
            $$
            Car $\tau(x)$ est dans le support de $\tau$ par le lemme, et donc est fixé par $\sigma$.
         \end{enumerate}
          
         \pagebreak        
      \subsection*{\subsecstyle{Les permutations à supports disjoints commutent}}
         Soit deux permutations $\sigma, \tau$ à supports disjoints et $x \in \{1, \ldots, n\}$, alors on distingue deux cas:
         \begin{itemize}
            \item Si $x$ est dans le support de $\tau$, alors il est fixé par $\sigma$ par hypothèse et \(\tau(x)\) est fixé par \(\sigma\) par le lemme précédent donc on a:
            $$
               \tau(\sigma(x)) = \tau(x) \; \text{ et } \; \sigma(\tau(x)) = \tau(x)
            $$
            \item Si $x$ n'est pas dans le support de $\tau$, alors par symétrie, il est dans le support de $\sigma$ et on conclut par le même calcul.
         \end{itemize}
         En particulier si on a:
         $$
            \sigma\tau = \text{Id}
         $$
         Alors les deux permutations sont l'identité car:
         \begin{itemize}
            \item Si $x$ est dans le support de $\tau$, alors il est fixé par $\sigma$ mais alors on a une absurdité, donc le support est vide.
            \item Si $x$ n'est pas dans le support de $\tau$, alors par symétrie, il est dans le support de $\sigma$ mais alors on a aussi une absurdité, donc le support est vide.
         \end{itemize}     
      \subsection*{\subsecstyle{Décomposition en cycles à supports disjoints}}
         Soit \(\sigma \in S_n\), alors on définit l'ensemble suivant appelé $\sigma$-orbite de \(x \in \inticc{1}{n}\) par:
         \[
            \mathcal{O}_x := \{\sigma^k(x)\; ; \; k \in \N\}
         \]
         Alors par un simple raisonnement sur l'ordre et les propriétés de cet ensemble on montre facilement que:
         \[
            \forall x, y \in \inticc{1}{n} \; ; \; \mathcal{O}_x \cap \mathcal{O}_y = \emptyset \text{ ou } \mathcal{O}_x = \mathcal{O}_y
         \]
         En outre si on définit la relation d'équivalence sur \(\inticc{1}{n}\) par:
         \[
            x \sim y \Longleftrightarrow \mathcal{O}_x = \mathcal{O}_y
         \]
         On peut alors \textbf{partitionner} \(\inticc{1}{n}\) en \(k \leq n\) orbites, et donc il existe une famille de représentants \((x_1, \ldots x_k) \in \inticc{1}{n}\) tels que:
         \[
            \inticc{1}{n} = \bigcup_{i \leq k} \mathcal{O}_{x_k}
         \]
         On pose alors:
         \[
            \tau = (x_1 \dots \sigma^{l_1}(x_1))\ldots(x_k \dots \sigma^{l_k}(x_k)) = c_1\ldots c_k
         \]
         Alors les supports de ces cycles sont exactement les orbites prédéfinies et il suffit de montrer que \(\sigma = \tau\):
         \begin{itemize}
            \item Si \(x \notin \text{Supp}(\sigma)\) alors son orbite est un singleton et \(\sigma(x) = \tau(x) = x\)
            \item Si \(x \in \text{Supp}(\sigma)\) alors il appartient à un unique cycle \(c_i\) et on a \(\tau(x) = \sigma(x)\) par définition de ces cyles.
         \end{itemize}
         \pagebreak     
      \subsection*{\subsecstyle{Ordre d'une permutation}}
         Sois $\sigma$ une permutation sur $\{1, \ldots, n\}$, alors on a d'aprés la propriété précédente que pour \(k \leq n\) il existe une famille \((c_j)_{j \leq k}\) de cycles à supports disjoints tels que:
         $$
            \sigma = c_1 \ldots c_k
         $$
         Soit $i\in \mathbb{N}$, alors on sait que les supports sont disjoints donc on a:
         \begin{align}
            \sigma^i = \text{Id} \implies c_1^i \ldots c_k^i = \text{Id} \implies \forall c \in (c_j) \; ; \; i \in \text{ord}(c)\N
         \end{align}
         En d'autres termes \(i\) doit être un multiple de l'ordre de chaque cycle, en effet supposons l'inverse ie qu'il existe un cycle \(c \in (c_j)\) tel que \(\text{ord}(c) \nmid k\), alors pour \(x\) dans le support de ce cycle, on aurait:
         \[
            \sigma^i(x) = c^i(x) \neq x \implies \sigma^i \neq \text{Id}
         \]
         En reformulant la dernière partie de \((1)\), on trouve donc que:
         \[
            \sigma^i = \text{Id} \implies i \in \bigcap_{c \in (c_j)} \text{ord}(c)\N
         \]
         Or le plus petit \(i\) qui vérifie \(\sigma^i = \text{Id}\) est donc le plus petit élément de l'ensemble ci-dessus des multiples communs des ordres, qui est par définition le plus petit commun multiple des ordres des \((c_j)\).    
      \subsection*{\subsecstyle{Conjugaison d'un cycle}}
         On considère un cycle $c = (a_1 \ldots a_k)$ de $S_n$, ainsi qu'une permutation quelconque $\sigma$. Cherchons à caractériser la permutation $\tau$ suivante:
         $$
            \tau = \sigma (a_1 \ldots a_k) \sigma^{-1}
         $$
         Soit $x \in \{1, \ldots, n\}$, alors on considère deux cas:
         \begin{itemize}
            \item Si $\sigma^{-1}(x)$ n'est pas dans le support de $c$.
            \item Si $\sigma^{-1}(x)$ est dans le support de $c$.
         \end{itemize}
         Le premier cas est trivial\footnote{En effet si $c \sigma^{-1}(x) = \sigma^{-1}(x)$, alors on a directement que $\tau(x) = x$}, examinons donc le second, on a tout d'abord que:
         \begin{align}
               \exists i \in \{1, \ldots, k\} \; ; \; \sigma^{-1}(x) = a_i
         \end{align}
         Or par définition d'un cycle on a que:
         $$
            c\sigma^{-1}(x) = a_{i+1}
         $$
         Et donc finalement:
         $$
            \tau(x) = \sigma(a_{i+1})
         $$
         Mais en reprenant l'équation obtenue en (2), on obtient aussi que:
         $$
            x = \sigma(a_i)
         $$
         Donc finalement on trouve que sur le support de $\tau$ est exactement l'image des $(a_i)$ par $\sigma$ et que la permutation effectuée correspond au cycle:
         $$
            \tau = (\sigma(a_1) \ldots \sigma(a_k))
         $$
         Le conjugué d'un cycle est donc un cycle de même longueur. En outre si deux permutations sont conjuguées, alors elle sont de même type, en effet si:
         \[
            \sigma' = \tau\sigma\tau^{-1}
         \]       
\end{document}