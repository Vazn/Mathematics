\documentclass{report}
% Maths Packages
\usepackage{mathtools, amsthm, amssymb, mathrsfs, interval, stmaryrd, centernot, esvect, cancel, commath, blkarray, empheq}
\usepackage{tabularx}
\usepackage{booktabs}
\usepackage{cellspace}
\setlength{\cellspacetoplimit}{5pt}
\setlength{\cellspacebottomlimit}{5pt}

% Sagemaths Formating Packages
\usepackage{listings}
\lstdefinelanguage{Sage}[]{Python}
{morekeywords={False,sage,True},sensitive=true}
\lstset{
  frame=none,
  showtabs=False,
  showspaces=False,
  showstringspaces=False,
  commentstyle={\ttfamily\color{dgreencolor}},
  keywordstyle={\ttfamily\color{dbluecolor}\bfseries},
  stringstyle={\ttfamily\color{dgraycolor}\bfseries},
  language=Sage,
  basicstyle={\fontsize{10pt}{10pt}\ttfamily},
  aboveskip=0.4em,
  belowskip=0.4em,
}

% TOC Packages
\usepackage{tocloft, titletoc, hyperref, bookmark}
% Formatting / Style Packages
\usepackage[T1]{fontenc}
\usepackage{geometry, subcaption, graphicx, fix-cm, accents, float, varwidth, soul, ulem, contour, multicol, enumitem}    
\usepackage[bottom]{footmisc}
\usepackage[x11names, table]{xcolor}
\usepackage[most, skins]{tcolorbox}
\usepackage{adjustbox}
\DeclareMathAlphabet{\mathmybb}{U}{bbold}{m}{n} % Indicatrices
\newcommand{\1}{\mathmybb{1}}

% Tikz
\usepackage{tikz, tkz-fct, tkz-euclide, tikz-cd, tkz-fct, pgfplots}
\pgfplotsset{compat=1.18}
\usetikzlibrary{
  angles, quotes, 3d, positioning,
  shapes,fit, arrows, arrows.meta, calc, 
  matrix, calligraphy, intersections, 
  quotes, patterns, patterns.meta, 
  decorations.pathreplacing, decorations.markings,decorations.pathmorphing,
}
\usepgfplotslibrary{fillbetween}
\tikzset{
  withparens/.style = {draw, outer sep=0pt,
    left delimiter= (, right delimiter=),
    above delimiter= (, below delimiter=),
    align=center},
  withbraces/.style = {draw, outer sep=0pt,
    left delimiter=\{, right delimiter=\},
    above delimiter=\{, below delimiter=\},
    align=center}
}
\tikzcdset{
  arrow style=tikz,
  diagrams={>={Straight Barb[scale=1]}},
}

% PAGE SETTINGS

\geometry{
  left=25mm, right=25mm, top= 15mm, bottom= 15mm,
  footskip=30pt
  }
\setlength{\parindent}{0cm}
\setlength{\parskip}{0cm}
\setlist[itemize]{itemsep=0pt, leftmargin=25pt}

\setlength{\cftbeforetoctitleskip}{0pt}
\setlength{\cftaftertoctitleskip}{0pt}


\begin{document}
   \begin{proof}[\unskip\nopunct]
      Soit l'application suivante de la sphère unité à valeurs scalaires:
      \[
         \begin{aligned}
            \phi : S^2 &\longrightarrow \R \\
            x &\longmapsto \phi(x) 
         \end{aligned}
      \]
      Montrons qu'il existe deux points \(x, x'\) antipodaux de \(S^2\) tels que \(\phi(x) = \phi(x')\). Soit \(x\) un point de la sphère, définissons la propriété d'un point \(x'\) d'être "antipodal" à \(x\).\<

      On considère la symétrie vectorielle \(s\) par rapport au vecteur nul, alors c'est une isométrie, on appelle alors point antipodal à \(x\) le point \(s(x) \in S^2\) (bien défini car \(s\) est une isométrie).\<

      Montrons alors qu'il existe un \(x \in S^2\) tel que l'application suivante s'annule en ce point:
      \[
         \begin{aligned}
            \psi : S^2 &\longrightarrow \R  \\
            x &\longmapsto \phi(x) - \phi(s(x))
         \end{aligned}
      \]
      En effet si il existe un tel \(x\), on aura bien montré que \(x, s(x)\) sont bien deux points antipodaux qui prennent la même valeur. On peut alors affirmer:
      \begin{itemize}
         \item \(\psi\) est continue comme composée et somme d'application continues sur \(S^2\).
         \item \(S^2\) est connexe.
      \end{itemize}
      Donc en particulier \(\psi(S^2)\) est un connexe de \(\R\) donc un intervalle. Montrons qu'il contient \(0\).\<

      Supposons que \(\psi(S^2) \subseteq \R_{-}^*\), alors si on applique cette propriété à \(x\) et \(s(x)\), on a:
      \[
         \begin{cases}
            \phi(x) - \phi(s(x)) < 0\\
            \phi(s(x)) - \phi(s(s(x))) < 0
         \end{cases} \Longleftrightarrow 
         \begin{cases}
            \phi(x) - \phi(s(x)) < 0\\
            \phi(x) - \phi(s(x)) > 0
         \end{cases} 
      \]
      Ce qui est absurde, et symétriquement, on montre que \(\psi(S^2) \not\subseteq \R_{+}^*\), donc \(\psi\) s'annule, et il existe donc bien deux points antipodaux qui ont même valeur pour \(\phi\).
   \end{proof}
\end{document}