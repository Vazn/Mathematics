\documentclass{report}
% Maths Packages
\usepackage{mathtools, amsthm, amssymb, mathrsfs, interval, stmaryrd, centernot, esvect, cancel, commath, blkarray, empheq}
\usepackage{tabularx}
\usepackage{booktabs}
\usepackage{cellspace}
\setlength{\cellspacetoplimit}{5pt}
\setlength{\cellspacebottomlimit}{5pt}

% Sagemaths Formating Packages
\usepackage{listings}
\lstdefinelanguage{Sage}[]{Python}
{morekeywords={False,sage,True},sensitive=true}
\lstset{
  frame=none,
  showtabs=False,
  showspaces=False,
  showstringspaces=False,
  commentstyle={\ttfamily\color{dgreencolor}},
  keywordstyle={\ttfamily\color{dbluecolor}\bfseries},
  stringstyle={\ttfamily\color{dgraycolor}\bfseries},
  language=Sage,
  basicstyle={\fontsize{10pt}{10pt}\ttfamily},
  aboveskip=0.4em,
  belowskip=0.4em,
}

% TOC Packages
\usepackage{tocloft, titletoc, hyperref, bookmark}
% Formatting / Style Packages
\usepackage[T1]{fontenc}
\usepackage{geometry, subcaption, graphicx, fix-cm, accents, float, varwidth, soul, ulem, contour, multicol, enumitem}    
\usepackage[bottom]{footmisc}
\usepackage[x11names, table]{xcolor}
\usepackage[most, skins]{tcolorbox}
\usepackage{adjustbox}
\DeclareMathAlphabet{\mathmybb}{U}{bbold}{m}{n} % Indicatrices
\newcommand{\1}{\mathmybb{1}}

% Tikz
\usepackage{tikz, tkz-fct, tkz-euclide, tikz-cd, tkz-fct, pgfplots}
\pgfplotsset{compat=1.18}
\usetikzlibrary{
  angles, quotes, 3d, positioning,
  shapes,fit, arrows, arrows.meta, calc, 
  matrix, calligraphy, intersections, 
  quotes, patterns, patterns.meta, 
  decorations.pathreplacing, decorations.markings,decorations.pathmorphing,
}
\usepgfplotslibrary{fillbetween}
\tikzset{
  withparens/.style = {draw, outer sep=0pt,
    left delimiter= (, right delimiter=),
    above delimiter= (, below delimiter=),
    align=center},
  withbraces/.style = {draw, outer sep=0pt,
    left delimiter=\{, right delimiter=\},
    above delimiter=\{, below delimiter=\},
    align=center}
}
\tikzcdset{
  arrow style=tikz,
  diagrams={>={Straight Barb[scale=1]}},
}

% PAGE SETTINGS

\geometry{
  left=25mm, right=25mm, top= 15mm, bottom= 15mm,
  footskip=30pt
  }
\setlength{\parindent}{0cm}
\setlength{\parskip}{0cm}
\setlist[itemize]{itemsep=0pt, leftmargin=25pt}

\setlength{\cftbeforetoctitleskip}{0pt}
\setlength{\cftaftertoctitleskip}{0pt}


\begin{document}

   \subsection*{\subsecstyle{Théorème Chinois:}}
      On se propose de montre l'isomorphisme d'anneaux suivant, pour \((n_i)\) des nombres premiers entre eux et \(n\) le produit de ces nombres:
      \[
         \Z/{n\Z} = \Z/{n_1\Z} \times \ldots \times \Z/{n_k\Z}
      \]
      On considère l'application suivante:
      \[
         \begin{aligned}
            \phi : \Z/{n\Z} &\longrightarrow \Z/{n_1\Z} \times \ldots \times \Z/{n_k\Z}\\
            x[n] &\longmapsto (x[n_1], \ldots, x[n_k])
         \end{aligned}
      \]
      Alors, on montre facilement par les propriétés des congurences que \(\phi\) est \textbf{un morphisme d'anneaux}. Il reste à montrer qu'il est bijectif, pour cela on remarque tout d'abord que les deux ensembles (finis) ont le même cardinal, il reste alors à montrer l'injectivité.\<

      On se donne alors \(x\) tel que \((x[n_1], \ldots, x[n_k]) = 0\), alors \(x\) est divisible par tout les \(n_i\) et ils sont premiers entre eux, on en déduis donc que \(x\) est divisible par leur produit.

   \subsection*{\subsecstyle{Petit théorème de Fermat:}}
   On se propose de montrer la propriété suivante pour \(p\) premier et pour tout entier \(a\):
   \[
      a^p \equiv a[p]
   \]
   Par récurrence sur \(a\), on a directement l'initialisation et si on suppose le résultat vrai pour \(a \geq 0\), alors on a que:
   \[
      (a + 1)^p = a^p + \sum_{k=1}^{p-1}\binom{p}{k}a^k + 1
   \]
   Or, on peut alors montrer, en utilisant \textbf{la formule du capitaine} et le \textbf{lemme de Gauss} que si \(p\) est premier, on a:
   \[
      \forall k \in \inticc{1}{p - 1}\; ; \; p \; \Big| \binom{p}{k}
   \]
   Donc finalement les termes de la somme disparaissent modulo \(p\) et on obtient:
   \[
      (a + 1)^p  \equiv a^p + 1 \underset{HR}{\equiv} a + 1
   \]
   En particulier, si \(a\) est inversible modulo \(p\) (ie si \(a\) est premier avec \(p\)), alors on a:
   \[
      a^{p-1} \equiv 1[p]
   \]
\end{document}