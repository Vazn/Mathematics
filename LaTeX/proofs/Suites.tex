\documentclass{report}
% Maths Packages
\usepackage{mathtools, amsthm, amssymb, mathrsfs, interval, stmaryrd, centernot, esvect, cancel, commath, blkarray, empheq}
\usepackage{tabularx}
\usepackage{booktabs}
\usepackage{cellspace}
\setlength{\cellspacetoplimit}{5pt}
\setlength{\cellspacebottomlimit}{5pt}

% Sagemaths Formating Packages
\usepackage{listings}
\lstdefinelanguage{Sage}[]{Python}
{morekeywords={False,sage,True},sensitive=true}
\lstset{
  frame=none,
  showtabs=False,
  showspaces=False,
  showstringspaces=False,
  commentstyle={\ttfamily\color{dgreencolor}},
  keywordstyle={\ttfamily\color{dbluecolor}\bfseries},
  stringstyle={\ttfamily\color{dgraycolor}\bfseries},
  language=Sage,
  basicstyle={\fontsize{10pt}{10pt}\ttfamily},
  aboveskip=0.4em,
  belowskip=0.4em,
}

% TOC Packages
\usepackage{tocloft, titletoc, hyperref, bookmark}
% Formatting / Style Packages
\usepackage[T1]{fontenc}
\usepackage{geometry, subcaption, graphicx, fix-cm, accents, float, varwidth, soul, ulem, contour, multicol, enumitem}    
\usepackage[bottom]{footmisc}
\usepackage[x11names, table]{xcolor}
\usepackage[most, skins]{tcolorbox}
\usepackage{adjustbox}
\DeclareMathAlphabet{\mathmybb}{U}{bbold}{m}{n} % Indicatrices
\newcommand{\1}{\mathmybb{1}}

% Tikz
\usepackage{tikz, tkz-fct, tkz-euclide, tikz-cd, tkz-fct, pgfplots}
\pgfplotsset{compat=1.18}
\usetikzlibrary{
  angles, quotes, 3d, positioning,
  shapes,fit, arrows, arrows.meta, calc, 
  matrix, calligraphy, intersections, 
  quotes, patterns, patterns.meta, 
  decorations.pathreplacing, decorations.markings,decorations.pathmorphing,
}
\usepgfplotslibrary{fillbetween}
\tikzset{
  withparens/.style = {draw, outer sep=0pt,
    left delimiter= (, right delimiter=),
    above delimiter= (, below delimiter=),
    align=center},
  withbraces/.style = {draw, outer sep=0pt,
    left delimiter=\{, right delimiter=\},
    above delimiter=\{, below delimiter=\},
    align=center}
}
\tikzcdset{
  arrow style=tikz,
  diagrams={>={Straight Barb[scale=1]}},
}

% PAGE SETTINGS

\geometry{
  left=25mm, right=25mm, top= 15mm, bottom= 15mm,
  footskip=30pt
  }
\setlength{\parindent}{0cm}
\setlength{\parskip}{0cm}
\setlist[itemize]{itemsep=0pt, leftmargin=25pt}

\setlength{\cftbeforetoctitleskip}{0pt}
\setlength{\cftaftertoctitleskip}{0pt}


\begin{document}
   \subsection*{\subsecstyle{Unicité de la limite}}
   \begin{proof}[\unskip\nopunct]
      Soit \((u_n)\) une suite qui converge vers \(l, l'\), montrons que \(l = l'\).\<

      Soit \(\epsilon > 0\), on a donc: 
      \begin{align*}
         &\bullet \;\; \text{On sait qu'il existe \(N_0\) tel que pour tout \(n > N_0\), on ait \(|u_n - l| < \frac{\epsilon}{2}\)}\\
         &\bullet \;\; \text{On sait qu'il existe \(N_1\) tel que pour tout \(n > N_1\), on ait \(|u_n - l'| < \frac{\epsilon}{2}\)}
      \end{align*}
      Donc pour tout \(n > \max(N_0, N_1)\):
      \[
         |l - l'| \leq |l - u_n| + |l' - u_n| < \epsilon   
      \]
      Donc \(l = l'\)
   \end{proof}

   \subsection*{\subsecstyle{Toute suite convergente est bornée}}
   \begin{proof}[\unskip\nopunct]
   Soit \((u_n)\) une suite convergente vers \(l\in\R\), alors il existe un \(N_0\) tel que pour tout \(n > N_0\), on ait:
   \[
      |u_n - l| < 1    
   \]
   Or d'aprés la deuxième inégalité triangulaire, on a:
   \[
      |u_n| - |l| \leq |u_n - l| < 1
   \]
   Donc si \(n > N_0\), on a \(|u_n| < 1 + |l|\).\+
   Aussi, si \(n \leq N_0\), alors \(|u_n| < \max(|u_0|, |u_1|, \ldots, |u_{N_0}|)\).\<

   On note ces deux majorants respectivement \(M_1, M_2\), alors dans tout les cas \(|u_n| < \max(M_1, M_2)\)
   \end{proof}

   \subsection*{\subsecstyle{Limite de la somme}}
   \begin{proof}[\unskip\nopunct]
      Soit \((u_n), (v_n)\) deux suites réelles qui convergent respectivement vers \(l, l'\), montrons que \((u_n + v_n)\) converge vers \(l + l'\).\<

      Soit \(\epsilon > 0\), alors:
      \begin{align*}
         &\bullet \;\; \text{On sait qu'il existe \(N_0\) tel que pour tout \(n > N_0\), on ait \(|u_n - l| < \frac{\epsilon}{2}\)}\\
         &\bullet \;\; \text{On sait qu'il existe \(N_1\) tel que pour tout \(n > N_1\), on ait \(|v_n - l'| < \frac{\epsilon}{2}\)}
      \end{align*}
      Alors pour \(N = \max(N_0, N_1)\), on a que pour tout \(n > N\), on a:
      \[
         |u_n + v_n - (l + l')| \leq |u_n - l| + |v_n - l'| < \epsilon
      \]
      Et on conclut alors par transitivité.
   \end{proof}
   \subsection*{\subsecstyle{Limite du produit avec un scalaire}}
   \begin{proof}[\unskip\nopunct]
   Soit \((u_n)\) une suite qui converge vers \(l\) et \(\lambda \in \R\), montrons que \((\lambda u_n)\) converge vers \(\lambda l\).\<

   Soit \(\epsilon > 0\), on sait qu'il existe un rang \(N_0\) tel que pour tout \(n > N_0\), on ait:
   \[
      |u_n - l| < \frac{\epsilon}{|\lambda|}   
   \] 
   Et donc à partir de ce même \(N_0\), on conclut en multipliant l'inégalité par \(|\lambda|\) et utilisant les propriétés élémentaires de la valeur absolue.
   \end{proof}

   \subsection*{\subsecstyle{Limite du produit}}
   \begin{proof}[\unskip\nopunct]
      Soit \((u_n), (v_n)\) deux suites réelles qui convergent respectivement vers \(l, l'\), montrons que \((u_nv_n)\) converge vers \(ll'\).\<

      Soit \(\epsilon > 0\), cherchons une expression de \(u_nv_n - ll'\) qui nous permettrait d'utiliser leurs convergences individuelles, alors on a:
      \[
         |u_nv_n - ll'| = |u_n(v_n - l') + u_nl' - ll'| = |u_n{\color{BrightRed1}(v_n - l')} + l'{\color{BrightRed1}(u_n - l)}|   
      \]
      Or alors on a:
      \begin{align*}
         |u_nv_n - ll'| &= |u_n(v_n - l') + l'(u_n - l)|  \\
         &\leq |u_n(v_n - l')| + |l'(u_n - l)|  \\
         &= |u_n||(v_n - l')| + |l'||(u_n - l)|  
      \end{align*}
      Or on sait qu'il existe un rang \(N_0\) tel que pour tout \(n > N_0\), on ait \(|u_n - l| < \frac{\epsilon}{2|l'|}\).\+
      Aussi, \(u_n\) converge donc est bornée par un \(M\) réel, et donc il existe un rang \(N_1\) tel que pour tout \(n > N_1\), on ait \(|v_n - l'| < \frac{\epsilon}{2M}\).\<

      Alors pour tout \(n\) à partir du rang \(\max(N_0, N_1)\), on a alors:
      \begin{align*}
         |u_n||(v_n - l')| + |l'||(u_n - l)| &\leq |u_n|\frac{\epsilon}{2M} + |l'|\frac{\epsilon}{2|l'|}\\
         &\leq |u_n|\frac{\epsilon}{2|u_n|} + \frac{\epsilon}{2} \\
         &\leq \epsilon
      \end{align*}
   \end{proof}
   \subsection*{\subsecstyle{Limite de l'inverse}}
   \begin{proof}[\unskip\nopunct]
      Soit \((u_n)\) une suite qui tends vers \(+\infty\), montrons que \((\frac{1}{u_n})\) tends vers \(0\).\<

      Soit \(\epsilon > 0\), on sait que pour tout \(M \in \R\), il existe \(N_0\) tel que pour tout \(n > N_0\), on ait:
      \[
         u_n > M   
      \]
      Alors en particulier pour \(M = \frac{1}{\epsilon} > 0\), il existe \(N_0\) tel que pour tout \(n > N_0\), on a:
      \[
         u_n > \frac{1}{\epsilon}
      \]
      Le passage à l'inverse permet alors de conclure. En particulier, à partir d'un certain rang, \(u_n\) n'est jamais nulle et donc le passage à l'inverse est valide.
   \end{proof}

   \subsection*{\subsecstyle{Théorème de la limite monotone}}
   Soit \(u_n\) une suite croissante et majorée, montrons que \(u_n\) converge.\<

   On pose \(E := \bigl\{ u_n \; ; \; n \in \N  \bigl\}\), alors \(E\) est majoré par hypothèse, et non-vide car \(u_0 \in E\), il admet donc une borne supérieure qu'on notera \(l\).\<

   Soit \(\epsilon > 0\), alors d'aprés la caractérisation de la borne supérieure, il existe un élément \(u_{N_0} \in E\) tel que \(l - \epsilon\) ne soit pas un majorant, on a donc:
   \[
      l - \epsilon < u_{N_0}
   \]
   Or \((u_n)\) est croissante donc pour tout \(n > N_0\), on a \(u_n > u_{N_0}\), et \(l\) majore \(E\) donc a fortiori \(l + \epsilon\) majore aussi \(E\), donc pour tout \(n > N_0\) on a:
   \[
      l - \epsilon < u_n < l + \epsilon
   \]
   La suite \(u_n\) converge donc bien vers \(l\).
   
   \subsection*{\subsecstyle{Théorème des suites adjacentes}}
   \begin{proof}[\unskip\nopunct]
      Soit \((u_n), (v_n)\) deux suites respectivement croissante et décroissante, avec \(u_n \leq v_n\) et \((v_n - u_n) \rightarrow 0\), montrons que ces deux suites convergent vers la même limite.\<

      Par la monotonie de ces suites on trouve:
      \[
         u_0 \leq u_n \leq v_n \leq v_0   
      \]
      En particulier, \(u_n\) est croissante et majorée par \(v_0\), donc elle converge vers \(l\).
      Aussi, \(v_n\) est décroissante et minorée par \(u_0\), donc elle converge vers \(l'\).\<
   
      De plus on sait que \((v_n - u_n) \rightarrow 0\) donc \((v_n - u_n) + u_n = v_n\) tends vers \(l\) donc \(l = l'\).
   \end{proof}

   \subsection*{\subsecstyle{Théorème des gendarmes}}
   \begin{proof}[\unskip\nopunct]
      Soit \((u_n), (w_n)\) deux suites qui tendent vers \(l \in \R\), et \((v_n)\) une suite telle que pour tout \(n \in \N\) on ait \(u_n \leq v_n \leq w_n\), montrons que \((v_n)\) tends vers \(l\).\<

      Soit \(\epsilon > 0\), alors on sait que:
      \begin{align*}
         &\bullet \;\; \text{A partir d'un certain rang \(N_0\), on a \( -\epsilon < w_n - l < \epsilon\)}\\
         &\bullet \;\; \text{A partir d'un certain rang \(N_1\), on a \(- \epsilon < u_n - l < \epsilon\)}
      \end{align*}
      Alors à partir de \(N = \max(N_0, N_1)\), on a:
      \[
         -\epsilon < u_n - l \leq v_n - l \leq w_n - l < \epsilon
      \]
      Et donc on conclut que \(|v_n - l| < \epsilon\).
      
   \end{proof}

\end{document}