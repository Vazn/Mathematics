\documentclass{report}
% Maths Packages
\usepackage{mathtools, amsthm, amssymb, mathrsfs, interval, stmaryrd, centernot, esvect, cancel, commath, blkarray, empheq}
\usepackage{tabularx}
\usepackage{booktabs}
\usepackage{cellspace}
\setlength{\cellspacetoplimit}{5pt}
\setlength{\cellspacebottomlimit}{5pt}


% Sagemaths Formating Packages
\usepackage{listings}
\lstdefinelanguage{Sage}[]{Python}
{morekeywords={False,sage,True},sensitive=true}
\lstset{
  frame=none,
  showtabs=False,
  showspaces=False,
  showstringspaces=False,
  commentstyle={\ttfamily\color{dgreencolor}},
  keywordstyle={\ttfamily\color{dbluecolor}\bfseries},
  stringstyle={\ttfamily\color{dgraycolor}\bfseries},
  language=Sage,
  basicstyle={\fontsize{10pt}{10pt}\ttfamily},
  aboveskip=0.4em,
  belowskip=0.4em,
}

% TOC Packages
\usepackage{tocloft, titletoc, hyperref, bookmark}
% Formatting / Style Packages
\usepackage[T1]{fontenc}
\usepackage{geometry, subcaption, graphicx, fix-cm, accents, float, varwidth, soul, ulem, contour, multicol, enumitem}    
\usepackage[bottom]{footmisc}
\usepackage[x11names, table]{xcolor}
\usepackage[most, skins]{tcolorbox}
\usepackage{adjustbox}
\DeclareMathAlphabet{\mathmybb}{U}{bbold}{m}{n} % Indicatrices
\newcommand{\1}{\mathmybb{1}}

% Tikz
\usepackage{tikz, tkz-fct, tkz-euclide, tikz-cd, tkz-fct, pgfplots}
\pgfplotsset{compat=1.18}
\usetikzlibrary{
  angles, quotes, 3d, positioning,
  shapes,fit, arrows, arrows.meta, calc, 
  matrix, calligraphy, intersections, 
  quotes, patterns, patterns.meta, 
  decorations.pathreplacing, decorations.markings,decorations.pathmorphing,
}
\usepgfplotslibrary{fillbetween}
\tikzset{
  withparens/.style = {draw, outer sep=0pt,
    left delimiter= (, right delimiter=),
    above delimiter= (, below delimiter=),
    align=center},
  withbraces/.style = {draw, outer sep=0pt,
    left delimiter=\{, right delimiter=\},
    above delimiter=\{, below delimiter=\},
    align=center}
}
\tikzcdset{
  arrow style=tikz,
  diagrams={>={Straight Barb[scale=1]}},
}

% PAGE SETTINGS

\geometry{
  left=25mm, right=25mm, top= 15mm, bottom= 15mm,
  footskip=30pt
  }
\setlength{\parindent}{0cm}
\setlength{\parskip}{0cm}
\setlist[itemize]{itemsep=0pt, leftmargin=25pt}

\setlength{\cftbeforetoctitleskip}{0pt}
\setlength{\cftaftertoctitleskip}{0pt}

\setcounter{secnumdepth}{-1}

% STYLE
\definecolor{BrightBlue1}{RGB}{95, 150, 210}

\definecolor{BrightRed1}{RGB}{210, 95, 95}
\definecolor{BrightRed2}{RGB}{210, 115, 115}

\definecolor{DarkBlueX}{RGB}{43, 68, 92}
\definecolor{DarkBlue0}{RGB}{53, 78, 102}
\definecolor{DarkBlue1}{RGB}{83, 108, 132}
\definecolor{DarkBlue2}{RGB}{58, 94, 132}
\definecolor{DarkBlue3}{RGB}{90, 126, 162}

\definecolor{DarkGreen3}{RGB}{83, 132, 108}
\definecolor{DarkGreen2}{RGB}{58, 132, 94}
\definecolor{DarkGreen1}{RGB}{90, 162, 126}
\tcbset{shield externalize, enhanced, sharp corners, halign=center, center}

\definecolor{dblackcolor}{rgb}{0.0,0.0,0.0}
\definecolor{dbluecolor}{rgb}{0.01,0.02,0.7}
\definecolor{dgreencolor}{rgb}{0.2,0.4,0.0}
\definecolor{dgraycolor}{rgb}{0.30,0.3,0.30}
\newcommand{\dblue}{\color{dbluecolor}\bf}
\newcommand{\dred}{\color{dredcolor}\bf}
\newcommand{\dblack}{\color{dblackcolor}\bf}

%Underline settings
\setlength{\ULdepth}{2pt}
\contourlength{0.8pt}
\renewcommand{\underline}[1]{
  \uline{\phantom{#1}}%
  \llap{\contour{white}{#1}}%
}

%drop shadow southwest=black!100!black
\newcommand{\secstyle}[1]{\color{DarkBlue1}\fbox{#1}}
\newcommand{\subsecstyle}[1]{\color{DarkBlue2}\underline{#1}}
\newcommand{\subsubsecstyle}[2]{\color{DarkBlue3}\underline{#1}}

\newcommand{\chapterstyle}[1]{
    \setlength{\fboxsep}{0.3em}
    \setlength{\fboxrule}{3pt}
    \centering\vspace{-70pt}
    
    \color{DarkBlue1}\huge\fbox{\textbf{\textsc{#1}}}
}

\newcommand{\customBox}[2]{
    \tcbset{boxrule=1.5pt, boxsep=-0.2mm, colframe=DarkBlue1, colback=BrightBlue1!05}
    \begin{tcolorbox}[#1]
        \abovedisplayskip=0pt % remove vertical space above align
        #2
    \end{tcolorbox}
}

\makeatletter % Crée une trés grosse taille de police pour la page de garde
\newcommand\HUGE{\@setfontsize\Huge{40}{60}}
\makeatother   

\makeatletter
\newcommand\footnoteref[1]{\protected@xdef\@thefnmark{\ref{#1}}\@footnotemark}
\makeatother

% COMMANDS
% TOC
\renewcommand{\cftchapfont}{\large \bfseries \scshape}
\renewcommand{\cftsecfont}{}
\renewcommand{\contentsname}{\hfill
\setlength{\fboxsep}{0.3em}\setlength{\fboxrule}{3pt}\vspace{20pt}
   \color{DarkBlue1}\Huge
   \fbox{\textbf{\textsc{Table des matières}}}
   \hfill
}

% MATHS
\newcommand{\C}{\mathbb{C}}
\newcommand{\R}{\mathbb{R}}
\newcommand{\Q}{\mathbb{Q}}
\newcommand{\Z}{\mathbb{Z}}
\newcommand{\N}{\mathbb{N}}
\newcommand{\U}{\mathbb{U}}
\newcommand{\K}{\mathbb{K}}

\newcommand{\A}{\mathbf{\mathscr{A}}}
\newcommand{\B}{\mathbf{\mathscr{B}}}
\newcommand{\Fam}{\mathbf{\mathscr{F}}}
\renewcommand{\P}{\mathbf{\mathscr{P}}}

\renewcommand{\epsilon}{\varepsilon}
\renewcommand{\rho}{\varrho}

\newcommand{\E}{\mathbf{\mathcal{E}}}
\newcommand{\F}{\mathbf{\mathcal{F}}}
\newcommand{\Pow}{\mathbf{\mathcal{P}}}
\newcommand{\G}{\mathbf{\mathfrak{G}}}

\newcommand{\<}{\bigskip}
\newcommand{\+}{\par}

% Notation equality

\newcommand\notationEq{\stackrel{\mbox{
    \begin{tiny}  
        notation
    \end{tiny}    
}}{=}}

% INTERVALS

\intervalconfig{separator symbol =  \,; \,}

\newcommand{\ioo}[2]{\interval[open]{#1}{#2}}
\newcommand{\ioc}[2]{\interval[open left]{#1}{#2}}
\newcommand{\ico}[2]{\interval[open right]{#1}{#2}}
\newcommand{\icc}[2]{\interval{#1}{#2}}

\newcommand{\intioo}[2]{\left\rrbracket{#1}\;;\;{#2}\right\llbracket}
\newcommand{\intioc}[2]{\left\rrbracket{#1}\;;\;{#2}\right\rrbracket}
\newcommand{\intico}[2]{\left\llbracket{#1}\;;\;{#2}\right\llbracket}
\newcommand{\inticc}[2]{\left\llbracket{#1}\;;\;{#2}\right\rrbracket}

% EQUATIONS NOTES

\newcommand{\shorteqnote}[1]{ &  & \text{\small\llap{#1}}}
\newcommand{\longeqnote}[1]{& & \\ \notag&  &  &  &  & \text{\small\llap{#1}}}

% FUNCTIONS NOTATIONS

\newcommand{\inject}{\hookrightarrow} 
\newcommand{\surject}{\twoheadrightarrow}

% MOD NOTATION

\newcommand{\eqmod}[1]{\underset{#1}{\equiv}} 

% LINEAR ALGEBRA

\newcommand{\dotproduct}[2]{\left\langle\;\! #1 \;\! | \;\! #2 \;\! \right\rangle}
\newcommand{\vectNorm}[1]{\left\Vert#1 \right\Vert}

\newcommand{\Ker}[1]{\text{Ker}#1}
\newcommand{\Sp}[1]{\text{Sp}(#1)}
\renewcommand{\Im}[1]{\text{Im}#1}

\NewDocumentCommand{\opNorm}{sO{}m}{%
  \IfBooleanTF{#1}{% automatic scaling, use with care
    \left|\opnormkern\left|\opnormkern\left|
    #3
    \right|\opnormkern\right|\opnormkern\right|
  }{
    \mathopen{#2|\opnormkern #2|\opnormkern #2|}
    #3
    \mathclose{#2|\opnormkern #2|\opnormkern #2|}
  }%
}
\newcommand{\opnormkern}{\mkern-1.5mu\relax}% adjust for the font

% TOPOLOGY
\newcommand{\ball}{\mathscr{B}}

% CALCULUS
\newcommand{\partialD}[2]{\frac{\partial #1}{\partial #2}}

% GEOMETRY
\newcommand{\RightAgnle}[4][5pt]
{%
    \draw($#3!#1!#2$)-- ($#3!2! ($ ($#3!#1!#2$)!.5! ($#3!#1!#4$)$)$)-- ($#3!#1!#4$);
}

% PROBABILITIES
\newcommand{\probability}[1]{\mathbb{E} (#1)}
\newcommand{\expectancy}[1]{\mathbb{E} (#1)}
\newcommand{\variance}[1]{\mathbb{V} (#1)}
\newcommand{\covariance}[1]{\mathbb{C} (#1)}


\begin{document}
   Dans la suite \((X, \mathscr{B}, \mu)\) est un espace mesuré.
   \subsection*{\subsecstyle{Propriétés de la mesure}}
   On rapelle qu'une mesure est nulle sur la partie vide et \textbf{sigma additive}, montrons les propriétés suivantes pour \( A, B \) des parties mesurables quelconques et \( (A_n), (B_n) \) deux suites de parties mesurables respectivement croissante et décroissante:
   \[ 
      \begin{cases}
         \mu(A \backslash B) = \mu(A) - \mu(B \cap A)\\
         \mu(A \cup B) = \mu(A) + \mu(B) - \mu(A \cap B)\\
         \mu(\bigcup A_n) = \lim_{n \rightarrow +\infty} \mu(A_n)\\
         \mu(\bigcap B_n) = \lim_{n \rightarrow + \infty} \mu(B_n)
      \end{cases} 
   \]
   1) On a directement que:
   \[ 
      A = A \backslash B \cup A \cap B 
   \]
   Donc:
   \[ 
      \mu(A) = \mu(A \backslash B) + \mu(A \cap B) 
   \]
   Si \( B \subseteq A \), on a donc le cas particulier usuel.\<

   2) On a directement l'union disjointe:
   \[ 
      A  \cup B = (A \backslash A \cap B) \cup (B \backslash A \cap B) \cup (A \cap B)  
   \]
   Et donc:
   \[ 
      \mu(A \cup B) = \mu(A) - \mu(A \cap B) + \mu(B) - \mu(A \cap B) + \mu(A \cap B) = \mu(A) + \mu(B) - \mu(A \cap B)
   \]\<

   3) Si \( (A_n) \) est une suite croissante, on peut écrire son union sous la forme de l'union disjointe suivante:
   \[ 
      \bigcup_{n \in \N} A_n = A_0 \cup \bigcup_{n \leq 1} A_n \backslash A_{n-1}
   \]
   Et donc on en déduis que:
   \[ 
      \mu\left(\bigcup_{n \in \N} A_n\right) = \mu(A_0) + \lim_{n \rightarrow \infty} \sum_{k=1}^n \mu(A_n) - \mu(A_{n-1}) 
   \]
   La somme se téléscope et on obtient:
   \[ 
      \mu\left(\bigcup_{n \in \N} A_n\right) = \mu(A_0) + \lim_{n \rightarrow \infty} \mu(A_n) - \mu(A_0) = \lim_{n \rightarrow \infty} \mu(A_n) 
   \]

   4) 
   
   \subsection*{\subsecstyle{Propriété de nullité presque partout}}
   \subsection*{\subsecstyle{Propriété de finitude presque partout}}
   \subsection*{\subsecstyle{Lemme de Borel-Cantelli}}

   
   \subsection*{\subsecstyle{Théorême de convergence dominée}}
   On se donne une suite \(f_n \in \mathscr{M}(X)\) qui converge simplement vers \(f\). On suppose qu'il existe \(g\) intégrable telle que \(\forall n \in \N \; ; \; |f_n| \leq g\). Alors on a:
   \begin{itemize}
      \item Pour tout \(n\in \N\), \(f_n\) est intégrable car par croissance de l'intégrale \(\int |f_n| d\mu \leq \int g d\mu < \infty\)
      \item Par passage à la limite, \(|f| \leq g\) et donc \(f\) est intégrable par le même raisonnement.
   \end{itemize}
   Montrons maintenant qu'on peut faire l'interversion limite/intégrale, on définit les suites (de fonctions) suivantes:
   \[
      \begin{cases}
         u_n := |f_n - f|\\
         v_n := \sup\{u_k \; ; \; k > n\}\\
         w_n := 2g - v_n
      \end{cases}
   \]
   Soit \(x \in X\), étudions tout d'abord la suite \((v_n)\), on peut montrer les deux propriétés suivantes:
   \begin{itemize}
   \item Elle est décroissante:
   En effet si \(n < m\), on a \(\{u_k(x) \; ; \; k > m\} \subseteq \{u_k(x) \; ; \; k > n\}\) en passant à la borne supérieure on a \(v_m(x) > v_n(x)\)
   \item Elle tends vers 0, en effet \((u_n)\) converge simplement vers 0 et on a:
   \[
      \lim_{n \rightarrow +\infty} v_n(x) =  \lim_{n \rightarrow +\infty} \sup\{u_k(x) \; ; \; k > n\} = \limsup u_n(x) = \lim_{n \rightarrow +\infty} u_n(x) = 0
   \]
   \end{itemize}
   On déduis donc de ces résultats que la suite \(w_n\) est une suite \textbf{croissante de fonctions mesurables}, en effet si \(n > n'\), on a:
   \[
      -v_n(x) > -v_n'(x) 
   \]
   Et donc:
   \[
      2g(x) - v_n(x) > 2g(x) -v_n'(x) 
   \]
   En outre cette suite est positive à partir d'un certain rang car \(v_n\) tends vers 0 et \(g\) est positive (donc \(|v_n| < 2g\) à partir d'un certain rang). On peut donc appliquer \textbf{le théorème de convergence monotone} et on obtient que:
   \[
      \lim_{n \rightarrow +\infty} \int 2g - v_n d\mu = \int 2g d\mu
   \]
   En appliquant la linéarité dans l'intégrale de gauche et en calculant la limite de cette manière on obtient alors que:
   \[
      \lim_{n \rightarrow +\infty} \int v_n d\mu = 0
   \]
   Finalement, pour tout \(n \in \N\) on a que:
   \[
      |f_n - f| \leq \sup\{|f_n - f| \; ; \; k > n\} = v_n
   \]
   Donc en intégrant ces fonctions et en passant à la limite, on trouve:
   \[
      \int |f_n - f| d\mu \leq \int v_n d\mu \underset{n \rightarrow \infty}{\longrightarrow} 0
   \]
   Et par inégalité triangulaire, on a: 
   \[
      \left|\int f_n d\mu - \int f d\mu  \right| \underset{n \rightarrow \infty}{\longrightarrow} 0
   \]
   Donc:
   \[
      \int f_n d\mu \underset{n \rightarrow \infty}{\longrightarrow} \int f d\mu
   \]

   \pagebreak
  
\end{document}