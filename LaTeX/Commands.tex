% TOC
\renewcommand{\cftchapfont}{\large \bfseries \scshape}
\renewcommand{\cftsecfont}{}
\renewcommand{\contentsname}{\hfill
\setlength{\fboxsep}{0.3em}\setlength{\fboxrule}{3pt}\vspace{20pt}
   \color{DarkBlue1}\Huge
   \fbox{\textbf{\textsc{Table des matières}}}
   \hfill
}

% MATHS
\newcommand{\C}{\mathbb{C}}
\newcommand{\R}{\mathbb{R}}
\newcommand{\Q}{\mathbb{Q}}
\newcommand{\Z}{\mathbb{Z}}
\newcommand{\N}{\mathbb{N}}
\newcommand{\U}{\mathbb{U}}
\newcommand{\K}{\mathbb{K}}

\newcommand{\A}{\mathbf{\mathscr{A}}}
\newcommand{\B}{\mathbf{\mathscr{B}}}
\newcommand{\Fam}{\mathbf{\mathscr{F}}}
\renewcommand{\P}{\mathbf{\mathscr{P}}}

\renewcommand{\epsilon}{\varepsilon}
\renewcommand{\rho}{\varrho}

\newcommand{\E}{\mathbf{\mathcal{E}}}
\newcommand{\F}{\mathbf{\mathcal{F}}}
\newcommand{\Pow}{\mathbf{\mathcal{P}}}
\newcommand{\G}{\mathbf{\mathfrak{G}}}

\newcommand{\<}{\bigskip}
\newcommand{\+}{\par}

% Notation equality

\newcommand\notationEq{\stackrel{\mbox{
    \begin{tiny}  
        notation
    \end{tiny}    
}}{=}}

% INTERVALS

\intervalconfig{separator symbol =  \,; \,}

\newcommand{\ioo}[2]{\interval[open]{#1}{#2}}
\newcommand{\ioc}[2]{\interval[open left]{#1}{#2}}
\newcommand{\ico}[2]{\interval[open right]{#1}{#2}}
\newcommand{\icc}[2]{\interval{#1}{#2}}

\newcommand{\intioo}[2]{\left\rrbracket{#1}\;;\;{#2}\right\llbracket}
\newcommand{\intioc}[2]{\left\rrbracket{#1}\;;\;{#2}\right\rrbracket}
\newcommand{\intico}[2]{\left\llbracket{#1}\;;\;{#2}\right\llbracket}
\newcommand{\inticc}[2]{\left\llbracket{#1}\;;\;{#2}\right\rrbracket}

% EQUATIONS NOTES

\newcommand{\shorteqnote}[1]{ &  & \text{\small\llap{#1}}}
\newcommand{\longeqnote}[1]{& & \\ \notag&  &  &  &  & \text{\small\llap{#1}}}

% FUNCTIONS NOTATIONS

\newcommand{\inject}{\hookrightarrow} 
\newcommand{\surject}{\twoheadrightarrow}

% MOD NOTATION

\newcommand{\eqmod}[1]{\underset{#1}{\equiv}} 

% LINEAR ALGEBRA

\newcommand{\dotproduct}[2]{\left\langle\;\! #1 \;\! | \;\! #2 \;\! \right\rangle}
\newcommand{\vectNorm}[1]{\left\Vert#1 \right\Vert}

\newcommand{\Ker}[1]{\text{Ker}#1}
\newcommand{\Sp}[1]{\text{Sp}(#1)}
\renewcommand{\Im}[1]{\text{Im}#1}

\NewDocumentCommand{\opNorm}{sO{}m}{%
  \IfBooleanTF{#1}{% automatic scaling, use with care
    \left|\opnormkern\left|\opnormkern\left|
    #3
    \right|\opnormkern\right|\opnormkern\right|
  }{
    \mathopen{#2|\opnormkern #2|\opnormkern #2|}
    #3
    \mathclose{#2|\opnormkern #2|\opnormkern #2|}
  }%
}
\newcommand{\opnormkern}{\mkern-1.5mu\relax}% adjust for the font

% TOPOLOGY
\newcommand{\ball}{\mathscr{B}}

% CALCULUS
\newcommand{\partialD}[2]{\frac{\partial #1}{\partial #2}}

% GEOMETRY
\newcommand{\RightAgnle}[4][5pt]
{%
    \draw($#3!#1!#2$)-- ($#3!2! ($ ($#3!#1!#2$)!.5! ($#3!#1!#4$)$)$)-- ($#3!#1!#4$);
}

% PROBABILITIES
\newcommand{\probability}[1]{\mathbb{E} (#1)}
\newcommand{\expectancy}[1]{\mathbb{E} (#1)}
\newcommand{\variance}[1]{\mathbb{V} (#1)}
\newcommand{\covariance}[1]{\mathbb{C} (#1)}