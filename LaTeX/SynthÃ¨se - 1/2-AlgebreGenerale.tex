\chapter*{\chapterstyle{II --- Introduction}}
\addcontentsline{toc}{section}{Introduction}

On appelle \textbf{structure algébrique} un ensemble muni d'une (ou plusieurs) opérations appelées \textbf{lois}, c'est l'étude de telles structures mathématiques, des relations entre celles-ci (que nous appeleront morphismes), et de leurs propriétés que nous appeleront \textbf{algèbre générale}.\<

Soit \(M\) un ensemble non-vide, on appelle \textbf{loi de composition interne} une opération binaire sur les éléments de \(M\) (qu'on notera temporairement \(\star\)) telle que:
\customBox{width=5cm}{
   \(\forall a, b \in M \; ; \; a \star b \in M\)
}
Soit \(E\) un ensemble non-vide, on appelle \textbf{loi de composition externe} une opération binaire entre un élément de \(E\) et un élément de \(M\) (qu'on notera temporairement \(\cdot\)) telle que:
\customBox{width=5cm}{
   \(\forall \lambda, a \in E \times M \; ; \; \lambda \cdot a \in M\)
}

On appelle \textbf{élément neutre} pour la loi un élément\footnote[1]{Dans la suite on notera l'élément neutre d'une structure \(M\) par \(e_M\)} \(e \in M\) tel que pour tout élément de \(a \in M\), on ait:
\[a \star e = e \star a = a\]

On appelle \textbf{inverse} pour la loi (si elle admet un élément neutre) un élément \(a^{-1} \in M\) tel que pour tout élément de \(a \in M\), on ait 
\[a \star a^{-1} = a^{-1} \star a = e\]

\subsection*{\subsecstyle{Monoides {:}}}
Soit \(M\) un ensemble qu'on munit d'une \textbf{loi de composition interne}, alors le couple \((M, \star)\) est appelé un \textbf{magma}, c'est la structure algébrique primitive la plus faible, en effet la seule contrainte étant que la loi soit interne.\<

On peut alors enrichir la structure de magma par les deux contraintes supplémentaires suivantes:
\begin{align*}
   &\bullet \;\; \text{La loi est \textbf{associative}.}\\
   &\bullet \;\; \text{Il existe \textbf{un élément neutre} pour la loi.}
\end{align*}
Cette structure plus riche, qu'on appelle \textbf{monoide} nous permet alors d'identifier des exemples remarquables:
\begin{align*}
   &\bullet \;\; \text{Les entiers naturels munis de l'addition forment un monoide.}\\
   &\bullet \;\; \text{L'ensemble des chaines de caractères muni de la concaténation forme un monoide.}
\end{align*}
Les éléments neutres respectifs de ces exemples sont \(0_\N\) et la chaine de caractère vide.
\subsection*{\subsecstyle{Morphismes {:}}}
Aprés avoir défini une structure, on peut alors définir les transformations qui \textbf{préservent cette structure}.\<

Soit \((M, \star)\) et \((N, \cdot)\) deux ensembles munis de la meme structure\footnote[2]{Si les structures présentent plusieurs lois, alors les morphismes doivent vérifier la compatibilité pour \textbf{toutes les lois}.}, \(x, y \in M\) et \(\varphi: M \rightarrow N\), alors \(\varphi\) est appelée \textbf{morphisme}, si elle vérifie:
\customBox{width=5cm}{
   \(
      \varphi(x \star y) = \varphi(x) \cdot \varphi(y)
   \)
}
Dans le cas particulier de structures qui requièrent l'existence d'un élément neutre, l'image de l'élement neutre de \(M\) par un morphisme doit etre l'élément neutre de \(N\).\<

En termes de vocabulaire, on définit alors: 
\begin{align*}
   &\bullet \;\; \text{\textbf{Les endormorphismes} commme les morphismes de \(M\) dans lui-meme.}\\
   &\bullet \;\; \text{\textbf{Les isomorphismes} commme les morphismes bijectifs.}\\
   &\bullet \;\; \text{\textbf{Les automorphismes} commme les morphismes bijectifs de \(M\) dans lui-meme.}
\end{align*}
Moralement, on comprends que:
\begin{center}
   \textit{Les morphismes préservent dans une certaine mesure la structure opératoire.}
\end{center}
En particulier:
\begin{center}
   \textit{Si il existe un isomorphisme entre deux structures, cela signifie que ces structures se comportent de la meme manière par rapport à leurs lois respectives.}
\end{center}
De manière plus subtile, si il existe un morphisme injectif d'une structure dans une autre, alors cela signifie qu'une partie de la seconde se comporte de la meme manière que la première. \<

Enfin, si il existe un morphisme surjectif d'une structure dans une autre, cela signifie qu'on peut regrouper des élements de la première structure et ces groupes d'éléments se comporteront commes les élements de la seconde\footnote[1]{Ces considérations plus avancées induiront l'idée générale des grands théorèmes de factorisation dans les groupes et les principaux morphismes canoniques.}.
\subsection*{\subsecstyle{Propriétés des morphismes {:}}}
Pour une structure donnée, si \(\varphi\) est un morphisme de \(E\) vers \(F\) et \(\psi\) est un morphisme de \(F\) vers \(G\), alors \(\psi \circ \varphi\) est un morphisme de \(E\) vers \(G\).\<

Si la structure admet un élément neutre pour la loi, alors:
\customBox{width=4cm}{
   \(\varphi(e_E) = e_F\)
}
Si la structure admet un symétrique pour la loi, alors:
\customBox{width=4cm}{
   \(\quad\;\varphi(x^{-1}) = \varphi(x)^{-1}\)
}
\subsection*{\subsecstyle{Exemples {:}}}
On peut considérer quelques exemples remarquables:
\begin{itemize}
   \item \(\phi: n \in \N \longmapsto 2n \in 2\N\) est un morphisme (de monoide) de \((\N, +)\) dans \((2\N, +)\).
   \item \(\phi: x \in \R \longmapsto \exp(x) \in \R^{+*}\) est un isomorphisme (de groupe) de \((\R, +)\) dans \((\R^{+*}, \times)\).
   \item \(\phi: M \in GL_{n}(\K) \longmapsto \det(M) \in \K^*\) est un morphisme (de groupe) de \((GL_{n}(\K), \times)\) dans \((\K, \times)\).
   \item \(\phi: z \in \C \longmapsto \overline{z} \in \C\) est un automorphisme (de corps) de \((\C, +, \times)\) dans \((\C, +, \times)\).
\end{itemize}
\pagebreak
\subsection*{\subsecstyle{Sous-structures {:}}}
Une fois une structure algébrique définie sur \(E\), on peut alors s'intéresser aux parties de \(E\) qui conservent cette structure, on les appellera alors \textbf{sous-structures} de \(E\).\<

En particulier, on dira que \(F\) est une \textbf{sous-structure} de \(E\) (et on notera \(F < E\)) si elle vérifie:
\begin{itemize}
   \item La partie \(F\) est stable par les lois.
   \item Les éléments neutres\footnote[1]{\textbf{Si la structure impose leur existence}} appartient à \(F\) 
   \item Les inverses\footnote[2]{\textbf{Si la structure impose leur existence}} des éléments de \(F\) appartient à \(F\)
\end{itemize}

On peut alors montrer les propositions suivantes:
\begin{itemize}
   \item L'image et la préimage d'une sous-structure par un morphisme est une sous-structure.
   \item L'intersection d'une famille de sous-structures est une sous-structure
\end{itemize}
En particulier les noyaux de morphismes sont des sous-structures de la structure de départ.
\subsection*{\subsecstyle{Structures Quotients {:}}}
Dans le domaine ensembliste, on sait créer des ensembles quotients via une relation d'équivalence, on cherche par la suite à créer des ensembles quotients qui \textbf{conservent les propriétés de structure}, en particulier, on cherche des conditions sur une relation d'équivalence \(\sim\) pour que \(G/_\sim\) soit un groupe, un anneau, ou un corps.

On peut alors montrer qu'il faut et il suffit que \(\sim\) soit \textbf{compatible} avec les lois, ie que pour toute loi \(\star\), on ait:
\customBox{width=9cm}{
   \(
      x_1 \sim x_2 \text{ et } y_1 \sim y_2 \implies x_1 \star y_1 \sim x_2 \star y_2  
   \)
}

On peut alors montrer que \(G/_\sim\) conserve les propriétés de structure de \(G\).\<

On peut alors définir la \textbf{surjection canonique} qui est le morphisme \(\pi : E \rightarrow E /\sim\) qui a chaque élément associe sa classe d'équivalence pour la relation\footnote[1]{Non spécifique aux structures, c'est une application générale liée aux ensembles quotients.}. 
\chapter*{\chapterstyle{II --- Groupes}}
\addcontentsline{toc}{section}{Groupes}

Soit \(G\) un ensemble \textbf{non-vide} muni d'une loi de composition interne associative\footnote[1]{Dans la suite, la loi de composition des groupes sera notée multiplicativement sauf exceptions.} telle que:
\begin{itemize}
   \item Il existe \textbf{un élément neutre} pour la loi.
   \item Tout élément de \(G\) admet \textbf{un inverse} pour la loi.
\end{itemize}
Alors le couple \((G, \star)\) est appellé \textbf{groupe}. De plus si le groupe est \textbf{commutatif}, on dira alors que c'est un groupe \textbf{abélien}.\<

On appellera \textbf{ordre du groupe} le cardinal (potentiellement infini) de l'ensemble sous-jacent, noté \(|G|\).
\subsection*{\subsecstyle{Exemples {:}}}
On peut alors considérer plusieurs groupes remarquables:
\begin{itemize}
   \item Les \textbf{entiers relatifs} muni de l'addition usuelle.
   \item Les \textbf{isométries du plan} muni de la composition, on l'apelle le \textbf{groupe dihédral}.  
   \item Les \textbf{matrices inversibles} muni de la multiplication, on l'apelle le \textbf{groupe linéaire}.
   \item Les \textbf{bijections} sur un ensemble muni de la composition, on l'apelle le \textbf{groupe symétrique}.
\end{itemize}
\subsection*{\subsecstyle{Morphismes de groupes {:}}}
Soit \(G, H\) deux groupes et \(\varphi: G \rightarrow H\), l'existence d'un élément neutre nous permet de définir alors \textbf{le noyau d'un morphisme} par:
\[
   \text{Ker}(\varphi) := \Bigl\{ x \in G \; ; \; \varphi(x) = e_H \Bigl\}
\]
En particulier, on peut alors montrer:
\customBox{width=15cm}{
   Un morphisme est injectif si et seulement si son noyau est réduit à l'élément neutre.
}
\subsection*{\subsecstyle{Sous-groupes {:}}}
Les sous-structures dans le cas des groupes sont naturellement les sous-groupes. Un cas remarquable est celui du \textbf{sous-groupe engendré} par \(H\) qu'on note:
\[ 
   \langle H \rangle := \Bigl\{ h_1^{k_1}h_2^{k_2} \ldots h_n^{k_n}\; ; \; n \in \N \; , \; h_i \in H \; , \; k_i \in \Z \Bigl\}
\]

Une propriété fondamentale est que \(\langle H \rangle\) est un opérateur de cloture par la loi du groupe, ie c'est une application \textbf{idempotente, croissante et extensive}.\<

On peut alors considérer le sous-groupe engendré par un élément \(h \in H\), en effet on a:
\[
   \langle h \rangle := \left\{ h^{k} \; ; \; k \in \Z \right\}
\]
On peut alors définir \textbf{l'ordre d'un élément} comme étant l'ordre du sous-groupe engendré associé (potentiellement infini).\<

Ce sous-groupe permet de définir des groupes remarquables, en effet si un groupe est engendré par un unique élément, il est appelé \textbf{groupe cyclique} dont nous parleront plus loin dans ce chapitre.
\pagebreak

\subsection*{\subsecstyle{Classes {:}}}
On considère maintenant un sous-groupe \( H \leq G \), alors on peut définir deux relations d'équivalences sur \( G \) par:
\[ 
   \begin{cases}
      g_1 \sim g_2 \iff \exists h \in H \; ; \; g_1 = g_2h\\
      g_1 \sim g_2 \iff \exists h \in H \; ; \; g_1 = hg_2
   \end{cases}
\] 
On appelle alors \textbf{classe à gauche} (resp. classe à droite) les classes d'équivalences pour ces deux relations et on note alors \( gH \) (resp. \( Hg \)) la classe d'un élément \( g \) pour cette relation. On note alors \( G/H \) l'ensemble quotient associé aux classes à gauche.
\subsection*{\subsecstyle{Théorème de Lagrange {:}}}
Ces classes induisent donc une partition de \( G \) en classes \textbf{de même cardinal}, en effet:
\[ 
   |gH| = \left|\left\{ gh \; ; \; h \in H \right\} \right| = |H|
\]
En outre on a une bijection qui associe à chaque élément de \( g \) sa classe et l'élément de \( H \) lui correspondant:
\[ 
   \begin{aligned}
      f : G &\longrightarrow (G/H, H)\\
      g &\longmapsto (aH, h)
   \end{aligned}
\]
Ceci nous permet donc de montrer le \textbf{théorème de Lagrange} qui nous donne que pour tout groupe fini \( G \), on a:
\[ 
   |G| = |G/H||H|
\]
Et comme corollaire immédiat la propriété suivante:
\begin{center}
   \textbf{Le cardinal d'un sous-groupe divise le cardinal du groupe.}
\end{center}
\subsection*{\subsecstyle{Sous-groupes normaux {:}}}
On cherche alors à caractériser les sous-groupes tels que la relation d'équivalence définie ci-dessous soit \textbf{compatible} avec les opération de groupe, en d'autres termes on cherche à définir un groupe quotient pour cette relation. On peut alors montrer que les sous-groupes vérifiant cette compatibilité vérifient:
\[ 
   \forall g \in G \; ; \; gH = Hg 
\]
En d'autres termes les classes à droite et à gauche coincident. C'est alors immédiat que \textbf{tout sous-groupe d'un groupe abélien est normal}. Par ailleurs on peut caractériser les sous-groupes normaux d'une autre façon (détaillée au chapitre sur les actions de groupe) comme les sous-groupes qui vérifient:
\[ 
   \forall h \in H \; , \; \forall g \in G \; ; \; ghg^{-1} \in H 
\]

\chapter*{\chapterstyle{II --- Actions de groupe}}
\addcontentsline{toc}{section}{Actions de groupe}
Soit \( G \) un groupe et \( X \) un ensemble quelconque, dans ce chapitre on définit un notion fondamentale en théorie des groupes, la notion \textbf{d'action d'un groupe sur un ensemble.} En effet on appelera \textbf{action} du groupe \( G \) sur \( X \) une application de la forme:
\[ 
   \begin{aligned}
      G \times X &\longrightarrow X\\
      (g, x) &\longmapsto g \cdot x
   \end{aligned}
\]
En outre une action doit vérifier deux autres propriétés:
\begin{itemize}
   \item \textbf{Le neutre n'agit pas: } \( \forall x \in X \; ; \; e \cdot x = x \)
   \item \textbf{Associativité mixte: } \( \forall g_1, g_2, x \in G \times G \times X \; ; \; (g_1g_2) \cdot x = g_1(g_2 \cdot x) \)
\end{itemize}
On dira alors que \( G \) \textbf{agît} sur \( X \) et on notera alors \( G \curvearrowright X \).

\subsection*{\subsecstyle{Morphisme structurel{:}}}
On se donne une action \( G \curvearrowright X\), alors il peut être utile de considérer la currifiée\footnote[1]{On rapelle que \( \mathcal{F}(E \times F, G) \cong \mathcal{F}(E, \mathcal{F}(F, G)) \) en tant qu'ensembles.} de cette action, ie:
\[ 
   \begin{aligned}
      \phi : G &&\longrightarrow (X &\longrightarrow X)\\
      g &&\longmapsto (x &\longmapsto g \cdot x)
   \end{aligned}
\]
On peut alors montrer que cette fonction prends son image dans l'ensemble des bijections sur \( X \) (dont on montrera que c'est un groupe au chapitre sur le groupe symétrique) et que c'est un \textbf{morphisme de groupe}. L'action de \( G \) induit donc un morphisme de groupe, appelé \textbf{morphisme structurel} de la forme:
\[ 
   \phi : G \longmapsto \mathfrak{S}(X)
\]
En outre cette correspondante est bijective, il est donc équivalent de considérer une action d'un groupe sur un ensemble ou un morphisme structurel.

\subsection*{\subsecstyle{Action induite sur l'ensemble des parties {:}}}
Si \(G\) agît sur \( X \) alors \( G\) agît alors naturellement sur \( \mathcal{P}(X)  \) par l'action:
\[ 
   (g, P) \mapsto g \cdot P := \left\{ g \cdot x \; ; \; x \in P \right\} 
\]
\subsection*{\subsecstyle{Action induite sur les sous structures{:}}}
On se pose alors deux questions naturelles:
\begin{itemize}
   \item Une action de \( G \) sur \( X \) induit-elle nécessairement une action de \( G \) sur \( Y \subseteq X \) ?
   \item Une action de \( G \) sur \( X \) induit-elle nécessairement une action de \( H \leq G \) sur \(X\) ?
\end{itemize}
On peut alors montrer que la première question admet une réponse positive si et seulement si \( Y \) est \textbf{stable par l'action}.\<

Pour la seconde question, elle admet toujours une réponse positive et on a même le résultat général suivant gràce au morphisme structurel, on considère deux groupes \( G, H \) reliés par un morphisme \( \phi \), et une action de \( H \) sur \( X \) de morphisme structurel \( \psi \), alors on a le diagramme:
\begin{center}
   \begin{tikzcd}[column sep=large, row sep=large]
      G \arrow[r, "\phi"]
         & H \arrow[r, "\psi"] & X
   \end{tikzcd}
\end{center}
Et donc \( \phi \circ \psi \) définit bien un morphisme structurel de \( G \) sur \( \mathfrak{S}(X) \) et donc une action. Le cas particulier des sous-groupes se déduit en considérant \( \phi \) le morphisme d'inclusion d'un sous-groupe dans le groupe total.
\pagebreak

\subsection*{\subsecstyle{Orbites {:}}}
Considérons un point \( a \in X \), alors on définit \textbf{l'orbite} de \( a \) sous l'action du groupe \( G \) par:
\[ 
   \text{Orb}_G(a) := \left\{ g \cdot a \; ; \; g \in G \right\} 
\]
Intuitivement, ce sont tout les points atteints par l'action de \( G \) sur le point initial \( a \). Une propriété fondamentale des orbites est la suivante, si on considère la relation suivante:
\[ 
   x \sim y \iff y \in \text{Orb}_G(x)
\]
Alors c'est une \textbf{relation d'équivalence}, et on a donc toujours une \textbf{partition} de \( X \) associée à l'action de \( G \), c'est la partition en orbites.
\subsection*{\subsecstyle{Stabilisateurs {:}}}
Considérons un point \( a \in X \), alors on définit \textbf{le stabilisateur} de \( a \) sous l'action du groupe \( G \) par:
\[ 
   \text{Stab}_G(a) := \left\{ g \in G \; ; \; g(a) = a\right\} 
\]
Intuitivement, ce sont tout les éléments du groupe qui laissent \( a \) invariant. Une propriété fondamentale des stabilisateurs est que c'est un \textbf{sous-groupe} du groupe \( G \). En outre si on considère le morphisme structurel \( \phi \) de l'action, on a:
\[ 
   \text{Ker}( \phi) = \bigcap_{x \in X} \text{Stab}_G(x) 
\]
\subsection*{\subsecstyle{Généralisations aux parties {:}}}
On peut alors noter qu'il est aussi possible de définir les orbites et stabilisateurs de \textbf{parties}, en considérant les orbites et stabilisateurs pour l'action induite sur les parties définie plus haut.
\subsection*{\subsecstyle{Vocabulaire {:}}}
On peut alors nommer les actions de groupes qui vérifient certaines propriétés relatives aux ensembles définis plus haut, on appelle alors:
\begin{itemize}
   \item Action \textbf{transitive} une action qui n'admet qu'une seule orbite.
   \item Action \textbf{libre} une action dont tout les stablisateurs sont triviaux.
   \item Action \textbf{fidèle} une action dont le noyau du morphisme structurel est trivial.
\end{itemize}
On a alors d'aprés la caractérisation du noyau ci-dessus que toute action \textbf{libre} est \textbf{fidèle}.
\subsection*{\subsecstyle{Action par automorphismes intérieurs{:}}}
On peut alors aussi étudier l'action du groupe \( G \) sur \textbf{lui-même}, on obtient 


\chapter*{\chapterstyle{II --- Théorèmes d'isomorphismes}}
\addcontentsline{toc}{section}{Théorèmes d'isomorphismes}
On considère ici deux groupes \(G, F\), et \(H\) un sous-groupe normal de \(G\), on sait que \(H\) induit une relation d'équivalence compatible et donc qu'on peut définir le groupe quotient \(G/H\).\<
\subsection*{\subsecstyle{Premier théorème d'isomorphisme {:}}}
Soit \(\phi : G \longrightarrow F\) un morphisme, on peut alors montrer qu'il existe un unique morphisme \(\widetilde{\phi} : G/\Ker{\phi} \longrightarrow F\) tel que le diagramme soit commutatif\footnote[1]{Un \textbf{diagramme commutatif} est une collection d'objets et de morphismes tels tout les chemins (de composition) partant d'un objet vers un autre donnent le meme résultat (ie sont le meme morphisme).}:
\begin{center}
   \begin{tikzcd}[column sep=large, row sep=large]
      G \arrow[d, "\pi", swap, two heads] \arrow[r, "\phi"] & F\\
      G/\Ker{\phi} \arrow[ru, "\widetilde{\phi}", swap, dashed]
   \end{tikzcd}
\end{center}
En particulier, le morphisme \(\widetilde{\phi}'\) est injectif, et on montre qu'il surjectif sur \(\text{Im}(\phi)\) et donc on a le théorème suivant:
\customBox{width=6cm}{
   \(G/\Ker{\phi} \cong \text{Im}(\phi)\) 
}

\chapter*{\chapterstyle{II --- Groupes Symétriques}}
\addcontentsline{toc}{section}{Groupe Symétrique}
On appelle \textbf{groupe symétrique} et on note \(\mathfrak{S}_n\) le groupe des \textbf{permutations} de l'ensemble \(\inticc{1}{n}\) muni de la composition des applications.\<

On remarque alors aisément que l'ordre de \(\mathfrak{S}_n\) est \(n!\).\<

Soit \(\sigma \in \mathfrak{S}_n\) une permutation de \(\inticc{1}{n}\), alors c'est une fonction bijective sur cet ensemble. En particulier, sachant que l'ensemble est fini, c'est une fonction définie par cas qu'on note alors par commodité horizontalement dans un tableau:
\[
   \sigma =  \begin{pmatrix}
      1 & 2 & \ldots & n\\
      \sigma(1) & \sigma(2) & \ldots & \sigma(n)
   \end{pmatrix}
\]
\subsection*{\subsecstyle{Support {:}}}
On appelle alors \textbf{support} d'une permutation le complémentaire des points fixes de \(\sigma\), ie on a:
\[
   \text{Supp}(\sigma) := \bigl\{ i \in \N \; ; \; \sigma(i) \neq i \bigl\}   
\]
Une des propriétés fondamentale qu'on peut déduire de cette définition est que:
\customBox{width=10cm}{
   Deux permutations à supports disjoints \textbf{commutent}.
}
\subsection*{\subsecstyle{Cycles {:}}}
On appelle \textbf{k-cycle} une permutation \(\sigma\) telle qu'il existe \(k \geq 2\) et \(k\) éléments deux à deux distincts \(a_1, \ldots, a_k\) tels que:
\[
   \begin{cases}
      \forall i \in \inticc{1}{k-1} \; ; \; \sigma(a_i) = \sigma(a_{i+1}) \\
      \forall i \notin \inticc{1}{k} \; ; \; \sigma(a_i) = \sigma(a_{i}) \\
      \sigma(a_k) = \sigma(a_1)
   \end{cases} 
\]
\begin{center}
   \textit{Un k-cycle laisse fixe tout les éléments sauf pour une certaine famille \((a_i)\) pour laquelle chaque élément est envoyé sur le suivant.}
\end{center}
On peut alors noter un tel cycle par la notation suivante qui décrit tout les éléments affectés par la permutation:
\[
   \sigma = (a_1, \ldots, a_n)  
\]

Le cas particulier des \(2\)-cycles est intéressant, en effet un \(2\)-cycle \textbf{échange deux valeurs} de \(\inticc{1}{n}\), ils sont d'une importance particulière et on les appelle \textbf{transpositions}.\<

\underline{Exemple:} La permutation suivante est un 3-cycle:
\[
   \sigma =  \begin{pmatrix}
      1 & 2 & 3 & 4\\
      2 & 3 & 1 & 4
   \end{pmatrix} = (1 \;\, 2 \;\, 3)
\]
Si \(\sigma\) est un \(k\)-cycle et que \(a\) n'est pas un point fixe, alors on en déduit\footnote[1]{En effet par exemple \(\sigma(a)\) est la valeur suivante dans le cycle, et le cycle parcourt tout les points non-fixes par construction} que le support de \(\sigma\) est donné par:
\[
   \text{Supp}(\sigma) = \bigl\{a, \sigma(a), \sigma^2(a), \ldots, \sigma^{k-1}(a) \bigl\}
\]

\subsection*{\subsecstyle{Ordre {:}}}
On peut alors démontrer une propriété fondamentale de l'ordre des cycles:
\customBox{width=7cm}{
   \textbf{Un \(k\)-cycle est d'ordre \(k\)}.
}
En effet si on considère le sous-groupe engendré par un tel cycle, on remarque que pour tout élément \(a \in \inticc{1}{n}\) \(\sigma^{k}(a) = a\), donc \(\sigma^{k} = \text{Id}\).

\subsection*{\subsecstyle{Théorèmes de décomposition {:}}}
Une des problématiques principales à propos des groupes symétriques est la question de la \textbf{décomposition d'une permutation} en cycles. On peut en effet montrer que \textbf{toute permutation se décompose en produit de cycles à support disjoints}.\<

Pour ceci, on utilise le fait que toute permutation induit une \textbf{partition en orbites} de \( \inticc{1}{n} \), ces orbites correspondront alors au supports des cycles dans la décomposition.\<

Par la suite, on peut alors constater directement que pour tout \(k\)-cycle \(\sigma = (a_1 \;\, \ldots \;\, a_k)\), on a:
\customBox{width=6cm}{
   \(\sigma = (a_1 \;\, a_2)(a_2 \;\, a_3)\ldots(a_{k-1} \;\, a_k)\)
}
Enfin, on conclura de ces deux propositions que \textbf{toute permutation se décompose en produit de transpositions}, ou en d'autres termes si on note \(\mathfrak{T}_n\) l'ensemble des transpositions:
\customBox{width=3cm}{
   \(\langle \mathfrak{T}_n \rangle = \mathfrak{S}_n\)
}
\subsection*{\subsecstyle{Conjugaison et permuations {:}}}
On considère alors l'action de \(  \mathfrak{S}_n \) sur lui-même par conjugaison, on peut alors montrer que pour toute permutation \(\sigma\), on a:
\[ 
   \sigma(a_1, \ldots, a_n)\sigma^{-1} = (\sigma(a_1), \ldots, \sigma(a_n))
\]
En particulier, on a alors que deux cycles sont conjugués si et seulement si ils ont la même longueur, et si on définit le \textbf{type d'une permutation} par le n-uplet \textbf{non ordonné} \( [l_1, \ldots, l_k] \) des longueurs des cycles dans sa décomposition en cycles, on a alors une caractérisation des classes de conjugaisons:
\begin{center}
   \textbf{Deux permutations sont conjuguées si et seulement si elles ont même type.}
\end{center}

\subsection*{\subsecstyle{Signature {:}}}
A REFAIRE.

\chapter*{\chapterstyle{II --- Groupes Cycliques}}
\addcontentsline{toc}{section}{Groupes Cycliques}
On appelle \textbf{groupe cyclique} un groupe \(G\) engendré par un unique élément qu'on notera \(g\). Le but de ce chapitre est de classifier ces groupes et d'identifier leurs caractéristiques.\<

Dans toute la suite, on utilisera le morphisme surjectif\footnote[1]{Car \(\langle g \rangle = G\) donc tout les éléments de \(G\) s'écrivent comme une puissance de \(g\).} suivant:
\[
   \begin{aligned}
      \phi: \Z &\longrightarrow G\\
      n &\longmapsto g^n
   \end{aligned}
\]

\subsection*{\subsecstyle{Cas infini {:}}}
Dans le cas ou \(G\) est infini, le morphisme \(\phi\) est injectif et donc on a la caractérisation suivante:
\customBox{width=2cm}{
   \(
      G \cong \Z
   \)
}
\begin{center}
   \textit{Il n'y a donc qu'un seul groupe cyclique d'ordre infini, celui des entiers naturels.}
\end{center}
\subsection*{\subsecstyle{Cas fini {:}}}
Dans le cas ou \(G\) est d'ordre \(n \in \N\), on peut utiliser le \textbf{premier théorème d'isomorphisme} pour montrer qu'il existe un isomorphisme:
\[
   \psi : \Z/\Ker{\phi} \longrightarrow G
\]
Et donc en particulier sachant que \(\Ker{\phi} = n\Z\), on a:
\customBox{width=3cm}{
   \(
      G \cong \Z/n\Z
   \)
}
\begin{center}
   \textit{Il n'y a donc qu'un seul groupe cyclique d'ordre n, celui des classes de congruences modulo n.}
\end{center}
\subsection*{\subsecstyle{Indicatrice d'Euler {:}}}
On définit \textbf{la fonction indicatrice d'Euler} par:
\[
   \begin{aligned}
      \varphi: \N &\longrightarrow \N\\
      n &\longmapsto n \prod_{p|n}{\left(1-\frac{1}{p}\right)}
   \end{aligned}
\]
Le produit se faisant sur tout les diviseurs premiers distincts de \(n\). L'utilité de cette fonction vient de la propriété suivante:
\customBox{width=13cm}{
   \textbf{Le nombre d'entiers inférieurs à \(n\) et premiers avec \(n\) est de \(\varphi(n)\).}
}

\underline{Exemple:} \(\varphi(30) = \varphi(2\times3\times5) = 30\times\left(1-\frac{1}{2}\right)\left(1-\frac{1}{3}\right)\left(1-\frac{1}{5}\right) = 30\times\frac{1}{2}\times\frac{2}{3}\times\frac{4}{5} = 8\)
\subsection*{\subsecstyle{Propriétés {:}}}
On peut alors de démontrer les propriétés suivantes:

\begin{center}
   \begin{itemize}
      \item Tout groupe cyclique est \textbf{abélien}.
      \item Tout \textbf{sous-groupe} d'un groupe cyclique est cyclique.
   \end{itemize}
\end{center}

\chapter*{\chapterstyle{II --- Anneaux}}
\addcontentsline{toc}{section}{Anneaux}
Soit \(A\) un ensemble \textbf{non-vide} muni de deux lois de composition internes associatives notées \(+, \times\) telles que:
\begin{align*}
   &\bullet \;\; (A, +) \text{ soit un groupe commutatif.} \\
   &\bullet \;\; \text{La loi \(\times\) est associative.}\\
   &\bullet \;\; \text{La loi \(\times\) est distributive sur la loi \(+\).}\\
   &\bullet \;\; \text{Il existe \textbf{un élément neutre} pour la loi \(\times\).}
\end{align*}
Alors le triplet \((A, +, \times)\) est appellé \textbf{anneau}. Si la loi multiplicative est \textbf{commutative}, on dira alors que c'est un anneau commutatif.
\subsection*{\subsecstyle{Exemples {:}}}
On peut alors considérer plusieurs anneaux remarquables:
\begin{itemize}
   \item Les \textbf{entiers relatifs} muni des opérations usuelles.
   \item Les \textbf{fonctions continues} muni de la somme et du produit.
   \item Les \textbf{polynomes}\footnote[1]{A coefficients dans un anneau} muni de la somme et du produit.
   \item Les \textbf{matrices}\footnote[1]{A coefficients dans un anneau} muni de la somme et du produit.
\end{itemize}

\subsection*{\subsecstyle{Propriétés Algébriques{:}}}
Pour deux éléments \(a, b \in A\) qui commutent, on a \textbf{la formule du binome de Newton}:
\[
   (a + b)^n = \sum_{k=0}^{n}\binom{n}{k} a^k b^{n-k}   
\]

\subsection*{\subsecstyle{Sous-anneaux {:}}}
Les sous-structures dans le cas des groupes sont naturellement les \textbf{sous-anneaux}. Un cas remarquable est celui du \textbf{sous-anneau engendré} par \(H\) qu'on note:
\customBox{width=10cm}{
   \(\langle H \rangle := \Bigl\{ \sum_{k=1}^{n} \pm h_1^{k_1}h_2^{k_2} \ldots h_n^{k_n}\; ; \; n \in \N \; , \; h_i \in H \; , \; k_i \in \N \Bigl\}\)
}
\begin{center}
   \textit{C'est l'ensemble des produits et sommes d'éléments de \(H\) et de leurs inverses pour la loi de groupe.}
\end{center}
Une propriété fondamentale est que \(\langle H \rangle\) est un \textbf{opérateur de cloture} par la loi du groupe, ie c'est une application \textbf{idempotente, croissante et extensive}.
\subsection*{\subsecstyle{Idéaux{:}}}
On appelle \textbf{idéal} tout sous groupe additif de \(A\) qui soit stable par multiplication (à droite et à gauche) par n'importe quel élément de l'anneau.
\begin{center}
   \textit{Les idéaux jouent alors le meme role que les sous-groupes normaux, ie on peut quotienter par ceux-ci.}
\end{center}
On peut alors montrer les propriétés suivantes:
\begin{itemize}
   \item La préimage d'un idéal par un morphisme est un idéal\footnote[2]{Donc en particulier, les noyaux de morphismes sont toujours des idéaux.}.
   \item L'intesection d'une famille d'idéaux est un idéal.
\end{itemize}

\subsection*{\subsecstyle{Inversibles {:}}}
On dit qu'un élément \(x \in A^*\) est \textbf{inversible} à droite\footnote[2]{On définit de meme les inversibles à gauche} si et seulement si il existe \(y \in A\) tel que:
\[
   xy = 1
\]
Si un élément est inversible bilatère, on dira alors simplement qu'il est inversible. L'ensemble des inversibles d'un anneau forme un groupe pour la loi multiplicative qu'on note \(\mathbb{U}(A)\).
\subsection*{\subsecstyle{Diviseurs de zéro {:}}}
On dit qu'un élément \(x \in A^*\) est \textbf{un diviseur de zéro} à droite\footnote[3]{On définit de meme les diviseurs de zéro à gauche} si et seulement si il existe \(y \in A^*\) tel que:
\[
   yx = 0
\]
\underline{Exemple}: Dans l'anneau \(\mathcal{M}_2(\R)\), la matrice \(\begin{pmatrix}1 & 1 \\ 0 & 0\end{pmatrix}\) est un diviseur de zéro car \(\begin{pmatrix}1 & 1 \\ 0 & 0\end{pmatrix}\begin{pmatrix}1 & 0 \\ -1 & 0\end{pmatrix} = 0\)
\subsection*{\subsecstyle{Nilpotents {:}}}
On dit qu'un élément \(a \in A\) est \textbf{nilpotent} si et seulement si il existe \(n \in \N\) tel que:
\[
   a^n = 0
\]
En particulier les nilpotents sont donc des diviseurs de zéro. \<

\underline{Exemple}: Dans l'anneau \(\mathcal{M}_2(\R)\), la matrice \(A = \begin{pmatrix}0 & 1 \\ 0 & 0\end{pmatrix}\) est nilpotente car \(A^2 = 0\)
\subsection*{\subsecstyle{Caractéristique {:}}}
On définit la caractéristique d'un anneau non-nul par:
\[
   \text{char}(A) := \min \left\{ n \in \N \; ; \;\underbrace{1 + \ldots + 1}_{\text{\(n\) sommandes}} = 0 \right\}
\]
Une autre formulation serait simplement que:
\begin{center}
   \textit{La caractéristique d'un anneau est l'ordre (additif) de l'unité multiplicative.}
\end{center}
\subsection*{\subsecstyle{Anneaux intègres {:}}}
On appelle \textbf{anneau intègre} tout anneau \(A\) non-nul \textbf{commutatif} et \textbf{sans diviseurs de zéro}. En particulier, dans un anneau intègre, on a alors la propriété qu'un produit est nul si et seulement si \textbf{l'un des facteurs est nul}.
\subsection*{\subsecstyle{Anneaux à PGCD {:}}}
\subsection*{\subsecstyle{Anneaux Factoriels {:}}}
\subsection*{\subsecstyle{Anneaux Principaux {:}}}
\subsection*{\subsecstyle{Anneaux Euclidiens {:}}}
On appelle \textbf{anneau Euclidiens} tout anneau \(A\) principal qui possède une \textbf{division euclidienne}. Dans un tel anneau, on peut alors faire \textbf{de l'arithmétique} comme dans l'anneau des entiers naturels.
\subsection*{\subsecstyle{Schéma heuristique des structures d'anneaux {:}}}
Pour mieux visualiser la hierarchie des différents types d'anneaux, on peut représenter la structure logique sous la forme de la suite d'implications suivantes:
\begin{center}
   \customBox{width = 16cm}{
      \textbf{Euclidien} \(\implies\) \textbf{Principal} \(\implies\) \textbf{Factoriel} \(\implies\) \textbf{PGCD} \(\implies\) \textbf{Intégre} \(\implies\) \textbf{Commutatif}
   }
\end{center}
\chapter*{\chapterstyle{II --- Corps}}
\addcontentsline{toc}{section}{Corps}
Soit \(A\) un anneau dont tout les éléments sauf \(0\) sont inversibles. Alors on dit que \(A\) est \textbf{un corps}.

\subsection*{\subsecstyle{Exemples {:}}}
On peut alors considérer plusieurs corps remarquables:
\begin{itemize}
   \item Les \textbf{réels} muni des opérations usuelles.
   \item Les \textbf{quaternions}\footnote[1]{C'est un exemple de corps non commutatif} muni des opérations usuelles.
   \item Les \textbf{corps finis} \(\mathbb{F}_p = \Z/p\Z\) pour \(p\) premier.
   \item Les \textbf{nombres constructibles} à la règle et au compas.
\end{itemize}

\chapter*{\chapterstyle{II --- Arithmétique dans \(\Z\)}}
\addcontentsline{toc}{section}{Arithmétique dans Z}

\subsection*{\subsecstyle{Division Euclidienne {:}}}
Soit \(a, b \in \Z \times \Z^*\), on montre qu'il existe un unique couple \((q, r) \in \Z \times \N\) avec \(r < |b|\) tel que:

\customBox{width=4cm}{
   \(
      a = bq + r
   \)
}
On appelle alors cette décomposition \textbf{la division euclidienne} de \(a\) par \(b\).

\subsection*{\subsecstyle{Plus grand diviseur commun {:}}}

Soit \(a, b \in \Z\) non simultanément nuls, alors le pgcd est l'entier \(d\) qui vérifie:
\customBox{width=5cm}{
   \begin{align*}
      &\bullet \;\; d \, | \, a \text{ et } d \, | \, b \\
      &\bullet \;\; d' \, | \, a \text{ et } d' \, | \, b \implies d' \leq d
   \end{align*}
}
Alors on l'appelle \textbf{plus grand diviseur commun} de \(a\) et de \(b\) et on le note \(a \wedge b\). Pour le trouver en pratique, on peut utiliser l'algorithme d'Euclide. \<

Une caractérisation utile permet de montrer que si \(d\) est un diviseur commun à \(a\) et \(b\), alors \(d = \text{pgcd}(a , b)\) si et seulement si pour tout diviseur commun \(d'\) de \(a\) et \(b\):
\customBox{width=2cm}{
   \(
      d' \, | \, d   
   \)
}

\subsection*{\subsecstyle{Théorème de Bézout {:}}}
Soit \(a, b\in \Z^2\), on peut montrer qu'il existe deux entiers \(n_1, n_2 \in \Z^2\) tels que:
\customBox{width=5cm}{
   \(
      n_1a+n_2b = a \wedge b
   \)
}
\begin{center}
   \textit{
       Il existe donc une combinaison linéaire (à coefficients entiers) de \(a, b\) qui donne leur PGCD.
   }
\end{center}

\subsection*{\subsecstyle{Lemme de Gauss {:}}}
Soit 3 entiers \(a, b, c \in \Z\), alors\footnote[1]{Partir d'une combinaison linéaire donnée par Bézout.}:
\customBox{width=5.5cm}{
   \[
      \begin{cases} 
         a \, | \, bc \\
         a \wedge b = 1
      \end{cases} \implies a \, | \, c
   \]
}
\chapter*{\chapterstyle{II --- Corps des Complexes}} % 99% Fini
\addcontentsline{toc}{section}{Corps des Complexes}

On définit le nombre imaginaire \(i\) dont le carré vaut \(-1\), et on construit alors \(\C\) comme l'extension du corps\footnote[1]{La motivation principale de l'introduction de \(i\) et de cette construction est que \(\C\) est algébriquement clos, ie tout les polynomes de degré \(n\) de \(\C[X]\) ont \(n\) racines.} \(\R\) avec les deux lois usuelles, ie on définit:
\[
   \C := \R[i] = \Bigl\{a + ib \; ; \; a, b \in \R \Bigl\}
\]
On peut alors montrer que c'est un ensemble stable pour les lois usuelles et qu'il vérifie toutes les propriétés qui font de lui un \textbf{corps}.\+
Chaque nombre complexe se définit alors comme des sommes ou produits de réels et du nombre imaginaire et on appelle alors cette expression la \textbf{forme algébrique} d'un nombre complexe et on appelle \(a\) \textbf{la partie réelle} et \(b\) \textbf{la partie imaginaire} de ce nombre. \<

Géométriquement, on peut identifier les nombres complexes à des points du plan, en effet, \(a + ib\) peut se comprendre comme une combinaison linéaire d'un nombre de l'axe réel, et d'un nombre de l'axe imaginaire.

\subsection*{\subsecstyle{Module {:}}}
On appelle \textbf{module} de \(z \in \C\) le \textbf{prolongement} de la fonction valeur absolue à \(\C\), c'est donc une \textbf{norme} et on la définit telle que {:}
\[
   |z| = \sqrt{a^2 + b^2} = \sqrt{z\overline{z}} 
\]
Dans la suite, on notera \(\rho\) le module de \(z\) pour faciliter la lecture.

\subsection*{\subsecstyle{Forme trigonométrique {:}}}
L'interprétation géométrique permet alors de montrer par passage en coordonées polaires qu'il existe un unique angle \(\theta\) (modulo \(2\pi\)) qu'on appelle \textbf{argument} de \(z\) tel que:
\[
    z = \rho(\cos\theta+i\sin\theta)   
\]

\subsection*{\subsecstyle{Forme exponentielle {:}}}
De même on définit alors \textbf{la forme exponentielle} de \(z\) l'expression:
\[
    z = \rho e^{i\theta} := \rho(\cos\theta+i\sin\theta)
\]
On peut alors étendre les propriétés usuelles de l'exponentielle à \(\C\) et on en déduit:
\customBox{width=6cm}{
   \begin{align*}
      \arg(zz') &\eqmod{2\pi} \arg(z) + \arg(z')\\
      \arg(\frac{z}{z'}) &\eqmod{2\pi} \arg(z) - \arg(z')
   \end{align*}
}

\subsection*{\subsecstyle{Conjugué {:}}}
On appelle conjugaison \textbf{l'involution} qui à \(z\) associe son \textbf{conjugué}, noté \(\overline{z}\) tel que:
\[
    \overline{z} := a - bi = \rho(\cos\theta - i\sin\theta) = \rho e^{-i\theta}
\]
C'est une application \textbf{additive} et \textbf{multiplicative}, on montre alors les formules suivantes {:}
\customBox{width=7.5cm}{
   \[
      \Re(z) := \frac{z + \overline{z}}{2}
      \quad\quad\quad\quad\quad\quad
      \Im(z) := \frac{z - \overline{z}}{2i}
   \]
}

En utilisant ces formules pour \(z\) sous forme exponentielle, on a alors les \textbf{formules d'Euler} qui sont très importantes car elle permettent de \textbf{linéariser} des expression trigonométriques.\<

\subsection*{\subsecstyle{Formule de Moivre {:}}}
Un propriété importante des formes trigonométriques et exponentielles apellée \textbf{formule de Moivre}\footnote[1]{Ici, on a choisi de considérer \(z \in \U\) mais ces propriétés sont vraies pour \textbf{tout nombre complexe}, il suffit alors d'appliquer la puissance au module.} est:
\customBox{width=9cm}{
   \begin{align*}
      (e^{i\theta})^n &= e^{n(i\theta)}\\
      (\cos\theta + i\sin\theta)^n &= \cos n\theta + i\sin n\theta
  \end{align*}  
}
\begin{center}
    \textit{Les différentes puissances d'un nombre complexe (de module 1) s'interprétent alors comme des points situés à equidistance sur un cercle.}
\end{center}
Graphiquement:
\begin{center}
   \begin{tikzpicture}[yscale=2, xscale=2]
      \coordinate (O) at (0,0);
      \coordinate (X) at (1.5,0);
      \coordinate (Z1) at (0.86, 0.5);
      \coordinate (Z2) at (0.5, 0.86);
   
      \draw[-latex] (-1.5,0) -- (X) node [right] {$\mathbb{R}$};
      \draw[-latex] (0, -1.5) -- (0,1.5) node [above] {$i\mathbb{R}$};
   
      \draw[color = DarkBlue1, thick] (0,0) circle (1cm);
   
      \draw node[color = BrightRed1] at (1.25, 0.6) {$z = e^{i\cdot\theta_1}$}; 
      \draw node[color = BrightRed1] at (0.95, 1) {$z^2 = e^{i\cdot 2\theta_1}$}; 
   
      \draw[color = BrightRed1!75, thick] (O) -- (Z1);
      \pic [color = BrightRed1, draw, thick, "$\theta_1$", angle eccentricity=1.5, angle radius=0.5cm] {angle = X--O--Z1};
   
      \draw[color = BrightRed1!75, thick] (O) -- (Z2);
      \pic [color = BrightRed1, draw, thick, "$\theta_1$", angle eccentricity=1.5, angle radius=0.75cm] {angle = Z1--O--Z2};
   
      \fill[color = BrightRed1!100] (0.21, 0.67) circle[radius=0.5pt];
      \fill[color = BrightRed1!80] (0.073, 0.70) circle[radius=0.5pt];
      \fill[color = BrightRed1!60] (-0.07, 0.70) circle[radius=0.5pt];
      \fill[color = BrightRed1!40] (-0.21, 0.67) circle[radius=0.5pt];
      \fill[color = BrightRed1!20] (-0.35, 0.61) circle[radius=0.5pt];
   \end{tikzpicture}  
\end{center}

\subsection*{\subsecstyle{Racines n-ièmes {:}}}
Soit \(n \in \N\), une partie importante des problèmes impliquant des nombres complexes proviennent d'équations d'inconnue \(Z\) de la forme:
\[
   Z^n = z
\]
On peut montrer que l'ensemble des solutions de ce type de problème est:
\customBox{width=8cm}{
   \begin{align*}
      S = \Bigl\{ \sqrt[n]{\rho}e^{i\frac{\theta + 2k\pi}{n}}\; ; \; k \in \bigl\{0, 1, \ldots , n-1 \bigl\} \Bigl\}   
  \end{align*}  
}
\underline{Cas particulier {:}}
Si on a une racine n-ième \(Z_0\) de \(Z\) et qu'on connaît les racines n-ièmes de l'unité, alors on peut obtenir toutes les racines n-ièmes de \(Z\) grâce à:

\[
   \Bigl\{ Z \in \C \; ; \; Z^n = z \Bigl\} = \Bigl\{ Z_0u \; ; \; u \in \U_n \Bigl\}   
\]

\subsection*{\subsecstyle{Le nombre complexe \(j\) {:}}}
On note \(j\) la première racine troisième de l'unité.
Le nombre \(j\) est singulier, car il vérifie:
\customBox{width=4cm}{
   \(
      j^2 = j^{-1} = \overline{j}
   \)
}

Graphiquement, on peut observer que les affixes des nombres \(1\), \(j\) et \(\overline{j}\) forment un triangle équilatéral inscrit dans le cercle trigonométrique.
\chapter*{\chapterstyle{II --- Anneau des Polynômes}} % 75% Fini
\addcontentsline{toc}{section}{Anneau des Polynômes}

\subsection*{\subsecstyle{Définition {:}}}
Soit \(K\) un corps, et \(n \in \N\), on appelle \textbf{polynômes} à coefficients dans \(K\) en l'indéterminée \(X\) les éléments de l'ensemble:
\[
   K[X] := \Biggl\{ \; \sum_{i=0}^n a_{i}X^{i} \; ; \; a_i \in K \;\Biggl\}
\]

\subsection*{\subsecstyle{Degré et Valuation {:}}}
Soit \(P, Q, R \in K[X]\).\+
On définit tout d'abord une propriété fondamentale appelée \textbf{degré} de \(P\) telle que deg\((P)\) est le plus grand coefficient non nul de P.
On a alors les propriétés du degré ci-dessous:
\customBox{width=7cm}{
   \begin{align*}
      \deg(P + Q) &\leq \max(\deg(P), \deg(Q))\\
      \deg(PQ) &= \deg(P) + \deg(Q)
  \end{align*}
}

La valuation est définie de manière analogue comme le plus petit coefficient non nul de P.
\subsection*{\subsecstyle{Opérations {:}}}
On considère ci-dessous que deg\((P)\) = \(n\) et deg\((Q)\) = \(m\) et on note \(a_i, b_i\) les coefficients de \(P\) (resp. \(Q\)).\<

On adjoint à cet ensemble une loi d'addition qui est simplement effectuée terme à termes.\+
On adjoint à cet ensemble une loi de multiplication se définit explicitement comme suit:
\[
   PQ := \sum_{k=0}^{n+m}\sum_{i+j = k} a_{i}b_{i}X^{k}
\]

Muni de ces deux opérations, on donne à cet ensemble une structure \textbf{d'anneau}\footnote[1]{Intègre car les coefficients viennent d'un corps.}.\<

On ajoute aussi une opération appelée \textbf{dérivation formelle} d'un polynôme qui s'effectue comme la dérivation analytique usuelle.
\subsection*{\subsecstyle{Divisibilité {:}}}
On définit tout d'abord une \textbf{division euclidienne} de deux polynômes qui se comporte comme la division euclidienne usuelle, à la différence que la condition d'arrêt porte sur le \textbf{degré du reste} qui doit être inférieur à celui du diviseur.\<

Soit \(U, V \in R[X]^2\), on peut aussi définir une relation de \textbf{divisibilité} entre deux polynômes, cette relation est \textbf{transitive et réflexive} et on a aussi:
\customBox{width=6.5cm}{
  \(
      D|A \land D|B \implies D| UA + VB   
  \)
}

\subsection*{\subsecstyle{Racines et Factorisation {:}}}
On dit que \(\alpha\) est une \textbf{racine} de \(P\) si \(P(\alpha) = 0\).\+
On montre alors le théorème fondamental ci-dessous:
\customBox{width=5cm}{
  \(
      P(\alpha) = 0 \Longleftrightarrow (X - \alpha) | P  
   \)
}
\pagebreak

On appelle \textbf{multiplicité d'une racine} \(\alpha\) l'entier \(m\) tel que:
\[
   \Bigr[ (X - \alpha)^m | P \Bigr] \; \land \; \Bigr[(X - \alpha)^{m+1} \nmid P\Bigr]
\]
Si on note \(P^{m}\) la dérivée n-ième de \(P\), on a aussi:
\[
   P^{m}(\alpha) = 0 \; \land \; P^{m + 1}(\alpha) \neq 0
\]

On en déduit que pour une racine \(\alpha\) de multiplicité \(m\), on peut \textbf{factoriser} \(P\) par \((X-\alpha)^m\).\<

Si on considère maintenant plusieurs racines \textbf{distinctes} \(a_0, a_1, \ldots, a_{n-1}, a_n\) de multiplicité respectivement \(m_0, m_1, \ldots, m_{n-1}, m_n\), on peut montrer la propriété suivante:
\begin{flalign*}
   \Bigr[ \prod_{i=0}^{n}(X-\alpha_i)^{m_i} \Bigr] \; \Biggr| \; P \shorteqnote{(On peut factoriser par le produit des \((X-\alpha_i)^{m_i}\))}
\end{flalign*}
\subsection*{\subsecstyle{Décomposition {:}}}
On appelle décomposition d'un polynôme \(P\) une factorisation \textbf{irréductible} de \(P\), cette décomposition dépend du corps considéré, en effet si on considère \(\C[X]\), on peut montrer le \textbf{théorème fondamental de l'Algèbre} ci-dessous:
\customBox{width=11cm}{
   Tout polynôme de degré \(n\) admet \(n\) racines dans \(\C\).
}

Par suite, on peut montrer que tout les polynômes de \(\C[X]\) sont \textbf{scindés}\footnote[1]{C'est à dire que tout les polynômes irréductibles de \(\C[X]\) sont de degré 1.}.\+
Par contre dans \(\R[X]\), il existe évidemment des polynômes de degré 2 irréductibles.\<

Soit \(z \in \C\), il existe une propriété très utile pour décomposer un polynôme \textbf{à coefficients réel} qui est:
\[
   P(z) = 0 \implies P(\overline{z}) = 0
\]
\subsection*{\subsecstyle{Fonctions symétriques des racines {:}}}
On définit le \(k\)-ième \textbf{polynôme symétrique} à \(n\) indéterminées comme la somme des produits à \(k\) facteurs de ses indéterminées, et on le note \(\sigma_k\). \<

Par exemple pour un polynôme en trois indéterminées \(a, b, c\), on a successivement:
\begin{flalign*}
    &\sigma_1 = a + b  + c\shorteqnote{(Somme des produits à 1 facteur)}\\
    &\sigma_2 = ab + ac + bc \shorteqnote{(Somme des produits à 2 facteurs)}\\
    &\sigma_3 = abc\shorteqnote{(Somme des produits à 3 facteurs)}
\end{flalign*}
De manière générale on a:
\[
    \sigma_k(x_1, \ldots, x_n) = \sum_{1 \leq i_1 < \ldots < i_k \leq n} x_{i_1}x_{i_2} \ldots x_{i_k}
\]
Soit \(P \in K[X]\) un polynôme de degré \(n\), on note \(x_1, x_2, \ldots, x_{n-1}, x_n\) ses \(n\) racines et \(a_1, a_2, \ldots, a_{n-1}, a_n\) ses \(n\) coefficients.\+
Alors, pour tout \(k \in \inticc{0}{n}\), on a le \textbf{théorème}:
\customBox{width=8cm}{
   \[\sigma_k(x_1, x_2, \ldots, x_{n-1}, x_n) = (-1)^{k}\frac{a_{n-k}}{a_n}\]
}

Ce théorème permet d'obtenir une relation \textbf{coefficient-racines}.\<

\underline{Exemple:} \(\begin{cases}
    &\text{Pour \(k = 1\), on trouve que la somme des racines de \(P\) vaut \(-\frac{a_{n-1}}{a_n}\)}\\
    &\text{Pour \(k = 2\), on trouve que la somme des doubles produits des racines de \(P\) vaut \(\frac{a_{n-2}}{a_n}\)}\\
    &\ldots\ldots\ldots\ldots\ldots\ldots
\end{cases}\)\<
