\documentclass{report}
% Maths Packages
\usepackage{mathtools, amsthm, amssymb, mathrsfs, interval, stmaryrd, centernot, esvect, cancel, commath, blkarray, empheq}
\usepackage{tabularx}
\usepackage{booktabs}
\usepackage{cellspace}
\setlength{\cellspacetoplimit}{5pt}
\setlength{\cellspacebottomlimit}{5pt}

% Sagemaths Formating Packages
\usepackage{listings}
\lstdefinelanguage{Sage}[]{Python}
{morekeywords={False,sage,True},sensitive=true}
\lstset{
  frame=none,
  showtabs=False,
  showspaces=False,
  showstringspaces=False,
  commentstyle={\ttfamily\color{dgreencolor}},
  keywordstyle={\ttfamily\color{dbluecolor}\bfseries},
  stringstyle={\ttfamily\color{dgraycolor}\bfseries},
  language=Sage,
  basicstyle={\fontsize{10pt}{10pt}\ttfamily},
  aboveskip=0.4em,
  belowskip=0.4em,
}

% TOC Packages
\usepackage{tocloft, titletoc, hyperref, bookmark}
% Formatting / Style Packages
\usepackage[T1]{fontenc}
\usepackage{geometry, subcaption, graphicx, fix-cm, accents, float, varwidth, soul, ulem, contour, multicol, enumitem}    
\usepackage[bottom]{footmisc}
\usepackage[x11names, table]{xcolor}
\usepackage[most, skins]{tcolorbox}
\usepackage{adjustbox}
\DeclareMathAlphabet{\mathmybb}{U}{bbold}{m}{n} % Indicatrices
\newcommand{\1}{\mathmybb{1}}

% Tikz
\usepackage{tikz, tkz-fct, tkz-euclide, tikz-cd, tkz-fct, pgfplots}
\pgfplotsset{compat=1.18}
\usetikzlibrary{
  angles, quotes, 3d, positioning,
  shapes,fit, arrows, arrows.meta, calc, 
  matrix, calligraphy, intersections, 
  quotes, patterns, patterns.meta, 
  decorations.pathreplacing, decorations.markings,decorations.pathmorphing,
}
\usepgfplotslibrary{fillbetween}
\tikzset{
  withparens/.style = {draw, outer sep=0pt,
    left delimiter= (, right delimiter=),
    above delimiter= (, below delimiter=),
    align=center},
  withbraces/.style = {draw, outer sep=0pt,
    left delimiter=\{, right delimiter=\},
    above delimiter=\{, below delimiter=\},
    align=center}
}
\tikzcdset{
  arrow style=tikz,
  diagrams={>={Straight Barb[scale=1]}},
}

% PAGE SETTINGS

\geometry{
  left=25mm, right=25mm, top= 15mm, bottom= 15mm,
  footskip=30pt
  }
\setlength{\parindent}{0cm}
\setlength{\parskip}{0cm}
\setlist[itemize]{itemsep=0pt, leftmargin=25pt}

\setlength{\cftbeforetoctitleskip}{0pt}
\setlength{\cftaftertoctitleskip}{0pt}


\begin{document}
   \chapter*{\chapterstyle{Le logarithme}}

   On considère la fonction \(x \in \R_+^* \mapsto \frac{1}{x}\), alors on sait que cette fonction est continue, et meme de classe \(\mathcal{C}^\infty\) sur son domaine de définition, en particulier elle est donc \textbf{Riemann-intégrable} sur un segment de celui-ci.\<

   On définit alors une nouvelle fonction appellée \textbf{logarithme népérien} définie sur \(\R_+^*\) par:
   \customBox{width=4cm}{
      \[
         \ln: x \longmapsto \int_1^x \frac{1}{t} \d t
      \]
   }
   Cette fonction est bien définie comme intégrale d'une fonction continue sur tout segment de \(\ioo{0}{+\infty}\), et en particulier, elle est bien unique en tant \textbf{qu'unique primitive de \(f\) qui s'annule en 1}.

   \subsection*{\subsecstyle{Lien avec l'exponentielle {:}}}
   Une propriété fondamentale, qu'on démontrera par la suite, est la suivante:
   \begin{center}
      \textit{Le logarithme est la bijection réciproque de l'exponentielle.}
   \end{center}
   Formellement, on a:
   \begin{align*}
      &\bullet \;\; \forall x \in \R ; \ln(e^x) = x   \\
      &\bullet \;\; \forall x \in \R_+^* ; e^{\ln(x)} = x
   \end{align*}
   Cette propriété sera démontrée en fin de chapitre gràce au \textbf{théorème d'inversion}, vu en cours d'analyse.

   \subsection*{\subsecstyle{Propriétés algébriques {:}}}
   Soit \(x, y \in \R_+^*\) et \(\alpha \in \R\) cette définition et des propriétés de l'intégrale et de la fonction \(x \mapsto \frac{1}{x}\), on peut alors déduire les propriétés algébriques suivantes:
   \begin{align*}
      &1) \;\; \ln(xy) = \ln(x) + \ln(y)\\
      &2) \;\; \ln\left(\frac{1}{x}\right) = -\ln(x)\\
      &3) \;\; \ln(x^\alpha) = \alpha\ln(x)\\
      &4) \;\; \ln\left(\frac{x+y}{2}\right) \geq \frac{\ln(x)}{2} + \frac{\ln(y)}{2}
   \end{align*}
   La propriété 1) est la propriété \textbf{fondamentale} du logarithme, en effet, sa caractéristique principale est la suivante:
   \begin{center}
      \textit{Le logarithme transforme les produits en somme.}
   \end{center}
   Plus tard on pourra interpréter ceci par l'idée que cette fonction est (l'unique) morphisme du groupe \((\R_+^*, \times)\) vers le groupe \((\R, +)\). Cette propriété se généralise aisément par récurrence évidente pour tout produit d'un nombre fini de termes. Cette propriété sera démontrée en fin de chapitre gràce directement à partir de la définition.\<

   La propriété 4) est une propriété plus particulière, géométrique, qui caractérise la \textbf{convexité} de la courbe représentative du logarithme, en effet, l'image d'un barycentre de deux points est supérieure au barycentre des images.
   \pagebreak

   \subsection*{\subsecstyle{Propriétés analytiques {:}}}
   Le théorème fondamental du calcul intégral nous permet alors d'affirmer que la fonction \(\ln\) est dérivable, et meme de classe \(\mathcal{C}^1\) sur \(\R_+^*\), et en particulier, on a évidemment:
   \[
      \forall x \in \R_+^* \; ; \; \left(\ln(x)\right)' = \frac{1}{x}
   \]
   Plus précisément, on sait que la fonction \(x \mapsto \frac{1}{x}\) est de classe \(\mathcal{C}^{\infty}\) sur \(\R_+^*\), donc on en déduit que \(\ln(x)\) est aussi de classe \(\mathcal{C}^{\infty}\) sur \(\R_+^*\).\<

   Si \(x > 1\), on a l'intégrale d'une fonction strictement positive sur un segment \(\icc{1}{x}\), donc le logarithme est positif.\+
   Si \(x < 1\), on a l'intégrale d'une fonction strictement positive sur un segment \(\icc{x}{1}\), donc le logarithme est négatif.\<

   Enfin on peut aussi montrer que la fonction \(x \mapsto \ln(1 + x)\) admet \textbf{dévloppement limité} en \(0\) de la forme:
   \[
      \ln(1 + x) \underset{0}{=} x - \frac{x^2}{2} + \frac{x^3}{3} + \ldots +(-1)^{n+1} \frac{x^n}{n} = \sum_{k=1}^n (-1)^{k+1} \frac{x^k}{k}
   \] 

   \subsection*{\subsecstyle{Intérprétation géométrique {:}}}
   On peut aussi déduire une interprétation géométrique de cette définition intégrale, en particulier, on sait que l'intégrale d'un fonction sur un segment est \textbf{l'aire algébrique} entre la courbe représentative de la fonction et l'axe des abscisses.\<

   Donc en particulier, on peut comprendre le logarithme comme l'aire sous la courbe de la fonction inverse, \textbf{comptée négativement} si \(x < 1\), en particulier, on comprends alors que le logarithme est positif sur \(\ico{0}{+\infty}\) et négatif sur \(\ioo{0}{1}\) (ce qui a été algébriquement vérifié au-dessus).

\end{document}