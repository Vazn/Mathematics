\documentclass{report}
% Maths Packages
\usepackage{mathtools, amsthm, amssymb, mathrsfs, interval, stmaryrd, centernot, esvect, cancel, commath, blkarray, empheq}
\usepackage{tabularx}
\usepackage{booktabs}
\usepackage{cellspace}
\setlength{\cellspacetoplimit}{5pt}
\setlength{\cellspacebottomlimit}{5pt}

% Sagemaths Formating Packages
\usepackage{listings}
\lstdefinelanguage{Sage}[]{Python}
{morekeywords={False,sage,True},sensitive=true}
\lstset{
  frame=none,
  showtabs=False,
  showspaces=False,
  showstringspaces=False,
  commentstyle={\ttfamily\color{dgreencolor}},
  keywordstyle={\ttfamily\color{dbluecolor}\bfseries},
  stringstyle={\ttfamily\color{dgraycolor}\bfseries},
  language=Sage,
  basicstyle={\fontsize{10pt}{10pt}\ttfamily},
  aboveskip=0.4em,
  belowskip=0.4em,
}

% TOC Packages
\usepackage{tocloft, titletoc, hyperref, bookmark}
% Formatting / Style Packages
\usepackage[T1]{fontenc}
\usepackage{geometry, subcaption, graphicx, fix-cm, accents, float, varwidth, soul, ulem, contour, multicol, enumitem}    
\usepackage[bottom]{footmisc}
\usepackage[x11names, table]{xcolor}
\usepackage[most, skins]{tcolorbox}
\usepackage{adjustbox}
\DeclareMathAlphabet{\mathmybb}{U}{bbold}{m}{n} % Indicatrices
\newcommand{\1}{\mathmybb{1}}

% Tikz
\usepackage{tikz, tkz-fct, tkz-euclide, tikz-cd, tkz-fct, pgfplots}
\pgfplotsset{compat=1.18}
\usetikzlibrary{
  angles, quotes, 3d, positioning,
  shapes,fit, arrows, arrows.meta, calc, 
  matrix, calligraphy, intersections, 
  quotes, patterns, patterns.meta, 
  decorations.pathreplacing, decorations.markings,decorations.pathmorphing,
}
\usepgfplotslibrary{fillbetween}
\tikzset{
  withparens/.style = {draw, outer sep=0pt,
    left delimiter= (, right delimiter=),
    above delimiter= (, below delimiter=),
    align=center},
  withbraces/.style = {draw, outer sep=0pt,
    left delimiter=\{, right delimiter=\},
    above delimiter=\{, below delimiter=\},
    align=center}
}
\tikzcdset{
  arrow style=tikz,
  diagrams={>={Straight Barb[scale=1]}},
}

% PAGE SETTINGS

\geometry{
  left=25mm, right=25mm, top= 15mm, bottom= 15mm,
  footskip=30pt
  }
\setlength{\parindent}{0cm}
\setlength{\parskip}{0cm}
\setlist[itemize]{itemsep=0pt, leftmargin=25pt}

\setlength{\cftbeforetoctitleskip}{0pt}
\setlength{\cftaftertoctitleskip}{0pt}


\begin{document}
   Dans la suite \((X, \mathscr{B}, \mu)\) est un espace mesuré.
   
   \subsection*{\subsecstyle{Théorême de convergence dominée}}
   On se donne une suite \(f_n \in \mathscr{M}(X)\) qui converge simplement vers \(f\). On suppose qu'il existe \(g\) intégrable telle que \(\forall n \in \N \; ; \; |f_n| \leq g\). Alors on a:
   \begin{itemize}
      \item Pour tout \(n\in \N\), \(f_n\) est intégrable car par croissance de l'intégrale \(\int |f_n| \d\mu \leq \int g \d\mu < \infty\)
      \item Par passage à la limite, \(|f| \leq g\) et donc \(f\) est intégrable par le même raisonnement.
   \end{itemize}
   Montrons maintenant qu'on peut faire l'interversion limite/intégrale, on définit les suites (de fonctions) suivantes:
   \[
      \begin{cases}
         u_n := |f_n - f|\\
         v_n := \sup\{u_k \; ; \; k > n\}\\
         w_n := 2g - v_n
      \end{cases}
   \]
   Soit \(x \in X\), étudions tout d'abord la suite \((v_n)\), on peut montrer les deux propriétés suivantes:
   \begin{itemize}
   \item Elle est décroissante:
   En effet si \(n < m\), on a \(\{u_k(x) \; ; \; k > m\} \subseteq \{u_k(x) \; ; \; k > n\}\) en passant à la borne supérieure on a \(v_m(x) > v_n(x)\)
   \item Elle tends vers 0, en effet \((u_n)\) converge simplement vers 0 et on a:
   \[
      \lim_{n \rightarrow +\infty} v_n(x) =  \lim_{n \rightarrow +\infty} \sup\{u_k(x) \; ; \; k > n\} = \limsup u_n(x) = \lim_{n \rightarrow +\infty} u_n(x) = 0
   \]
   \end{itemize}
   On déduis donc de ces résultats que la suite \(w_n\) est une suite \textbf{croissante de fonctions mesurables}, en effet si \(n > n'\), on a:
   \[
      -v_n(x) > -v_n'(x) 
   \]
   Et donc:
   \[
      2g(x) - v_n(x) > 2g(x) -v_n'(x) 
   \]
   En outre cette suite est positive à partir d'un certain rang car \(v_n\) tends vers 0 et \(g\) est positive (donc \(|v_n| < 2g\) à partir d'un certain rang). On peut donc appliquer \textbf{le théorème de convergence monotone} et on obtient que:
   \[
      \lim_{n \rightarrow +\infty} \int 2g - v_n \d\mu = \int 2g \d\mu
   \]
   En appliquant la linéarité dans l'intégrale de gauche et en calculant la limite de cette manière on obtient alors que:
   \[
      \lim_{n \rightarrow +\infty} \int v_n \d\mu = 0
   \]
   Finalement, pour tout \(n \in \N\) on a que:
   \[
      |f_n - f| \leq \sup\{|f_n - f| \; ; \; k > n\} = v_n
   \]
   Donc en intégrant ces fonctions et en passant à la limite, on trouve:
   \[
      \int |f_n - f| \d\mu \leq \int v_n \d\mu \underset{n \rightarrow \infty}{\longrightarrow} 0
   \]
   Et par inégalité triangulaire, on a: 
   \[
      \left|\int f_n \d\mu - \int f \d\mu  \right| \underset{n \rightarrow \infty}{\longrightarrow} 0
   \]
   Donc:
   \[
      \int f_n \d\mu \underset{n \rightarrow \infty}{\longrightarrow} \int f \d\mu
   \]
\end{document}