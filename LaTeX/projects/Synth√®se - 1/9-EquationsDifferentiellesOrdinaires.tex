\chapter*{\chapterstyle{IX --- Introduction}}
\addcontentsline{toc}{section}{Introduction} 
Soit \(n, p\) des entiers et \(F\) une fonction \textbf{continue} sur \(\Omega \subseteq \R \times (\R^{p})^n\), alors on appelle \textbf{équation différentielle ordinaire} d'ordre \(n\) tout équation de la forme:
\[
   F(t, y(t), \ldots, y^{(n)}(t)) = 0
\]
Où \(y\) est une fonction d'un intervalle \( I \) dans \(\R^p\) à déterminer. Par exemple, on pose:
\[
   \begin{cases}
      F_1(t, y, y') = y'(t) - y(t)\\
      F_2(t, y, y') = cos(t)y'(t) - y^2(t)\\
      F_3(t, y, y') = 3 + arctan(t) + y'(t) - e^xy(t)\\ 
      F_4(t, y, y', y'') = t^2 + 3y(t)y'(t) + cos(y''(t))
   \end{cases} \implies
   \begin{cases}
      E_1 : y'(t) - y(t) = 0 \\
      E_2 : cos(t)y'(t) - y^2(t) = 0 \\
      E_3 : 3 + arctan(t) + y'(t) - e^xy(t) = 0 \\
      E_4 : t^2 + 3y(t)y'(t) + cos(y''(t)) = 0
   \end{cases}
\]
On peut aussi remarque que cette définition permet à \(y\) d'être à valeurs vectorielles et dans ce cas on obtient alors un \textbf{système différentiel}, par exemple pour \(E = \R^2\) et \(F_1\), on obtient:
\[
   E : \begin{cases}
      y'_1(t) = y_1(t)\\
      y'_2(t) = y_2(t)
   \end{cases}
\]
\subsection*{\subsecstyle{Forme résolue{:}}}
Dans des cas trés précieux, on peut isoler la plus grand dérivée, et on dira alors que l'équation différentielle est \textbf{sous forme résolue} si et seulement si il existe une fonction \(F\) continue sur \(\R \times (\R^p)^{n-1}\) telle que:
\[
   y^{(n)}(x) = F(x, y(x), \ldots, y^{(n-1)}(x))
\]
On ne s'intéressera dans ce cours qu'à ce cas particulier pour simplifier la compréhension, sauf cas simples où l'on peut se ramener à une forme réduite. Par la suite on appelera \textbf{équation associée à \( F \)} d'ordre \( n \) une équation de la forme ci-dessus.
\subsection*{\subsecstyle{Solution d'une équation différentielle{:}}}
On dira que \((I, y)\) est une \textbf{solution} de l'équation différentiellé d'ordre \( n \) associée à \( F \) si et seulement si \( I \) est un intervalle, \(y \in \mathcal{C}^n(I, \R^p)\) et que:
\[ 
   \forall t \in I \; ; \; (t, y(t), \ldots, y^{(n-1)}(t)) \in \Omega \; \text{ et } \; y^{(n)}(t) = F(t, y(t), \ldots, y^{(n-1)}(t))
\]
On remarque alors que le domaine de définition d'une même fonction solution peut changer et on peut alors définir une notion de solution \textbf{maximale et globale} par:
\begin{itemize}
   \item Une solution est \textbf{maximale} si et seulement si elle ne peut pas être prolongée en une autre solution définie sur une intervalle.
   \item Une solution est \textbf{globale} si et seulement \( \Omega \) est de la forme \(I \times (\R^p)^{n-1}\) et que \( y \) est une solution définie sur \( I \).
\end{itemize}
\subsection*{\subsecstyle{Réduction de l'ordre{:}}}
On se donne une équations différentielle résolue d'ordre \(n\), alors on pose:
\[
   Y(x) := \begin{pmatrix}
      y(x)\\
      y'(x)\\
      \vdots\\
      y^{(n-1)}(x)
   \end{pmatrix} \quad \quad \quad \quad\quad
   \mathbb{F}(x, Y(x)) := \begin{pmatrix}
      y'(x)\\
      y''(x)\\
      \vdots\\
      F(x, U)
   \end{pmatrix}
\]
Alors on a directement que:
\[
   y^{(n)}(x) = F(x, y(x), \ldots, y^{(n-1)}(x)) \Longleftrightarrow Y'(x) = \mathbb{F}(x, Y(x))
\]
\begin{center}
   \textit{Fondamentalement, comprendre les EDO à l'ordre 1 c'est comprendre toutes les EDO.}
\end{center}
\subsection*{\subsecstyle{Problème de Cauchy{:}}}
Etant donné une équation d'ordre 1, et \( (t_0, y_0) \in \Omega \), on appele \textbf{problème de Cauchy} associée à l'équation \( (E) \) le problème suivant:
\[ 
   P: \begin{cases}
      y' = f(t, y)\\
      y(t_0) = y_0
   \end{cases} 
\]
Une question fondamentale de la théorie des equations differentielles ordinaires consiste à savoir sous quelles hypothèses sur \(  f \)tout problème de Cauchy associé à une équation admet une solution et les propriété de celles ci.
\subsection*{\subsecstyle{Raccordement de solutions{:}}}
Si on obtient deux solutions \( ( \ioo{a}{b}, y_1), (\ioo{b}{c}, y_2) \), alors on peut s'intéresser à raccorder ces deux solutions en une unique solution sur \( \ioo{a}{c} \). C'est en fait possible sous la condition suivante:
\[ 
   \lim_{x \rightarrow b} y_1(b) = \lim_{x \rightarrow b} y_2(b)
\]
\subsection*{\subsecstyle{Expression intégrale des solutions{:}}}
Considérons l'équation du premier ordre \((E)\) et le problème de Cauchy associé au couple \( (t_0, y_0) \), alors \( y \) est solution du problème de Cauchy ssi:
\[ 
   \forall t \in I \; ; \; y(t) = y_0 + \int_{t_0}^t f(s, y(s))ds 
\]
En réinterprétant cette équation et en définissant l'opérateur suivant:
\[ 
   \Phi : y \in \mathcal{C}^0(I, \R^p) \longmapsto (t \mapsto y_0 + \int_{t_0}^t f(s, y(s))ds) \in \mathcal{C}^0(I, \R^p)
\]
Alors une solution est exactement un \textbf{point fixe} de cet opérateur.

\chapter*{\chapterstyle{IX --- Théorie linéaire}}
\addcontentsline{toc}{section}{Théorie linéaire} 
On appele donc une équation différentielle linéaire une équation différentielle de la forme suivante:
\[ 
   y'(t) = A(t)y(t) + B(t) 
\]
Où \(A \in \mathcal{C}^0(I, \mathcal{M}_p( \R))\) et \(B \in \mathcal{C}^0(I, \mathcal{M}_{p, 1}( \R))\). Ce type d'équation trés spécifique permet d'utiliser la théorie de l'algèbre linéaire pour en trouver des solutions et/ou étudier leurs solutions.
\subsection*{\subsecstyle{Cauchy-Lipschitz linéaire{:}}}
\subsection*{\subsecstyle{Exponentielle matricielle{:}}}
\subsection*{\subsecstyle{Exponentielle matricielle{:}}}
\subsection*{\subsecstyle{Exponentielle matricielle{:}}}


\chapter*{\chapterstyle{IX --- Théorie non-linéaires}}
\addcontentsline{toc}{section}{Théorie non-linéaires} 

\chapter*{\chapterstyle{IX --- Introduction à l'étude qualitative}}
\addcontentsline{toc}{section}{Introduction à l'étude qualitative} 
