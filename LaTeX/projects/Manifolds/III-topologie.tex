\chapter{Elements de topologie}
On définit dans ce chapitre les outils principaux qui nous serviront à réaliser nos objectifs pour la suite, notamment pour pouvoir passer d'objets définis localement à des objets définis sur tout un espace, mais aussi le théorème d'invariance du domaine qui sera utile pour les constructions de base.
\section{Fonction bosses}
On appelle \textbf{fonction bosse} sur un ouvert \( U \subseteq \R^n \) toute fonction \( f \in \mathcal{C}^\infty_c(U, \R) \) et telle que:
\[ 
   f\big|_{\text{supp}(f)} = 1 
\]
Un résultat classique d'analyse affirme que de telles fonctions existent toujours. Celles ci nous permettront d'étendre des objets définis localement, en objets définis globalement.
\section{Partition de l'unité}
Dans toute la suite, on considérera un espace topologique \( M \) muni d'un recouvrement ouvert \((U_i)_{i \in I}\). Une \textbf{partition de l'unité} induite par ce recouvrement est une famille de fonctions \( (\rho_i)_{i \in I} \) de \( M \) dans \( \icc{0}{1} \) et telles que:  
\[ 
   \begin{cases}
      \text{supp}( \rho_i) \subseteq U_i\\
      \sum_{i \leq n} \rho_i = 1
   \end{cases}
\]
On imposera un caractère \textbf{localement fini} au recouvrement, c'est à dire que tout point \(x \in M\) appartient à un nombre fini d'ouverts. Ceci permet la finitude (ponctuellement) de la somme considérée.
\begin{figure*}[h]
   \centering
      \begin{tikzpicture}[xscale=1.5,yscale=1.25]       
         % 5 fonctions lisses phi_1 à phi_5
         \draw[domain=0:4,smooth,BrightBlue1,thick] plot (\x,{2*exp(-5*(\x-0)*(\x-0))/(exp(-5*(\x-0)*(\x-0))+exp(-5*(\x-1)*(\x-1))+exp(-5*(\x-2)*(\x-2))+exp(-5*(\x-3)*(\x-3))+exp(-5*(\x-4)*(\x-4)))});
         \node[BrightBlue1] at (0.175,1.5) {$\phi_1$};
         
         \draw[domain=0:4,smooth,BrightBlue1,thick] plot (\x,{2*exp(-5*(\x-1)*(\x-1))/(exp(-5*(\x-0)*(\x-0))+exp(-5*(\x-1)*(\x-1))+exp(-5*(\x-2)*(\x-2))+exp(-5*(\x-3)*(\x-3))+exp(-5*(\x-4)*(\x-4)))});
         \node[BrightBlue1] at (1,1.5) {$\phi_2$};
         
         \draw[domain=0:4,smooth,BrightBlue1,thick] plot (\x,{2*exp(-5*(\x-2)*(\x-2))/(exp(-5*(\x-0)*(\x-0))+exp(-5*(\x-1)*(\x-1))+exp(-5*(\x-2)*(\x-2))+exp(-5*(\x-3)*(\x-3))+exp(-5*(\x-4)*(\x-4)))});
         \node[BrightBlue1] at (2,1.5) {$\phi_3$};
         
         \draw[domain=0:4,smooth,BrightBlue1,thick] plot (\x,{2*exp(-5*(\x-3)*(\x-3))/(exp(-5*(\x-0)*(\x-0))+exp(-5*(\x-1)*(\x-1))+exp(-5*(\x-2)*(\x-2))+exp(-5*(\x-3)*(\x-3))+exp(-5*(\x-4)*(\x-4)))});
         \node[BrightBlue1] at (3,1.5) {$\phi_4$};
         
         \draw[domain=0:4,smooth,BrightBlue1,thick] plot (\x,{2*exp(-5*(\x-4)*(\x-4))/(exp(-5*(\x-0)*(\x-0))+exp(-5*(\x-1)*(\x-1))+exp(-5*(\x-2)*(\x-2))+exp(-5*(\x-3)*(\x-3))+exp(-5*(\x-4)*(\x-4)))});
         \node[BrightBlue1] at (3.825,1.5) {$\phi_5$};
         
         % Somme des fonctions (constante à 1)
         \draw[domain=0:4,smooth,black,dashed, line width = 1.1] plot (\x,{2*exp(-5*(\x-0)*(\x-0))/(exp(-5*(\x-0)*(\x-0))+exp(-5*(\x-1)*(\x-1))+exp(-5*(\x-2)*(\x-2))+exp(-5*(\x-3)*(\x-3))+exp(-5*(\x-4)*(\x-4))) + 2*exp(-5*(\x-1)*(\x-1))/(exp(-5*(\x-0)*(\x-0))+exp(-5*(\x-1)*(\x-1))+exp(-5*(\x-2)*(\x-2))+exp(-5*(\x-3)*(\x-3))+exp(-5*(\x-4)*(\x-4))) + 2*exp(-5*(\x-2)*(\x-2))/(exp(-5*(\x-0)*(\x-0))+exp(-5*(\x-1)*(\x-1))+exp(-5*(\x-2)*(\x-2))+exp(-5*(\x-3)*(\x-3))+exp(-5*(\x-4)*(\x-4))) + 2*exp(-5*(\x-3)*(\x-3))/(exp(-5*(\x-0)*(\x-0))+exp(-5*(\x-1)*(\x-1))+exp(-5*(\x-2)*(\x-2))+exp(-5*(\x-3)*(\x-3))+exp(-5*(\x-4)*(\x-4))) + 2*exp(-5*(\x-4)*(\x-4))/(exp(-5*(\x-0)*(\x-0))+exp(-5*(\x-1)*(\x-1))+exp(-5*(\x-2)*(\x-2))+exp(-5*(\x-3)*(\x-3))+exp(-5*(\x-4)*(\x-4)))});
         \node[] at (4.3,2) {$\sum \phi_i$};

         % Axe des x (intervalle [0, 1])
         \draw[->, thick] (0,0) -- (5,0);
         \node[] at (0, -0.25){$0$};
         \node[] at (4, -0.25){$1$};
                  
         % Axe des y
         \draw[->, thick] (0,0) -- (0,2.5);
         \node[] at (-0.25,2) {$1$};
   \end{tikzpicture}
   \caption{Partition de l'unité de l'intervalle \( \ioo{0}{1} \)}
\end{figure*}
\pagebreak
\section{Opérateur local}
      Soit \( X \) un espace topologique et \( E \) un espace vectoriel, on considère alors l'espace de fonctions \(\mathcal{F}(X, E)\) et on appelle \textbf{opérateur local} sur cet espace un opérateur linéaire \(D\) qui vérifie:
      \[ 
         \forall f, g \in \mathcal{F}(X, E) \; ; \;  \exists U \text{ ouvert } \; ; \; f|_U = g|_U \implies (Df)|_U = (Dg)|_U 
      \]
      Moralement, les opérateurs locaux sont ceux tels que si \( f, g \) sont indistinguables localement, alors \( Df, Dg \) le seront aussi.
      \begin{itemize}
         \item Les exemples types sont les opérateurs de dérivation, par rexemple \(T : f \longmapsto \frac{df}{dt}\)
         \item Les contre-exemples types sont les opérateurs intégraux, par exemple \(T : f \longmapsto \int_0^1 f(t)dt\)
      \end{itemize}

\section{Théorème de l'invariance du domaine}
   En outre, on aura besoin du résultat de topologie suivant, qui sera admis, appelé \textbf{théorème de l'invariance du domaine}:

   \begin{center}
      Si \( f : U \longrightarrow V \) est une \textbf{injection continue} (en particulier, si c'est un homéomorphisme), alors \( f(U) \) est un \textbf{ouvert} de \( \R^n \).
   \end{center}

