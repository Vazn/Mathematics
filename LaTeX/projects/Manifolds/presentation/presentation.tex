\documentclass{beamer}
\usepackage{fontspec}
\usepackage[french]{babel}
\usepackage[T1]{fontenc}
\usepackage{amsmath}
\usepackage{unicode-math}   % Pour définir les polices mathématiques
\usepackage{pgfplots}
\usepackage{bookmark}
\usepackage{hyperref}
\usepackage{esvect}
\usepackage{tcolorbox}
\usepackage{cmbright}

\usepackage{xcolor}
\usepackage{tikz, tkz-fct, tkz-euclide}
\usetikzlibrary{calc,patterns,angles,quotes}

\renewcommand{\comment}[1]{}

\pgfplotsset{compat = newest}
\usetikzlibrary{calc}
\usetikzlibrary{intersections}
\usetikzlibrary{shapes,arrows,arrows.meta,angles,quotes,patterns,patterns.meta}
\usetikzlibrary{decorations.markings,decorations.pathmorphing}


\definecolor{BrightBlue1}{RGB}{95, 150, 210}
\definecolor{BrightRed1}{RGB}{210, 95, 95}
\definecolor{BrightRed2}{RGB}{210, 115, 115}
\definecolor{DarkBlueX}{RGB}{43, 68, 92}
\definecolor{DarkBlue0}{RGB}{53, 78, 102}
\definecolor{DarkBlue1}{RGB}{83, 108, 132}
\definecolor{DarkBlue2}{RGB}{58, 94, 132}
\definecolor{DarkBlue3}{RGB}{90, 126, 162}
\definecolor{DarkGreen3}{RGB}{83, 132, 108}
\definecolor{DarkGreen2}{RGB}{58, 132, 94}
\definecolor{DarkGreen1}{RGB}{90, 162, 126}
\newcommand{\firasans}[1]{{\usefont{T1}{FiraSans}{m}{n}#1}}

\usetheme{metropolis}
\setbeamercolor{frametitle}{bg=DarkBlue1, fg=white}

\setbeamercolor{progress bar}{fg=DarkBlue1}
\makeatletter
\setlength{\metropolis@titleseparator@linewidth}{1.5pt}
\setlength{\metropolis@progressonsectionpage@linewidth}{1.5pt}
\setlength{\metropolis@progressinheadfoot@linewidth}{1.5pt}
\makeatother

\setmainfont{Fira Sans}[Weight=300, ItalicFont={Fira Sans Italic}]
\tikzset{every picture/.style={font=\rmfamily}}

% Title page details: 
\title{Modélisation}
\subtitle{Systêmes périodiques et pseudo-périodiques}
\author{Cavazzoni Christophe}
\begin{document}
    \begin{frame}
        \titlepage
    \end{frame}
    \begin{frame}{Table des matières}
        \setbeamertemplate{section in toc}[sections numbered]
        \tableofcontents
    \end{frame}
    
    \section{Introduction}
        \begin{frame}{Introduction}
            \begin{itemize}
                \item On appelle \textbf{système dynamique continu} un ensemble d'éléments qui intéragissent entre eux et dont l'évolution dans le temps est décrite par une loi continue.
                \item On appelle \textbf{systèmes périodiques} un système dynamique tel qu'il évolue de part et d'autres d'un point d'équilibre donné.
                \item On appelle \textbf{systèmes pseudo-périodiques} un système dynamique tel qu'il évolue de part et d'autres d'un point d'équilibre donné mais dont \textbf{l'amplitude décroît} avec le temps.
            \end{itemize}
        \end{frame}
        \begin{frame}{Introduction}
            Nous étudiront principalement 2 cas de sytèmes périodiques:
            \begin{itemize}
               \item Le cas du \textbf{système masse-ressort}.
               \item Le cas du \textbf{pendule simple non-linéaire}.
            \end{itemize}
        \end{frame}
        \begin{frame}{Introduction}
        Les systèmes que l'on veut modéliser sont des systèmes basées sur les lois de la physique, on utilisera:
            \begin{itemize}
            \item Le \textbf{principe fondamental de la dynamique} qui nous permettra de déterminer l'accélération de l'objet.
            \item Certains objets on des propriétés particulières qui demanderont d'autres concepts physiques (ressorts notamment).
            \end{itemize}
        \end{frame}

    \section{Systême masse-ressort}
        \subsection{Modélisation}
            \begin{frame}{Modélisation}
                Voici un schéma illustrant la situation \textbf{à l'équilibre}:
                \begin{figure}
                    \centering
                    \begin{tikzpicture}[>=stealth]
                        \path[pattern={Lines[angle=45,distance={8pt/sqrt(2)}]}] (-2,5) edge ++(4,0)
                        rectangle ++ (4,0.5);
                        \draw[decorate,decoration={coil,segment length=5pt,aspect=0.7,amplitude=4pt,
                            pre=lineto,pre length=3mm,post=lineto,post length=3mm},thick] (0,5) -- (0,2.5)
                            node[below,draw,minimum size=1cm,fill=DarkGreen1!50](m){$m$};

                        \draw[<->, DarkGreen3] (-0.5,4.96) -- node[left=1mm] {$l$} (-0.5, 2.54);

                    \draw[DarkGreen3] (m.east) edge[dashed] (m.east-|2,0);
                    \draw[DarkGreen3] (m.west) edge[dashed] (m.west-|-2,0);
                    \end{tikzpicture}
                    \caption{Schéma de la situation}
                \end{figure}
            \end{frame}
            \begin{frame}{Paramètres}
                On peut tout d'abord identifier les \textbf{paramétres du modèle}:
                \begin{itemize}
                    \item La masse de l'objet.
                    \item La longeur du ressort.
                    \item La force de gravité.
                \end{itemize}
            \end{frame}
            \begin{frame}{Paramètre spécifique}
                Aussi, on peut remarque que si quand la masse est à l'équilibre et si on note \(l_0\) la longueur du ressort à vide, alors la force de gravité est proportionelle à l'élongation du ressort, ie on a:
                \[
                mg = k(l - l_0)
                \]
                On appelle alors \(k\) la \textbf{constante de raideur} du ressort. Cette constante sera un autre paramêtre de notre modèle.
            \end{frame}
            \begin{frame}{Bilan des forces}
                On effectue le bilan des forces:
                \begin{figure}
                    \centering               
                    \begin{tikzpicture}[>=stealth]
                    \path[pattern={Lines[angle=45,distance={8pt/sqrt(2)}]}] (-2,5) edge ++(4,0)
                    rectangle ++ (4,1);
                    \draw[decorate,decoration={coil,segment length=7pt,aspect=0.7,amplitude=4pt,
                        pre=lineto,pre length=3mm,post=lineto,post length=3mm},thick] (0,5) -- (0,2.02)
                        node[below,draw,minimum size=1cm,fill=DarkGreen1!50](m){$m$};
            
                    \draw[dashed, DarkGreen3] (0.5,2) -- (1.5, 2);
                    \draw[dashed, DarkGreen3] (0.5,2) -- (-1.5, 2);
                    \draw[->, thick, DarkGreen3] (0,1) -- node[right]{$\vv{F}$} (0, 0.30);
                    \draw[->, thick, BrightRed2] (0,5.0) -- node[right]{$\vv{F}$} (0, 5.70);
                \end{tikzpicture}
                \caption{Bilan des forces}
                \end{figure}    
            \end{frame}     
            \begin{frame}{Mise en équation}
                la force qu'exercerce le ressort dans la direction de l'équilibre est donc linéairement proportionelle à la distance avec l'équilibre, et donc d'aprés le \textbf{principe fondamental de la dynamique}, ie on a:    
                \[
                    \vv{a}(t) = \frac{1}{m}\sum \vv{F}(t)
                \]
                Et donc l'équation du mouvement est donnée par:
                \[
                    \boxed{y''(t) = -\frac{k}{m}y(t)}
                \]
            \end{frame}      
        \subsection{Résolution}
            \begin{frame}{Résolution}
                L'équation précédente et facilement résoluble, en effet c'est une equation linéaire et on peut remarquer facilement que les fonctions solutions sont de la forme:
                \[
                    y(t) = c_1\text{\fontfamily{cmr}\selectfont cos}(\omega t) + c_2\text{\fontfamily{cmr}\selectfont sin}(\omega t) \; ; \; \omega = \sqrt{\frac{k}{m}}
                \]
            \end{frame}
        \subsection{Etude des solutions}
            \begin{frame}{Portraits de phase non-amorti}
                \vspace{7.5pt}
                \begin{figure}
                    \centering
                    \begin{tikzpicture}
                        \tikzset{%
                                my arrow/.style={
                                postaction={decorate,decoration={
                                markings,
                                mark=between positions 0.25 and 1 step 0.15 with {\arrow[line width=1.5pt]{stealth}}}}
                            }
                        }
                        \begin{axis}[
                        xmin = -2, xmax = 2,
                        ymin = -2, ymax = 2,
                        zmin = 0, zmax = 1,
                        axis equal image,
                        xtick distance = 1,
                        ytick distance = 1,
                        view = {0}{90},
                        scale = 1,
                        xlabel = {$y$},
                        ylabel = {$y'$},
                        colormap/viridis,
                        ]
                        \addplot3[
                            point meta = {sqrt(x^2+y^2)},
                            quiver = {
                                u = {y/sqrt(x^2+y^2)},
                                v = {-x/sqrt(x^2+y^2)},
                                scale arrows = 0.15,
                            },
                            quiver/colored = {mapped color!50},
                            -stealth,
                            domain = -2:2,
                            domain y = -2:2,
                        ] {0};
                        \addplot [my arrow, samples=100, domain=0:2*pi, line width = 1, postaction={decorate},] ({0.5*cos(deg(x))}, {-0.5*sin(deg(x))});      
                        \addplot [my arrow, samples=100, domain=0:2*pi, line width = 1, postaction={decorate},] ( {cos(deg(x))}, {-sin(deg(x))} );   
                        \addplot [my arrow, samples=100, domain=0:2*pi, line width = 1, postaction={decorate},] ( {1.5*cos(deg(x))}, {-1.5*sin(deg(x))} );      
                        \end{axis}
                    \end{tikzpicture}
               \end{figure}
            \end{frame}
            \begin{frame}{Ajout d'un terme d'amortissement}
                On essaie maintenant de rendre le modèle plus réaliste en ajoutant la contribution des frottements de l'air. Pour \(\lambda > 0\) un coefficient de frottements, on ajoute la contrainte \textbf{linéaire} suivante:
                \[
                   y''(t) = -\frac{k}{m}y(t) - \lambda y'(t)
                \]
            \end{frame}
            \begin{frame}{Portrait de phase amorti}
                \vspace{7.5pt}
                \begin{figure}
                    \centering
                    \begin{tikzpicture}
                        \tikzset{%
                           my arrow/.style={
                           postaction={decorate,decoration={
                           markings,
                           mark=between positions 0.001 and 0.15 step 0.03 with {\arrow[line width=1.5pt]{stealth}}}}
                           }
                        }
                        \begin{axis}[
                           xmin = -2, xmax = 2,
                           ymin = -2, ymax = 2,
                           zmin = 0, zmax = 1,
                           axis equal image,
                           xtick distance = 1,
                           ytick distance = 1,
                           view = {0}{90},
                           scale = 1,
                           xlabel = {$y$},
                           ylabel = {$y'$},
                           colormap/viridis,
                        ]
                        \def\length{sqrt(y^2 + (-x - 0.5*y)^2)}
                           \addplot3[
                              point meta = {\length},
                              quiver = {
                                 u = {y/\length},
                                 v = {(-x - 0.5*y)/\length},
                                 scale arrows = 0.15,
                              },
                              quiver/colored = {mapped color!50},
                              -stealth,
                              domain = -2:2,
                              domain y = -2:2,
                              axis equal image
                           ] {0};
                           \addplot[my arrow, mark = none, color = black, postaction={decorate},] table {../data/springMass/springMass1.dat}; 
                           \addplot[my arrow, mark = none, color = black, postaction={decorate},] table {../data/springMass/springMass2.dat};
                           \addplot[my arrow,mark = none, color = black, postaction={decorate},] table {../data/springMass/springMass3.dat};
                        \end{axis}
                    \end{tikzpicture}
                \end{figure}
            \end{frame}
    
    \section{Pendule simple}
        \subsection{Modélisation}
            \begin{frame}{Modélisation}
                Voici un schéma qui illustre la situation \textbf{à l'équilibre}:
                \begin{figure}
                    \centering
                    \begin{tikzpicture}[>=stealth, scale=1.2]
                        \draw[thick] (0,5) -- (0,2.5) node[circle, draw, minimum size=0.7cm, fill=DarkGreen1!50]{$m$};
                        \draw[DarkGreen3] (0, 1.5) edge[dashed] (0, 2.1) node[below] {Equilibre};
                        \draw[<->, DarkGreen3] (-0.3,4.96) -- node[left=1mm] {$l$} (-0.3, 2.84);

                        \filldraw (0, 5) circle (2.5pt);
                    \end{tikzpicture}
                    \caption{Schéma de la situation}
                \end{figure}
            \end{frame}
            \begin{frame}{Paramètres}
                On peut tout d'abord identifier les \textbf{paramétres du modèle}:
                \begin{itemize}
                \item La masse de l'objet.
                \item La longeur du ressort.
                \item La constante de gravité.
                \end{itemize}        
            \end{frame}
            \begin{frame}{Bilan des forces}
                On effectue le bilan des forces:
                \begin{figure}
                    \begin{tikzpicture}[>=stealth, scale=1.2]
                        \coordinate (pivot) at (0, 5);
                        \coordinate (equilibrium) at (0, 1.5);
                        \coordinate (pendulum) at (1.76, 3.23);
                        \pic[<-, draw, angle eccentricity=1.6, angle radius = 0.7cm, thick] {angle = equilibrium--pivot--pendulum};
                        \pic[<-, draw, angle eccentricity=0.85, angle radius = 2.9cm, color = DarkGreen1, thick] {angle = equilibrium--pivot--pendulum};
                        \node at (-0.35, 4.4) {$\theta(t)$};
                        \node[color = DarkGreen1] at (-0.35, 2.6) {$l\theta(t)$};
            
                        \filldraw (pivot) circle (2.5pt);
                        \draw[thick] (pivot) -- (pendulum) node[circle, draw, minimum size=0.7cm, fill=DarkGreen1!50]{$m$};
                        \draw[DarkGreen3] (equilibrium) edge[dashed] (pivot) node[below] {Equilibre};
            
                        \draw[->, thick, DarkGreen3] (1.76, 2.93) -- node[right]{$\vv{F}_G$} (1.76, 2);
                        \draw[->, thick, BrightRed1] (1.545, 3.445) -- (1, 3.999);
                        \node[BrightRed1] at (1.55, 3.9) {$\vv{F}_T$};
                    \end{tikzpicture}
                    \caption{Bilan des forces}
                \end{figure}
            \end{frame}
            \begin{frame}{Mise en équation}
                D'aprés le \textbf{principe fondamental de la dynamique}, on a l'accélaration angulaire donnée par:
                \[
                    \vv{a}(t) = \frac{1}{m}\sum\vv{F}(t)
                \]
                Et donc finalement en utilisant des relations trigonométriques, la composante d'acceleration dans le sens de la course du pendule est donnée par:
                \[
                    \boxed{\theta''(t) = - \frac{g}{l}\text{\fontfamily{cmr}\selectfont sin}(\theta)}
                \]
            \end{frame}
        \subsection{Résolution}
            \begin{frame}{Résolution impossible}
                Malheureusement, la résolution analytique de ce problême est \textbf{trés complexe} et est impossible pour nous. On se contente donc d'une \textbf{étude qualitative} et de la \textbf{résolution numérique} du problème.
            \end{frame}
        \subsection{Etude des solutions}
            \begin{frame}{Portraits de phase non-amorti}
                \begin{tikzpicture}
                       \tikzset{%
                          my arrow/.style={
                          postaction={decorate,decoration={
                          markings,
                          mark=between positions 0.25 and 1 step 0.15 with {\arrow[line width=1.5pt]{stealth}}}}
                          }
                       }
                       \def\xdom{3}
                       \def\ydom{1.5}
                       \begin{axis}[
                          xmin = -\xdom, xmax = \xdom + 2*3.1415,
                          ymin = -\ydom, ymax = \ydom,
                          zmin = 0, zmax = 1,
                          xtick distance = 1,
                          ytick distance = 1,
                          view = {0}{90},
                          title = {\bf Portrait de phase: $F = [y,-sin(x)]$},
                          height=7cm,
                          xlabel = {$y$},
                          ylabel = {$y'$},
                          colormap/viridis,
                          colorbar,
                          colorbar style = {
                             ylabel = {Norme}
                          }
                       ]
                       
                       \def\length{sqrt(y^2 + 0.125 * (sin(deg(x)))^2)}
                          \addplot3[
                             point meta = {\length},
                             quiver = {
                                u = {y/\length},
                                v = {- 0.125 *(sin(deg(x)))/\length},
                                scale arrows = 0.25,
                             },
                             quiver/colored = {mapped color!50},
                             -stealth,
                             domain = -\xdom : \xdom + 2*3.1415 ,
                             domain y = -\ydom : \ydom,
                          ] {0};
                          \addplot[mark = none, color = black!75, line width = 1.2,] table {../data/simplePendulum/undamped/pendulum1.dat}; 
                          \addplot[mark = none, color = black!75, line width = 1.2,] table {../data/simplePendulum/undamped/pendulum2.dat}; 
                          \addplot[mark = none, color = black!75, line width = 1.2,] table {../data/simplePendulum/undamped/pendulum3.dat}; 
                          \addplot[mark = none, color = black!75, line width = 1.2,] table {../data/simplePendulum/undamped/pendulum4.dat};
              
                          \addplot[mark = none, color = black!75, line width = 1.2,] table {../data/simplePendulum/undamped/pendulum5.dat}; 
                          \addplot[mark = none, color = black!75, line width = 1.2,] table {../data/simplePendulum/undamped/pendulum6.dat}; 
                          \addplot[mark = none, color = black!75, line width = 1.2,] table {../data/simplePendulum/undamped/pendulum7.dat}; 
                          \addplot[mark = none, color = black!75, line width = 1.2,] table {../data/simplePendulum/undamped/pendulum8.dat}; 
                       \end{axis}
                \end{tikzpicture}
            \end{frame}
            \begin{frame}{Ajout d'un terme d'amortissement}
                On essaie maintenant de rendre le modèle plus réaliste en ajoutant la contribution des frottements de l'air. Pour \(\lambda > 0\) un coefficient de frottements, on ajoute la contrainte \textbf{linéaire} suivante:
                \[
                    \theta''(t) = - \frac{g}{l}\text{\fontfamily{cmr}\selectfont sin}(\theta) - \lambda \theta'(t)
                 \]
            \end{frame}
            \begin{frame}{Portraits de phase amorti}
                    \begin{tikzpicture}
                       \tikzset{%
                          my arrow/.style={
                          postaction={decorate,decoration={
                          markings,
                          mark=between positions 0.25 and 1 step 0.15 with {\arrow[line width=1.5pt]{stealth}}}}
                          }
                       }
                       \def\xdom{3}
                       \def\ydom{1.5}
                       \begin{axis}[
                          xmin = -\xdom, xmax = \xdom + 2*3.1415,
                          ymin = -\ydom, ymax = \ydom,
                          zmin = 0, zmax = 1,
                          xtick distance = 1,
                          ytick distance = 1,
                          view = {0}{90},
                          title = {\bf Portrait de phase: $F = [y,-sin(x)-y]$},
                          height=7cm,
                          xlabel = {$y$},
                          ylabel = {$y'$},
                          colormap/viridis,
                          colorbar,
                          colorbar style = {
                             ylabel = {Norme}
                          }
                       ]
                       
                       \def\length{sqrt(y^2 + 0.125 * (sin(deg(x)) + 0.2*y)^2)}
                          \addplot3[
                             point meta = {\length},
                             quiver = {
                                u = {y/\length},
                                v = {(- 0.125 *(sin(deg(x))) - 0.2*y)/\length},
                                scale arrows = 0.25,
                             },
                             quiver/colored = {mapped color!50},
                             -stealth,
                             domain = -\xdom : \xdom + 2*3.1415 ,
                             domain y = -\ydom : \ydom,
                          ] {0};
                          \addplot[mark = none, color = black!75, line width=1.05] table {../data/simplePendulum/damped/pendulum1.dat}; 
                          \addplot[mark = none, color = black!75, line width=1.05] table {../data/simplePendulum/damped/pendulum2.dat}; 
                          \addplot[mark = none, color = black!75, line width=1.05] table {../data/simplePendulum/damped/pendulum3.dat}; 
                          \addplot[mark = none, color = black!75, line width=1.05] table {../data/simplePendulum/damped/pendulum4.dat};
              
                          \addplot[mark = none, color = black!75, line width=1.05] table {../data/simplePendulum/damped/pendulum5.dat}; 
                          \addplot[mark = none, color = black!75, line width=1.05] table {../data/simplePendulum/damped/pendulum6.dat}; 
                          \addplot[mark = none, color = black!75, line width=1.05] table {../data/simplePendulum/damped/pendulum7.dat}; 
                          \addplot[mark = none, color = black!75, line width=1.05] table {../data/simplePendulum/damped/pendulum8.dat}; 
                       \end{axis}
                    \end{tikzpicture}
            \end{frame}
\end{document}