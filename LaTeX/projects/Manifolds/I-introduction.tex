\chapter{Introduction}
   Dans ce projet d'étude, on cherche à généraliser le calcul différentiel usuel dans \( \R^n \) sur des objets plus généraux, espaces courbes, qui ne seront pas des espaces vectoriels simples. En particulier, on cherche à définir le concept de \textbf{varitété différentielle}, qui est la formalisation mathématique de ce types d'espaces.\< 
   
   Le projet suivra la progression suivante:
   \begin{itemize}
      \item Tout d'abord nous définirons le concept de \textbf{variété topologique} puis \textbf{différentielle} ainsi leurs propriétés, une partie spécifique sera dédiée à la construction d'espace courbes qui possèdent un "bord".
      \item Nous préseterons ensuite quelques exemples de tels objets.
      \item Par la suite nous chercherons à construire une \textbf{structure différentielle} sur ces objets, ce qui reviendra à définir la notion \textbf{d'espace tangent} en un point de l'objet et à étudier ses propriétés.
      \item Ensuite, nous pourrons définir le concept de \textbf{forme différentielle} sur un variété qui sera l'objet fondamental qui nous servivra à généraliser la théorie de l'intégration.
      \item Enfin, aprés avoir définit l'intégrale de tels objets, on pourra alors montrer le \textbf{théorème de Stokes}, généralisation du théorème fondamental de l'analyse à toute variété à bord orientée et compacte.
   \end{itemize}