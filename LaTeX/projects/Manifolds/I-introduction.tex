\setcounter{chapter}{-1}
\chapter{Introduction}
   Dans ce projet d'étude, on cherche à généraliser le calcul différentiel usuel dans \( \R^n \) sur des objets plus généraux, espaces courbes, qui ne seront pas des espaces vectoriels simples. En particulier, on cherche à définir le concept de \textbf{varitété différentielle}, qui est la formalisation mathématique de ce types d'espaces.\< 
   
   Le projet suivra la progression suivante:
   \begin{itemize}
      \item \textbf{Le chapitre 1 - Algèbre tensorielle:} Il pose les bases d'algèbre linéaire qui seront nécessaires pour construire la théorie dans l'espace connu \( \R^n \).
      \item \textbf{Le chapitre 2 - Topologie:} Il pose le concept de partition de l'unité qui sera fondamental pour la suite
      \item \textbf{Le chapitre 3 - Variétés:} Il définit le concept de \textbf{variété topologique} puis \textbf{différentielle}, modèles d'espaces courbes généraux, qui seront les objets d'étude de cet exposé. 
      \item \textbf{Le chapitre 4 - Variétés à bord:} Il s'attarrde sur la construction de tels espaces qui possèdent un "bord".
      \item \textbf{Le chapitre 5 - Exemples:} Une partie succinte pour présenter des variétés différentielles usuelles.
      \item \textbf{Les chapitres 6/7 - Espaces tangents:} Il pose le concept de vecteur tangent dans \( \R^n \), puis dans une variété abstraite.
      \item \textbf{Le chapitre 8 - Espaces cotangents/exterieurs:} Il pose le concept de vecteur cotangent dans une variété abstraite ainsi que celui de tenseur d'ordre quelconque.
      \item \textbf{Le chapitre 9 - Formes différentielles:} Il pose le concept de formes différentielles sur un variété ainsi que leurs propriétés et des opérations fondamentales sur celles ci.
      \item \textbf{Le chapitre 10 - Orientation d'une variété à bord:} Il s'attarde sur le concept d'orientation d'une variété et comment celui-ci induit une orientation sur le bord de celle-ci.
      \item \textbf{Le chapitre 11 - Intégrale locale d’une forme:} Il  définit le concept d'intégrale d'une forme différentielle localement sur la variété.
      \item \textbf{Le chapitre 12 - Intégrale d’une forme:} Il définit le concept d'intégrale d'une forme différentielle sur la variété.
      \item \textbf{Le chapitre 13 - Théorème de Stokes:} Théorème-objectif de ce rapport, qui généralise le théorème fondamental de l'analyse dans un cadre trés général en toutes dimensions.
      \item \textbf{Le chapitre 14 - Applications:} Quelques applications de tout les outils théoriques développés dans divers domaines des mathématiques si le temps le permet.
   \end{itemize}