\chapter{Introduction}
   Dans ce projet d'étude, on cherche à généraliser le calcul différentiel usuel dans \( \R^n \) sur des objets plus généraux, espaces courbes, qui ne seront pas des espaces vectoriels simples. En particulier, on cherche à définir le concept de \textbf{varitété différentielle}, qui est la formalisation mathématique de ce types d'espaces.\< 
   
   Le projet suivra la progression suivante:
   \begin{itemize}
      \item Tout d'abord nous exposerons un chapitre \textbf{d'algèbre tensorielle} dans l'espace connu \( \R^n \), ceci aura pour but de poser les bases d'algèbre linéaire qui seront nécessaires pour construire la théorie.
      \item Ensuite nous définirons le concept fondamental de \textbf{variété topologique} puis \textbf{différentielle}, modèles d'espaces courbes généraux, une partie spécifique sera dédiée à la construction de tels espaces qui possèdent un "bord".
      \item Une partie succinte pour présenter des variétés différentielles usuelles.
      \item Nous chercherons ensuite à construire des objets de calcul différentiel sur ces espaces, ie des \textbf{champs de vecteurs, des fonctions différentiables, des vecteurs tangents}. Ceci reviendra à définir la notion \textbf{de fibré tangent et cotangent} et étudier leurs propriétés.
      \item Ensuite, nous pourrons étendre les notions d'algèbre tensorielle aux variétés abstraites, en définissant le concept de \textbf{forme différentielle} sur un variété qui sera l'objet fondamental qui nous servivra à généraliser la théorie de l'intégration.
      \item Enfin, aprés avoir définit l'intégrale de tels objets, on pourra alors montrer le \textbf{théorème de Stokes}, généralisation du théorème fondamental de l'analyse à toute variété à bord orientée et compacte.
      \item Finalement, le dernier chapitre sera uniquement consacré aux différentes applications de la théorie, idéalement à la fois dans des cas concrets et théoriques.
   \end{itemize}