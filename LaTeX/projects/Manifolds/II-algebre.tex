\chapter{Elements d'algèbre tensorielle}
   On se donne un \(\K\)-espace vectoriel \(E\) de dimension \( n \), alors on appele \textbf{tenseur} d'ordre \((p, q)\) une application multilinéaire de la forme suivante:
   \begin{align*}
      T : \underbrace{E^* \times \ldots \times E^*}_\text{p} \times \underbrace{E \times \ldots \times E}_\text{q} \longrightarrow \K
   \end{align*}
   Dans le contexte de ce projet, on s'intéressera surtout au cas où \( p = 0 \). On dira alors que \( T \) est un tenseur \textbf{covariant}. Si on a aussi \( q = 0 \), alors les tenseurs ainsi défini sont des fonctions sur l'espace nul, qui s'identifient aux scalaires du corps de base.
   \section{Structure de l'ensemble des tenseurs covariants}
   On note alors \(\mathscr{T}^p(E)\) l'ensemble des \(p\)-tenseurs covariants, alors l'addition de deux formes et la multiplication par un scalaire étant bien définie, on peut montrer la propriété suivante:
   \begin{center}
      L'espace \( \mathscr{T}^p(E) \) a une structure de \( \K \)-espace vectoriel.
   \end{center}
   \section{Produit tensoriel de deux tenseurs covariants}
   On peut alors définir un produit sur des tels objets appelé \textbf{produit tensoriel} défini par:
   \begin{align*}
      \otimes : \mathscr{T}^p(E) \times \mathscr{T}^q(E) &\longrightarrow \mathscr{T}^{p+q}(E)\\
      (\alpha, \beta) &\longmapsto \alpha \otimes \beta
   \end{align*}
   Avec le tenseur \(\alpha \otimes \beta\) défini par:
   \[
      (\alpha \otimes \beta)(x_1, \ldots, x_p, y_1, \ldots, y_q) = \alpha(x_1, \ldots, x_p)\beta(y_1, \ldots, y_q)
   \]
   \section{Base et dimension}
      On peut alors se demander si on peut trouver une base de cet espace, et en effet si on note \((e_i)_{i \leq n}\) une base de \(E\), alors on peut montrer qu'une base de \(\mathcal{T}^p(E)\) est donnée par:
      \[
         \Bigl\{ e^{i_1} \otimes \ldots \otimes e^{i_p} \; ; \; 1 \leq i_1, \ldots, i_p \leq n  \Bigl\}
      \]
   \section{Permutations des indices}
   Alors pour tout permuation \(\sigma \in S_p\), on définit l'action d'une permutation sur un tenseur \( T \in \mathscr{T}^p(E) \) par:
   \[ 
      T_\sigma(x_1, \ldots, x_n) = T(x_{\sigma(1)}, \ldots, x_{\sigma(n)})
   \]
   \pagebreak

   \section{Tenseurs antisymétriques}
   On appelle \textbf{tenseur antisymétrique} tout \( p \)-tenseur \( T \) tel que:
   \[ 
      \forall i, j \in \inticc{1}{p} \; ; \; T(\ldots, x_j, \ldots, x_i, \ldots) = -T(\ldots, x_i, \ldots, x_j, \ldots)
   \]
   On peut alors montrer une propriété naturelle de tels tenseurs:
   \[ 
      \forall \sigma \in S_p \; ; \; T_\sigma(x_1, \ldots, x_n) = \epsilon(\sigma)T(x_1, \ldots, x_n) 
   \]
   \section{Antisymétrisation}
   On se donne un tenseur \(T\) qui soit \(p\)-covariant, alors on chercher à construire un tenseur \(p\)-covariant \textbf{antisymétrique} à partir de \(T\), et on peut alors montrer que le tenseur suivant convient:
   \[
      \text{Asym}(T)(x_1, \ldots, x_p) = \frac{1}{k!}\sum_{\sigma \in S_p}\epsilon(\sigma)T(x_{\sigma(1)}, \ldots, x_{\sigma(p)})
   \]
   En d'autres termes que \( T \) est bien antisymétrique. 
   \section{Produit extérieur}
   On peut alors définir un produit antisymétrique appelé surtout \textbf{produit extérieur} de deux tenseurs d'ordre respectifs \( p, q \) par:
   \[
      (T \wedge T') = \frac{(p+q)!}{p!q!}\text{Asym}(T \otimes T')
   \]
   La présence du coefficient est motivée par des considérations techniques\footnote[1]{Ce coefficient permet alors d'exprimer une base simple des tenseurs antisymétriques.}. En d'autres termes:
   \[
      (T \wedge T')(x_1, \ldots, x_p, x_{p+1}, \ldots, x_{p+q}) =  \frac{1}{p!q!}\sum_{\sigma \in S_{p+q}}\epsilon(\sigma) T(x_{\sigma(1)}, \ldots, x_{\sigma(p)})T'(x_{\sigma(p+1)}, \ldots, x_{\sigma(p+q)})
   \]
   \section{Puissance extérieure}
   On appelle alors \textbf{p-ième puissance extérieure} l'ensemble de toutes les formes \(p\)-linéaires alternées qu'on note \(\Lambda^p E^*\). Une base est alors donnée par l'ensemble:
   \[
      \Bigl\{ e^{i_1} \wedge \ldots \wedge e^{i_p} \; ; \; 1 \leq i_1 < \ldots < i_p \leq n  \Bigl\}
   \]
   Si \( T \in \Lambda^p E^*\) alors on le notera (selon que l'on explicite les indices ou qu'on utilise la notation multi-indices) sous une des deux formes différentes suivantes dans cette base:
   \[ 
      T = \sum_{1 \leq i_1 < \ldots < i_p \leq n} T_{i_1, \ldots, i_p} e^{i_1} \wedge \ldots \wedge e^{i_p} = \sum_{I} T_{I} e^I 
   \]  
   \section{Propriétés algébriques du produit extérieur}
   On peut alors montrer que le produit extérieur est \textbf{bilinéaire et alterné}, en particulier, on peut montrer une expression explicite du produit extérieur de deux tenseurs \( T, T' \) d'ordre différents dans une base fixée:
   \[ 
      \sum_{I} T_{I} e^I \wedge \sum_{J} T'_{J} e^J = \sum_{I, J} T_{I}T'_{J} e^I \wedge e^J
   \]
   \pagebreak
   \section{Déterminant}
   Si \( (e^i)_{i \leq n} \) est une base duale d'une base \( \mathcal{B} \) de \( E \), on a:
   \[ 
      \forall x_1, \ldots, x_n \in E \; ; \; e^1 \wedge \ldots \wedge e^n(x_1, \ldots, x_n) = \text{det}_{\mathcal{B}}(x_1, \ldots, x_n)
   \]
   En particulier on remarque alors que le déterminant canonique dans \( \R^n \) n'est que l'application de la forme multinéaire alternée \( c^1 \wedge \ldots \wedge c^n \) de \( \Lambda^n(\R^n) \) où l'on a choisi l'ordre usuel (d'orientation directe) des vecteurs de la base canonique.
   \section{Cas dégénéré des 0-tenseurs}
   Dans le cas de \(\mathscr{T}^0(E)\) toutes les définitions restent cohérentes:
   \begin{itemize}
      \item Le produit tensoriel se réduit en la multiplication sur \( \K \).
      \item Le produit extérieur dévient alors une somme sur l'unique symétrie de \( S_0 \), qui n'agit pas sur \( T \), ce qui se réduit aussi en la multiplication sur \( \K \).
      \item La condition d'antisymétrie sur \( \mathscr{T}^0(E) \) étant vide, on a aussi \(\Lambda^0 E^* = \R\) de base le produit extérieur vide, par convention égal à 1.
   \end{itemize}
