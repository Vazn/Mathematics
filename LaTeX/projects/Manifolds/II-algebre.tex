\chapter{Elements d'algèbre tensorielle}
   Dans ce chapitre, nous allons nous intéresser à l'object d'étude du domaine appelé \textbf{algèbre multilinéaire}, qui sont les \textbf{formes multilinéaires}, en particulier, on se donne un \(\R\)-espace vectoriel \(E\) de dimension \( n \), alors on appele \textbf{tenseur} d'ordre \((p, q)\) une application de la forme suivante:
   \begin{align*}
      T : \underbrace{E^* \times \ldots \times E^*}_\text{p} \times \underbrace{E \times \ldots \times E}_\text{q} \longrightarrow \K
   \end{align*}
   Dans le contexte de ce projet, on s'intéresse principalement à la construction des fromes différentielles, et donc on s'intéressera surtout au cas où \( p = 0 \). On dira alors que \( T \) est un tenseur \textbf{covariant}.
   \section{Structure de l'espace des tenseurs covariants}
   On note alors \(\mathscr{T}^p(E)\) l'ensemble des \(p\)-tenseurs covariants, alors l'addition de deux formes et la multiplication par un scalaire étant bien définie, on peut montrer la propriété suivante:
   \begin{center}
      L'espace \( \mathscr{T}^p(E) \) a une structure de \( \K \)-espace vectoriel.
   \end{center}
   \section{Produit tensoriel de deux tenseurs covariants}
   On peut alors définir un produit sur des tels objets appelé \textbf{produit tensoriel} défini par:
   \begin{align*}
      \otimes : \mathscr{T}^p(E) \times \mathscr{T}^q(E) &\longrightarrow \mathscr{T}^{p+q}(E)\\
      (\alpha, \beta) &\longmapsto \alpha \otimes \beta
   \end{align*}
   Avec le tenseur \(\alpha \otimes \beta\) défini par:
   \[
      (\alpha \otimes \beta)(x_1, \ldots, x_p, y_1, \ldots, y_q) = \alpha(x_1, \ldots, x_p)\beta(y_1, \ldots, y_q)
   \]
   \section{Base et dimension}
      On peut alors se demander si on peut trouver une base de cet espace, et en effet si on note \((e_i)_{i \leq n}\) une base de \(E\), alors on peut montrer que l'on a:
      \begin{flalign*}
         T(x_1, \ldots, x_p) &= T\left( \sum_{i_1 \leq n} x_{1, i_1}e_{i_1}, \ldots, \sum_{i_p \leq n} x_{p, i_p}e_{i_p} \right)\\
         &= \sum_{i_1, \ldots, i_p \leq n} x_{1, i_1} \ldots x_{p, i_p} T(e_{i_1}, \ldots, e_{i_p})
      \end{flalign*}
      Mais on remarque alors que le produit \( x_{1, i_1} \ldots x_{p, i_p} \) consiste alors en l'évaluation de \(e^{i_1} \otimes \ldots \otimes e^ {i_p} \) en \( (x_1, \ldots, x_p) \) et donc on obtient:
      \begin{flalign*}
         T &= \sum_{i_1, \ldots, i_p \leq n} T(e_{i_1}, \ldots, e_{i_p}) e^{i_1} \otimes \ldots \otimes e^{i_p}
      \end{flalign*}
      En d'autres termes tout \( p \)-tenseur \( T \) est engendré par la famille de \( n^p \) vecteurs \( (e^{i_1} \otimes \ldots \otimes e^{i_p})_{i_1, \ldots, i_p \leq n} \). On peut alors montrer qu'elle est libre et donc que c'est une base de \(\mathcal{T}^p(E)\).
  
   \pagebreak
   \section{Permutations des indices}
   Alors pour tout permuation \(\sigma \in S_p\), on définit l'action d'une permutation sur un tenseur \( T \in \mathscr{T}^p(E) \) par:
   \[ 
      T_\sigma(x_1, \ldots, x_n) = T(x_{\sigma(1)}, \ldots, x_{\sigma(n)})
   \]
   \section{Tenseurs antisymétriques}
   On appelle \textbf{tenseur antisymétrique} tout \( p \)-tenseur \( T \) tel que:
   \[ 
      \forall i, j \in \inticc{1}{p} \; ; \; T(\ldots, x_j, \ldots, x_i, \ldots) = -T(\ldots, x_i, \ldots, x_j, \ldots)
   \]
   On peut alors montrer une propriété naturelle de tels tenseurs:
   \[ 
      \forall \sigma \in S_p \; ; \; T_\sigma(x_1, \ldots, x_n) = \epsilon(\sigma)T(x_1, \ldots, x_n) 
   \]
   \section{Antisymétrisation}
   On se donne un tenseur \(T\) qui soit \(p\)-covariant, alors on chercher à construire un tenseur \(p\)-covariant \textbf{antisymétrique} à partir de \(T\), et on peut alors montrer que le tenseur suivant convient:
   \[
      \text{Asym}(T)(x_1, \ldots, x_p) = \frac{1}{k!}\sum_{\sigma \in S_p}\epsilon(\sigma)T(x_{\sigma(1)}, \ldots, x_{\sigma(p)})
   \]
   En d'autres termes que \( T \) est bien antisymétrique. 
   \section{Produit extérieur}
   On peut alors définir un produit antisymétrique appelé surtout \textbf{produit extérieur} de deux tenseurs d'ordre respectifs \( p, q \) par:
   \[
      (T \wedge T') = \frac{(p+q)!}{p!q!}\text{Asym}(T \otimes T')
   \]
   La présence du coefficient est motivée par des considérations techniques\footnote[1]{Ce coefficient permet alors d'exprimer une base simple des tenseurs antisymétriques.}. En d'autres termes:
   \[
      (T \wedge T')(x_1, \ldots, x_p, x_{p+1}, \ldots, x_{p+q}) =  \frac{1}{p!q!}\sum_{\sigma \in S_{p+q}}\epsilon(\sigma) T(x_{\sigma(1)}, \ldots, x_{\sigma(p)})T'(x_{\sigma(p+1)}, \ldots, x_{\sigma(p+q)})
   \]
   C'est ce produit extérieur qui nous sera surtout utile pour définir les formes différentielles.
   \section{Algèbre extérieure}
   On appelle alors \textbf{p-ième puissance extérieure} l'ensemble de toutes les formes \(p\)-linéaires alternées qu'on note \(\Lambda^p E^*\). Une base est alors donnée par l'ensemble:
   \[
      \Bigl\{ e^{i_1} \wedge \ldots \wedge e^{i_p} \; ; \; 1 \leq i_1 < \ldots < i_p \leq n  \Bigl\}
   \]
   Dans le cas où \( E = \R^n \), on note alors \( (dx_1, \ldots, dx_n) \) la base duale de la base canonique et on exprime généralement une forme \( p \)-linéaire alternée dans celle ci.\<

   \uline{Exemple:} Si \( E = \R^3 \), on note la base duale \((dx, dy, dz)\), alors on a que:
   \[
      \Lambda^2E^* = \text{Vect}(dx \wedge dy, dx \wedge dz, dy \wedge dz)
   \]
   \pagebreak
   \section{Propriétés algébriques du produit extérieur}
   On peut alors montrer plusieurs propriététés importantes du produit extérieur:
   \begin{itemize}
      \item Le produit extérieur est \textbf{bilinéaire et alterné}.
      \item Si \( \mathcal{B} = (e^i)_{i \leq n} \) est une base duale de \( E \), on a:
      \[ 
         \forall x_1, \ldots, x_n \in E \; ; \; e^1 \wedge \ldots \wedge e^n(x_1, \ldots, x_n) = \text{det}_{\mathcal{B}}(x_1, \ldots, x_n)
      \]
   \end{itemize}
   En particulier on remarque alors que le déterminant canonique dans \( \R^n \) n'est que l'application de la forme multinéaire alternée canonique \( e^1 \wedge \ldots \wedge e^n \) de \( \Lambda^n(\R^n) \).

\chapter{Formes différentielles dans \( \R^n \)}
   Dans ce chapitre on peut maintenant définir un object fondamental de la géométrie différentielle, le concept de \textbf{k-forme différentielle} sur \(\R^n\) qui sera simplement définie par:
   \begin{center}
      \textbf{Une k-forme différentielle est un champs de k-tenseurs covariants antisymétriques.}
   \end{center}
   Ceci s'intérprète alors comme la donnée en chaque point \(p\) de \(\R^n\) d'un tenseur covariant antisymétrique. Formellement, on a dans la base canonique:
   \[
      \omega(p) = \sum_{1 \leq i_1 < \ldots < i_k \leq n} f_{{i_1, \ldots, i_k}}(p) dx^{i_1} \wedge \ldots \wedge dx^{i_k}
   \]
   Où ici on considérera que \(f\) est de classe \( \mathcal{C}^\infty \). On note alors \( \Omega^k(E)\) l'ensemble des fonctions lisses de \( E \) dans \(\Lambda^kE^*\), ie l'ensemble des \( k \)-formes différentielles sur \( E \).
   \section{Notation multi-indices}
   Trés souvent pour simplifier l'expression d'une \( k- \)forme, il sera pratique de noter \(I = (1 \leq i_1 < \ldots < i_k \leq n)\) et d'écrire alors sous forme plus condensée:
   \[ 
      \omega(p) = \sum_I f_{I}(p) dx^I
   \]
   \section{Produit extérieur des formes}
   On sait définir le produit extérieur des tenseurs covariants, on peut alors définir le produit extérieur de deux formes \( \omega, \eta \) d'ordre respectifs \( p, q \) par:
   \[ 
      \begin{aligned}
         \omega \wedge \eta : \R^n &\longrightarrow \Lambda^{p+q}E^* \\
         p &\longmapsto \omega(p) \wedge \eta(p)
      \end{aligned}
   \]
   \section{Produit intérieur des formes}
      Le produit extérieur permet pour une forme \( \omega \) donnée d'augmenter son ordre par produit avec une autre forme \( \eta \). On aura besoin par la suite de la propriété inverse qui à partir de \( \omega \) construit une $k-1$-forme.\<

      Cette opération apellée \textbf{produit intérieur} nécessite alors une autre donnée, celle d'un \textbf{champs de vecteurs} $X$, en effet on peut alors définir:
      $$
         \iota_X(\omega) : p \mapsto \iota_X(\omega)(p) = \omega(F(p), \cdot, \ldots, \cdot)
      $$
      Moralement, la donnée d'un vecteur en tout point permet de définir une $k-1$ forme en considérant la forme de départ avec sa première variable fixée sur le vecteur donné. Ceci se généralise et on peut alors construire une \( k-p \) forme avec la donnée de \( p \) champs de vecteurs.
   \section{Dérivée extérieure}
   On introduit alors un opérateur sur les \(k\)-formes appelée \textbf{dérivée extérieure} qu'on définit par:
   \begin{align*}
      d_k : \Omega^k(E) &\longrightarrow \Omega^{k+1}(E)\\
      \omega &\longmapsto d\omega
   \end{align*}
   Il agit alors sur une \(k\)-forme par différentiation de la fonction coefficients, ie on a:
   \[
      d\omega(p) = \sum_I df(p) \wedge dx^I
   \]
   En particulier, si \( \omega \) est une \( 0 \)-forme, ie une fonction, on retrouve la différentielle d'une fonction.
   \section{Propriétés de la dérivée}
   On peut alors montrer que la dérivée extérieure est bien définie et possède les propriétés suivantes:
   \begin{itemize}
      \item On a que la dérivée est un opérateur linéaire.
      \item On a la \textbf{formule de Leibniz généralisée}, pour \(\omega\) une \(k\)-forme et \(\eta\) une \(l\)-forme : \[d(\omega \wedge \eta) = d\omega \wedge \eta + (-1)^k\omega \wedge d\eta\]
      \item On a la \textbf{propriété fondamentale} de la dérivée extérieure:
      \[ 
         d_{k+1} \circ d_k = 0 
      \]
   \end{itemize}
   \section{Forme volume}
   Dans \(\R^n\), le cas particulier des \(n\)-formes différentielles est fondamental, en effet soit \( \omega \) une telle forme, alors elle vérifie dans la base canonique:
   \[ 
      \forall x \in \R^n \; ; \; \omega(p) =  f(p)dx^1 \wedge \ldots \wedge dx^n
   \]
   Si \( f \) ne s'annule jamais, on dira alors que \( \omega \) est une \textbf{forme volume}. C'est en fait la donnée en tout point de \( \R^n \) d'un multiple du déterminant canonique. On peux alors identifier le déterminant canonique à la \( n \)-forme différentielle (constante) suivante:
   \[ 
      \omega(p) = dx^1 \wedge \ldots \wedge dx^n = \text{det}
   \]
   On a alors, par exemple:
   \begin{itemize}
      \item Dans \(\R^1\), on a \(\text{det} = dx\)
      \item Dans \(\R^3\) on a \(\text{det} = dx \wedge dy \wedge dz\)
   \end{itemize}
   En particulier, on a par exemple dans \(\R^2\), que \((dx \wedge dy)(u, v) = u_1v_2 - u_2v_1 = \text{det}(u, v)\) comme on l'attendrais intuitivement.
