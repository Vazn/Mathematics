\chapter{Intégrale d'une forme dans \( \R^n \)}
\chapter{Intégrale d'une forme sur une variété}
\chapter{Théorème de Stokes-Cartan}

