\chapter{Conclusion \& Bibliographie}
Ce projet particulièrement intéressant m'a permi de découvrir beaucoup de nouvelles théories, de la géométrie différentielle, à l'algèbre tensorielle en passant par des perspectives de topologie algébrique. Les preuves et définitions de ce projet sont principalement tirées de deux ouvrages:
\begin{itemize}
   \item \textit{Introduction to Smooth Manifolds, Springer, 2012} de John M. Lee
   \item \textit{An Introduction to Manifolds, Springer, 201} de Loring W. Tu
\end{itemize}

L'objectif était de réaliser une synthèse extrème de la théorie permettant de démontrer le théorème de Stokes dans le cadre le plus général possible. J'ai l'impression d'avoir réussi ce travail, j'ai fais en sorte de ne jamais reculer devant la généralité tout en essayant de faire le tri entre les notions pertinentes ou non pour l'objectif visé. On pourrait objecter qu'il manque beaucoup de choses importantes, par exemple et sans être exhaustif:
\begin{itemize}
   \item \textit{Quid des sous-variétés ?}
   \item \textit{Quid des formes fermées, exactes ?}
   \item \textit{Quid de variétés riemanniennes, à coins ?}
   \item \textit{Quid des propriétés des champs de vecteurs, les flots, les courbes intégrales ?}
   \item \textit{Quid de la cohomologie de De Rham ?}
\end{itemize}
De manière générale, je pense que la seule réponse à ces questions c'est que j'ai été obligé de faire des choix qui m'ont guidé tout le long de ce projet, et qui m'ont permi (d'essayer) de ne pas trop m'éparpiller. A terme, c'est un projet que je compte continuer dans tout les cas pour ma curiosité personelle, et donc rajouter des sections à propros de tout ces sujets fascinants.\<

Enfin, je voudrais quand même remercier Mr. Berthomieux et Mr. Levy pour leur aide (trés précieuse) pour la bonne réalisation de ce projet.
