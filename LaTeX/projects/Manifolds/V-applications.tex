\chapter{Applications de la théorie des formes}
   \section{Cas de la dimension 3}
   Dans le cas de \(\R^3\), on a la chaîne suivante:
   \[
      \Lambda^0 \R^3 \overset{d_0}{\longrightarrow} \Lambda^1 \R^3 \overset{d_1}{\longrightarrow} \Lambda^2 \R^3 \overset{d_2}{\longrightarrow} \Lambda^3 \R^3
   \]
   On peut alors montrer facilement que les dimensions des différents espaces suivent la suite \((1, 3, 3, 1)\) et les propriétés surprenantes suivantes:
   \begin{itemize}
      \item On a \(d_0\) qui s'identifie \textbf{au gradient de la fonction}.
      \item On a \(d_1\) qui s'identifie \textbf{au rotationnel du champ de vecteurs}.
      \item On a \(d_2\) qui s'identifie \textbf{à la divergence du champ de vecteurs}.
   \end{itemize}
   Et par la propriété fondamentale de la dérivée extérieure, on a alors les formules classiques suivantes comme simple conséquence:
   \[
      \begin{cases}
         \text{rot}(\nabla f) = 0\\
         \text{div}(\text{rot}(F)) = 0\\
      \end{cases}
   \]

\chapter{Applications du théorème de Stokes-Cartan}
