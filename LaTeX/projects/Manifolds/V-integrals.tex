\chapter{Intégrale d'une forme sur une variété}
Dans ce chapitre, nous définissons le concept principal qui menera au théorème de Stokes, ie la notion \textbf{d'intégrale d'une forme volume sur une variété orientée}. L'optique étant de définir tout d'abord l'intégrale d'une telle forme dans \( \R^n \) et \( \mathbb{H}^n \), puis de s'y ramener dans le cas d'une variété.
\section{Intégrale d'une fonction}
Le problème principal dans la définition de l'intégrale sur une variété est le suivant, \textbf{intégrer une fonction dépends des cordonées choisies}. En effet on pourra imaginer définir l'intégrale d'une fonction \( f : M \longrightarrow \R \) à support compact \( D \) et dont celui ci est inclu dans une carte \((U, \phi)\) par:
\[ 
   \int_D f = \int_{\phi(D)} f  \circ \phi^{-1} 
\] 
Mais en fait cette intégrale serait alors mal définie ! En effet si \( D \) est inclu dans deux cartes différentes \( (U, \phi), (V, \psi) \), les deux intégrales correspondantes sont différentes ! Ceci permet alors de justifier l'assertion suivante: \textbf{Les fonctions ne sont en fait pas la bons objets à intégrer.}
\section{Intégrale locale sur \( \R^n \)}
Les objets naturels pour être intégrés sur une variété de dimension \( n \) sont donc en fait les \(n\)-formes. Dans le cas simple de \( \R^n \), on considère une telle forme \(\omega = f(x_1, \ldots, x_n) dx^1 \wedge \ldots \wedge dx^n\). Si \( f \in L^1(\R^n) \), alors on dira que cette forme est \textbf{intégrable} et on définit son intégrale par: 
\[ 
   \int_{\R^n} \omega := \int_{\R^n} f(x_1, \ldots, x_n)dx^1 \wedge \ldots \wedge dx^n = \int_{\R^n} f(x_1, \ldots, x_n)dx^1\ldots dx^n
\]
On notera alors plus simplement \(\omega \in L^1(\R^n)\) si la fonction coefficient de \( \omega \) est intégrable, même si c'est un léger abus de notation.
\section{Formule du changement de variable dans \( \R^n \)}
Etant donnée une \( n-\)forme \( \omega \in L^1(\R^n) \) et un difféomorphisme \( F : \R^n \longrightarrow \R^n \), on montrer la formule suivante:
\[ 
   \int_{\R^n} F^*\omega = 
   \begin{cases}
      \displaystyle+\int_{\R^n} \omega \text{ Si $F$ \textbf{préserve} l'orientation.}\vspace{5pt}\\
      \displaystyle-\int_{\R^n} \omega \text{ Si $F$ \textbf{inverse} l'orientation.}
   \end{cases}
\]
En particulier, si \( F \) est de déterminant positif, on a égalité et la formule du changement de variable:
\[ 
   \int_{\R^n} \omega dx^1 \wedge \ldots \wedge dx^n = \int_{\R^n} (\omega \circ F) \text{det}(J_F) dx^1 \wedge \ldots \wedge dx^n
\]
\section{Intégrale locale sur \( M \)}
On souhaite alors définir l'intégrale d'une \( n-\)forme \(\omega\) sur une variété orientée \( M \) et munie d'un \textbf{atlas orienté}. Si elle est telle que son support soit inclu dans une carte \( (U, \phi) \). Alors on dira que \( \omega \) est \textbf{intégrable} ssi \( (\phi^{-1})^*\omega \) l'est, et alors on définit:
\[ 
   \int_U \omega := \int_{ \phi(U)} (\phi^{-1})^*\omega 
\]
Cette expression est bien définie, ie ne dépends pas du choix de la carte. En effet par construction, les changements de cartes sont de déterminant positif et on applique alors la formule du changement de variables.
\section{Intégrale sur \( M \)}
On peut alors finalement définir l'intégrale d'une \( n-\)forme sur toute la variété. En effet soit \( \omega \in \Omega^n(M) \), alors on considère une partition de l'unité \( (\rho_\alpha) \) subordonée à l'atlas. On dira alors que \( \omega \) est \textbf{intégrable} si et seulement si \( (\phi_\alpha^{-1})^*\rho_\alpha \omega  \) l'est pour tout \( \alpha \) et alors on définit son intégrale par:
\[ 
   \int_M \omega := \sum_\alpha \int_{U_\alpha} \rho_\alpha \omega 
\]
On peut alors montrer que celle ci est bien définie, ie:
\begin{itemize}
   \item Elle ne dépend pas du choix de l'atlas orienté.
   \item Elle ne dépend pas du choix de la partition de l'unité.
\end{itemize}
\section{Exemples et cas particuliers}
On peut alors chercher à exprimer l'intégrale d'une fonction lisse à support compact sur un domaine simple comme un cas particulier d'intégrale d'une forme différentielle, et en effet c'est le cas:
\begin{itemize}
   \item Si on considère la 1-forme \( \omega = f(x)dx \) et \( \Gamma = \ioo{a}{b} \subseteq \R\), on obtient:
   \[ 
      \int_\Gamma \omega := \int_{\ioo{a}{b}}  \gamma^*\omega = \int_{\ioo{a}{b}}  \omega_{t}(\text{Id}(t)) = \int_{\ioo{a}{b}} f(t)dt
   \]
   \item Si on considère la 2-forme \( \omega = f(x, y)dx \wedge dy \) et \( \Sigma = \ioo{0}{1}^2 \subseteq \R^2 \), on obtient:
   \[ 
      \int_\Sigma \omega := \int_{\ioo{0}{1}^2}  \Sigma^*\omega = \int_{\ioo{0}{1}^2}   \omega_{u, v}(\text{Id}(u, v)) = \int_{\ioo{0}{1}^2}  f(u, v)dudv
   \]
\end{itemize}
Aussi, ces dernières intégrales s'interpètent à nouveau comme des intégrales sur les variétés \( \ioo{a}{b}, \ioo{0}{1}^2 \) ? Peut on toujours interpréter le signe intégrale comme l'intégrale d'une forme sur une variété ?\<

Néanmoins c'est bien une notion plus générale car elle nous permettra, à terme, de calculer l'intégrale de la 1-forme \( xdy + ydx \in \Omega^1(\R^2) \) qui n'est pas de la forme \( f(t)dt \) ceci sur une courbe (sous-variété de \( \R^2 \)), mais ce concept sera probablement omis car non nécessaire à Stokes et le temps manquera probablement.

\chapter{Théorème de Stokes-Cartan}
Dans tout les chapitres précédents, nous avons présenté un cadre théorique suffisant pour énoncer et comprendre le théorème fondamental de l'intégration, généralisation du théorème fondamental de l'analyse appellé \textbf{théorème de Stokes-Cartan}. On considère une variété \( M \) vérifiant plusieurs hypothèses:
\begin{itemize}
   \item Elle est \textbf{compacte}.
   \item Elle est \textbf{orientable}.
\end{itemize}
On considère aussi une \( n-1 \) forme \( \omega \), alors on peut montrer le théorème suivant:
\[ 
   \int_M d\omega = \int_{\partial M} \omega
\]
Où ici \( \partial M \) est munie de l'orientation induite par \( M \). On peut alors faire plusieurs remarques sur cet énoncé:
\begin{itemize}
   \item Si \( M \) est \textbf{sans bords}, alors on a \( \partial M = \emptyset \) donc l'intégrale est nulle.
   \item Si \( M \) est de dimension \( 1 \), et notamment si \( M = \icc{a}{b} \), on retrouve le théorème fondamental de l'analyse.
\end{itemize}
