\chapter{Elements d'algèbre tensorielle}
   Dans ce chapitre, nous allons nous intéresser à l'object d'étude du domaine appelé \textbf{algèbre multilinéaire}, qui sont les \textbf{formes multilinéaires}, en particulier, on se donne un \(\R\)-espace vectoriel \(E\) de dimension \( n \), alors on appele \textbf{tenseur} d'ordre \((p, q)\) une application de la forme suivante:
   \begin{align*}
      T : \underbrace{E^* \times \ldots \times E^*}_\text{p} \times \underbrace{E \times \ldots \times E}_\text{q} \longrightarrow \K
   \end{align*}
   Dans le contexte de ce projet, on s'intéresse principalement à la construction des fromes différentielles, et donc on s'intéressera surtout au cas où \( p = 0 \). On dira alors que \( T \) est un tenseur \textbf{covariant}.
   \section{Structure de l'espace des tenseurs covariants}
   On note alors \(\mathscr{T}^p(E)\) l'ensemble des \(p\)-tenseurs covariants, alors l'addition de deux formes et la multiplication par un scalaire étant bien définie, on peut montrer la propriété suivante:
   \begin{center}
      L'espace \( \mathscr{T}^p(E) \) a une structure de \( \K \)-espace vectoriel.
   \end{center}
   \section{Produit tensoriel de deux tenseurs covariants}
   On peut alors définir un produit sur des tels objets appelé \textbf{produit tensoriel} défini par:
   \begin{align*}
      \otimes : \mathscr{T}^p(E) \times \mathscr{T}^q(E) &\longrightarrow \mathscr{T}^{p+q}(E)\\
      (\alpha, \beta) &\longmapsto \alpha \otimes \beta
   \end{align*}
   Avec le tenseur \(\alpha \otimes \beta\) défini par:
   \[
      (\alpha \otimes \beta)(x_1, \ldots, x_p, y_1, \ldots, y_q) = \alpha(x_1, \ldots, x_p)\beta(y_1, \ldots, y_q)
   \]
   \section{Base et dimension}
      On peut alors se demander si on peut trouver une base de cet espace, et en effet si on note \((e_i)_{i \leq n}\) une base de \(E\), alors on peut montrer que l'on a:
      \begin{flalign*}
         T(x_1, \ldots, x_p) &= T\left( \sum_{i_1 \leq n} x_{1, i_1}e_{i_1}, \ldots, \sum_{i_p \leq n} x_{p, i_p}e_{i_p} \right)\\
         &= \sum_{i_1, \ldots, i_p \leq n} x_{1, i_1} \ldots x_{p, i_p} T(e_{i_1}, \ldots, e_{i_p})
      \end{flalign*}
      Mais on remarque alors que le produit \( x_{1, i_1} \ldots x_{p, i_p} \) consiste alors en l'évaluation de \(e_{i_1}^* \otimes \ldots \otimes e_{i_p}^* \) en \( (x_1, \ldots, x_p) \) et donc on obtient:
      \begin{flalign*}
         T &= \sum_{i_1, \ldots, i_p \leq n} T(e_{i_1}, \ldots, e_{i_p}) e_{i_1}^* \otimes \ldots \otimes e_{i_p}^*
      \end{flalign*}
      En d'autres termes tout \( p \)-tenseur \( T \) est engendré par la famille de \( n^p \) vecteurs \( (e_{i_1}^* \otimes \ldots \otimes e_{i_p}^*)_{i_1, \ldots, i_p \leq n} \). On peut alors montrer qu'elle est libre et donc que c'est une base de \(\mathcal{T}^p(E)\).
  
   \pagebreak
   \section{Tenseurs antisymétriques}
   On appelle \textbf{tenseur antisymétrique} tout \( p \)-tenseur \( T \) tel que:
   \[ 
      \forall i, j \in \inticc{1}{p} \; ; \; T(\ldots, x_j, \ldots, x_i, \ldots) = -T(\ldots, x_i, \ldots, x_j, \ldots)
   \]
   \section{Antisymétrisation}
   On se donne un tenseur \(T\) qui soit \(p\)-covariant, alors on chercher à construire un tenseur \(p\)-covariant \textbf{antisymétrique} à partir de \(T\), et on peut alors montrer que le tenseur suivant convient:
   \[
      \text{Asym}(T)(x_1, \ldots, x_p) = \frac{1}{k!}\sum_{\sigma \in S_p}\epsilon(\sigma)T(x_{\sigma(1)}, \ldots, x_{\sigma(p)})
   \]
   En d'autres termes que \( T \) est bien antisymétrique.
   \section{Produit extérieur}
   On peut alors définir un produit antisymétrique appelé surtout \textbf{produit extérieur} de deux tenseurs par \textbf{\color{red}\underline{Problème}: Il manque un coefficient pour que la base de la section d'aprés soit correcte ?}:
   \[
      (T \wedge T') = \text{Asym}(T \otimes T')
   \]
   En d'autres termes:
   \[
      (T \wedge T')(x_1, \ldots, x_p, x_{p+1}, \ldots, x_{p+q}) =  \frac{1}{(p+q)!}\sum_{\sigma \in S_{p+q}}\epsilon(\sigma) T(x_{\sigma(1)}, \ldots, x_{\sigma(p)})T'(x_{\sigma(p+1)}, \ldots, x_{\sigma(p+q)})
   \]
   C'est ce produit extérieur qui nous sera surtout utile pour définir les formes différentielles.
   \section{Algèbre extérieure}
   On appelle alors \textbf{p-ième puissance extérieure} l'ensemble de toutes les formes \(p\)-linéaires alternées qu'on note \(\Lambda^p E^*\). Une base est alors donnée par l'ensemble:
   \[
      \Bigl\{ e_{i_1}^* \wedge \ldots \wedge e_{i_p}^* \; ; \; 1 \leq e_{i_1}^* < \ldots < e_{i_p}^* \leq n  \Bigl\}
   \]
   En effet, on utilise la même approche que pour le produit tensoriel, soit \( T \) un p-tenseur antisymétrique, alors:
   \begin{flalign*}
      T &= \sum_{i_1, \ldots, i_p \leq n} T(e_{i_1}, \ldots, e_{i_p}) e_{i_1}^* \otimes \ldots \otimes e_{i_p}^*
   \end{flalign*}
   Mais \( T \) est antisymétrique, donc les indices des termes non nuls sont différents, ie on somme en fait sur \( I = \left\{ (i_1, \ldots, i_p) \; ; \; \forall p, q \; i_p \neq i_q \right\}\). Aussi par un raisonnement combinatoire, il est équivalent de sommer sur des indices distincts et de sommer sur des indices strictement croissants puis sur toutes les permutations de ceux ci, ie on a:
   \begin{flalign*}
      T &= \sum_{1 \leq i_1 < \ldots < i_p \leq n} \sum_{\sigma \in S_p} T(e_{\sigma(i_1)}, \ldots, e_{\sigma(i_p)}) e_{\sigma(i_1)}^* \otimes \ldots \otimes e_{\sigma(i_p)}^*\\
      &= \sum_{1 \leq i_1 < \ldots < i_p \leq n} T(e_{i_1}, \ldots, e_{i_p}) \sum_{\sigma \in S_p} \epsilon(\sigma) e_{\sigma(i_1)}^* \otimes \ldots \otimes e_{\sigma(i_p)}^*\\
      &= {\color{red}p!} \sum_{1 \leq i_1 < \ldots < i_p \leq n} T(e_{i_1}, \ldots, e_{i_p}) e_{i_1}^* \wedge \ldots \wedge e_{i_p}^*
   \end{flalign*}
   En particulier, un tel tenseur est uniquement déterminé par son image sur toutes les vecteurs de la base indéxés par une suite strictement croissante, et il y a \( \binom{n}{p} \) telles suites, c'est donc la dimension de \(\Lambda^p E^*\).\<

   \uline{Exemple:} Si \( E = \R^3 \), on note la base duale \((dx, dy, dz)\), alors on a que:
   \[
      \Lambda^2E^* = \text{Vect}(dx \wedge dy, dx \wedge dz, dy \wedge dz)
   \]
   On appele aussi les éléments de la \(p\)-ième puissance extérieure des \textbf{multivecteurs}.

\chapter{Formes différentielles dans \( \R^n \)}
   Dans ce chapitre on peut maintenant définir un object fondamental de la géométrie différentielle, le concept de \textbf{p-forme différentielle} sur \(\R^n\) qui sera simplement définie par:
   \begin{center}
      \textbf{Une p-forme différentielle est un champs de tenseurs covariants antisymétriques.}
   \end{center}
   Ceci s'intérprète alors comme la donnée en chaque point \(x\) de \(\R^n\) d'un tenseur covariant antisymétrique. Formellement, on a:
   \[
      \omega(x) = \sum_{1 \leq i_1 < \ldots < i_p \leq n} f_{{i_1, \ldots, i_p}}(x) dx_{i_1} \wedge \ldots \wedge dx_{i_p}
   \]
   Enfin, on considérera pour simplifier que \(f\) est de classe \( \mathcal{C}^\infty \), on note alors \( \Omega^k(E)\) l'ensemble des fonctions lisses de \( E \) dans \(\Lambda^kE^*\), ie l'ensemble des \( k \)-formes différentielles sur \( E \).

   \section{Dérivée extérieure}
   On introduit alors un opérateur sur les \(p\)-formes appelée \textbf{dérivée extérieure} qu'on définit par:
   \begin{align*}
      d_k : \Omega^k(E) &\longrightarrow \Omega^{k+1}(E)\\
      \omega &\longmapsto d\omega
   \end{align*}
   Il agit alors sur une \(k\)-forme par différentiation de la fonction coefficients, ie on a:
   \[
      d\omega(x) = \sum_{1 \leq i_1 < \ldots < i_p \leq n} df(x) \wedge dx_{i_1} \wedge \ldots \wedge dx_{i_p}
   \]
   \uline{Exemple 1:} Si on considère la 1 forme de \(\R^3\) suivante:
   \[
      \omega = y dx 
   \]
   Alors on différentie simplement la fonction coefficient et on obtient:
   \[
      d\omega = d(ydx) = \left(\partialD{y}{x}dx + \partialD{y}{y}dy\right) \wedge dx = dy \wedge dx
   \]
   \uline{Exemple 2:} On considère la 2-forme de \(\R^3\) suivante:
   \[
      \omega = A(x, y, z)dx \wedge dy
   \]
   Alors on différentie simplement la fonction coefficient et on obtient:
   \begin{align*}
      d\omega = d(A(x, y, z)dx \wedge dy) &= \left(\partialD{A}{x}dx + \partialD{A}{y}dy + \partialD{A}{z}dz\right) \wedge dx \wedge dy \\
      &= \left(\partialD{A}{x} - \partialD{A}{y}\right) dx \wedge dy - \partialD{A}{z}dx \wedge dz - \partialD{A}{z}dy \wedge dz
   \end{align*}
   \section{Propriétés de la dérivée}
   On peut alors montrer que la dérivée extérieure est bien définie et possède les propriétés suivantes:
   \begin{itemize}
      \item On a que la dérivée est un opérateur linéaire.
      \item On a la \textbf{formule de Leibniz généralisée}, pour \(\omega\) une \(k\)-forme et \(\alpha\) une \(l\)-forme : \[d(\omega \wedge \alpha) = d\omega \wedge \alpha + (-1)^k\omega \wedge d\alpha\]
      \item On a la \textbf{propriété fondamentale} de la dérivée extérieure:
      \[ 
         d_{k+1} \circ d_k = 0 
      \]
   \end{itemize}
   \section{Forme volume}
   Dans \(\R^n\), le cas particulier des \(n\)-formes différentielles est fondamental, en effet on appelle ces formes \textbf{formes volumes} et si on fixe une base \( \mathcal{B} = (e_1, \ldots, e_n) \), elles s'écrivent toutes de la forme:
   \[
      \text{vol}_\mathcal{B} = f(e_1, \ldots, e_n)e_1^* \wedge \ldots \wedge e_n^*
   \]
   En particulier \( \dim(\Lambda^n( \R^n)) = 1\) et si on fixe la contrainte que \( \text{vol}_\mathcal{B}(e_1, \ldots, e_n) = 1 \), on retrouve:
   \[ 
      \text{vol}_\mathcal{B}(x_1, \ldots, x_n) = \sum_{\sigma \in S_n} \epsilon(\sigma) x_{1, \sigma(1)} \ldots x_{n, \sigma(n)} = \text{det}_B(x_1, \ldots, x_n)
   \]
   De manière générale dans \( \R^n \), on fixe la base canonique \( \mathcal{C} \) comme base de référence et on a alors, par exemple:
   \begin{itemize}
      \item Dans \(\R^1\), on a \(\text{det} = dx\)
      \item Dans \(\R^3\) on a \(\text{det} = dx \wedge dy \wedge dz\)
   \end{itemize}
   En particulier, on a par exemple dans \(\R^2\), que \(\text{vol}(u, v) = (dx \wedge dy)(u, v) = u_1v_2 - u_2v_1 = \text{det}_\mathcal{C}(u, v)\) comme on l'attendrais intuitivement.

\chapter{Formes différentielles sur une variété}
   Dans ce chapitre et le chapitre précédent, on a défini les tenseurs, les formes différentielles et la dérivée extérieure sur un espace vectoriel \( E \) de dimension \( n \) et plus simplement sur \( \R^n \). Le cas qui nous intéressera alors dans toute la suite est celui où \( E = TM_x \), alors dans ce cas, on peut étendre ces notions au cas des variétés en définissant un tenseur d'un espace tangent, une forme différentielle d'un espace tangent.\< 
   
   Alors, par exemple, une forme différentielle sur un espace tangent sera un champs de $p$-tenseurs antisymétriques, ie un champs d'éléments de \( \Lambda^p(TM^*_x) \).\<

   Et la dérivée extérieure sur un espace tangent sera de la forme:
   \[ 
      \begin{aligned}
         d : \Omega^k(TM_x) &\longrightarrow \Omega^{k+1}(TM_x)\\
         \omega &\longmapsto d\omega
      \end{aligned}
   \]

\chapter{Orientation d'une variété}
   Dans ce chapitre on s'intéresse plus particulière aux \textbf{formes volumes} sur une variété \( M \), en effet celles ci permettent de définir l'orientabilité d'une variété, et si c'est le cas de convenir d'une orientation. En outre, si la variété est à bord, alors l'orientation d'une variété induit une orientation du bord de celle-ci.