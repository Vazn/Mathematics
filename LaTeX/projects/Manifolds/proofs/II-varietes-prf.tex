\section*{\uline{Chapitre 4 - Variétés {:}}}
   \subsection*{La différentiabilité est bien définie {:}}
   Soit \( f : M \longrightarrow \R \), il s'agit de montrer que pour \( p \in M \) et deux cartes \( (U, \phi), (V, \psi) \) qui le contiennent, alors:
   \[ 
      f \circ \phi \text{ différentiable} \iff  f \circ \psi \text{ différentiable}
   \]
   Alors on a: 
   \[ 
      f \circ \psi^{-1} = f \circ \phi^{-1} \circ \phi \circ \psi^{-1} = f \circ \phi^{-1} \circ c_{\psi, \phi}
   \]
   Où \( c \) est l'application de changement de carte associée. On en déduit donc que si \( f \circ \psi^{-1} \) est différentiable alors \(f \circ \phi^{-1}\). De manière analogue si \( f : M \longrightarrow N \) est différentiable pour deux cartes \( (U, \phi), (V, \psi) \) de \( p, f(p) \), alors par composition, on en déduis de même.
\section*{\uline{Chapitre 7 - Espaces tangents dans \( \R^n \) {:}}}
   \subsection*{Propriété des dérivations {:}}
   Si \( f \) est une fonction lisse constante égale à \( c \in \R \), et \( D \) une dérivation au point \( p \in \R^n \), alors \( Df = 0 \), en effet on a:
   \begin{flalign*}
      D(fg) &= D(f)g(p) + f(p)D(g) \iff \\
      D(cg) &= D(f)g(p) + cD(g) \iff \shorteqnote{(Car \( fg = cg \) en tant que fonction.)}\\
      cD(g) &= D(f)g(p) + cD(g) \shorteqnote{(Linéarité de \( D \).)}
   \end{flalign*}
   On conclut de la dernière égalité que \( D(f)g(p) = 0 \) et il suffit alors de choisir \( g \) de telle sorte qu'elle ne s'anulle pas en \( p \) pour conclure.
   \subsection*{\( T\R^n_p \) est un espace vectoriel isomorphe à \( \R^n \) {:}}
   On considère l'application suivante:
   \[ 
      \begin{aligned}
         \Phi : \R^n_p &\longrightarrow T\R^n_p \\
         v &\longmapsto \sum_{i \leq n} v_i \partialD{}{x_i}\biggr|_p
      \end{aligned} 
   \]
   \uline{Linéarité:} Découle directement de la définition ..\+
   \uline{Injectivité:} Supposons que \( v \in \R^n_p \) soit tel que:
   \[ 
      D = \Phi(v) = 0
   \]
   Alors pour toute fonction lisse \( f \in \mathcal{F}(\R^n_p, \R) \), on a \( Df = 0 \). Alors en particulier pour les fonctions coordonées \( (x_i)_{i \leq n} \), on a:
   \[ 
      \sum_{j \leq n}v_j \partialD{x_i}{x_j}\biggr|_p = v_i = 0
   \]
   Donc toutes les composantes de \( v \) sont nulles et \( v = 0 \).\+
   \uline{Surjectivité:}
   {\textbf{\color{red} A finir !}}
\section*{\uline{Chapitre 8 - Espaces tangents dans \( M \) {:}}}
   \subsection*{Propriété de la différentielle {:}}
      On considère \( f: M \longrightarrow N \) lisse et un point \( p \in M \), et on considère l'application suivante:
      \[ 
         df_p : D \in TM_p \longmapsto D( \cdot  \circ f) \in TN_{f(p)}
      \]
      \begin{itemize}
         \item \textbf{Bien définie:} Si \( D \) est une dérivation en \( p \), alors \( g \circ f \) est bien lisse et on a: 
         \begin{flalign*}
            df_p(D)(gh) &= D(gh \circ f)\\
            &= D((g \circ f)(h \circ f))\\
            &= D(g \circ f)(h \circ f)(p) + (g \circ f)(p)D(h \circ f) \shorteqnote{(Car \( D \) est une dérivation en \( p \).)}\\
            &= df_p(g)h(f(p)) + g(f(p))df_p(h)
         \end{flalign*}
         Et donc \( df_p(D) \) est bien une dérivation en \( f(p) \).
         \item \textbf{Linéarité:} Découle directement de la définition.
         \item \textbf{Règle de la chaîne:} Soit \( f : M \longrightarrow N \) et \( g : N \longrightarrow L \) deux applications lisses entre des variétés. Soit \( p \in M \), alors par définition:
         \[ 
            d(g \circ f)_p : D \in TM_p \longmapsto D(\cdot \circ g \circ f)) \in TL_{g(f(p))} 
         \]
         Il s'agit de montrer que cette fonction est égale à \(dg_{f(p)} \circ df_p \), les domaines de définition coincident bien, soit \( D \) une dérivation de \( TM_p \), alors on a:
         \[ 
            dg_{f(p)} \circ df_p(D) = dg_{f(p)}(df_p(D)) = dg_{f(p)}(D( \cdot \circ f)) = D(\cdot \circ g \circ f)
         \]
         On a donc égalité des images et donc égalité des fonctions.
      \end{itemize}
   \subsection*{Propriété de structure {:}}
      Finalement montrons que si \( f : M \longrightarrow N \) est un \textbf{difféomorphisme}, alors \( df_p \) en un \textbf{isomorphisme} en tout point. On utilisera le lemme (presque évident) suivant:
      \[ 
         d(Id_M)_p = Id_{TM_p} 
      \]
      Alors si \( f \) est un difféormorphisme, en tout point \( p \), on a:
      \[ 
         \begin{cases}
            d(f^{-1} \circ f)_p = df^{-1}_{f(p)} \circ df_p = Id_{TM_p}\\
            d(f \circ f^{-1})_{f(p)} = df_{p} \circ df^{-1}_{f(p)} = Id_{TN_{f(p)}}
         \end{cases}
      \]
      Finalement, ceci montre que \( df_p \) est bijective d'inverse \( df^{-1}_{f(p)} \) et linéaire par définition, on a donc bien:
      \[ 
         TM_p \cong TN_{f(p)} 
      \]
      On en déduit directement en utilisant que toute carte \( \phi \) est un difféomorphisme et donc qu'en tout point d'une variété on a:
      \[ 
         TM_p \cong T\R^n_{\phi(p)} \cong \R^n 
      \]

   \subsection*{Changement de représentation des vecteurs tangents {:}}
   Si \( (U, \phi), (V, \psi) \) sont deux cartes qui contiennent un point \( p \), alors si on note \( x_i, y_i \) les coordonées respectives dans la première et la deuxième carte et \( c = \psi \circ \phi^{-1} \) l'application de changement de carte alors pour toute fonction lisse \( g \), on a:   
   \begin{flalign*}
      \partialD{}{x_i}\biggr|_p g &= \partialD{g \circ \phi^{-1}}{x_i}(\phi(p))\\
      &= \partialD{g \circ \psi^{-1} \circ \psi \circ \phi^{-1}}{x_i}(\phi(p))\\
      &= \sum_{j \leq n} \partialD{g \circ \psi^{-1}}{y_j}(\psi(p))\partialD{c_j}{x_i}(\phi(p)) \shorteqnote{(Règle de la chaîne dans \( \R^n \))}\\
      &= \sum_{j \leq n} \partialD{c_j}{x_i}(\phi(p))\partialD{}{y_j}\biggr|_pg
   \end{flalign*}
   Ceci étant vrai pour toute application \( g \), on a donc:
   \[ 
      \partialD{}{x_i}\biggr|_p = \sum_{j \leq n} \partialD{c_j}{x_i}(\phi(p)) \partialD{}{y_j}\biggr|_p
   \]
   Si on note \( (x_1, \ldots, x_n)\) les composantes de \( \phi(p) \), on a donc:
   \[ 
      \partialD{}{x_i}\biggr|_{\phi^{-1}(x_1, \ldots, x_n)} = \sum_{j \leq n} \partialD{c_j}{x_i}(x_1, \ldots, x_n) \partialD{}{y_j}\biggr|_{\phi^{-1}(x_1, \ldots, x_n)}
   \]
   \subsection*{Le fibré tangent est un variété de dimension \( 2n \) - Partie 1 {:}}
      On définit donc le fibré tangent à \( M \) comme la réunion disjointe \( \bigsqcup_{p \in M} TM_p \), alors le seul objet à notre disposition est la projection:
      \[ 
         \begin{aligned}
            \pi : TM &\longrightarrow M \\
            (p, v) &\longmapsto p
         \end{aligned} 
      \]
      On procède par étapes:
      \begin{itemize}
         \item Tout d'abord on définit une \textbf{topologie} sur \( TM \) gràce à \( \pi \) en effet, on considère la topologie engendrée par la famille suivante:
         \[ 
            \left\{  \pi^{-1}(\mathcal{O}) \; ; \; \mathcal{O} \in \mathcal{T}_M \right\} 
         \]
         C'est en fait la plus petite topologie qui rende \( \pi \) continue.
         \item On doit alors construire un atlas de \( TM \), soit \( (p, v) \in TM\), alors étant donnée une carte \( (U, \phi) \) qui contient \( p \), on remarque que \( \pi^{-1}(U) \) est un ouvert de \( TM \), et on définit un prototype de carte locale pour un point du fibré par:
         \[ 
            \begin{aligned}
               \widetilde{\phi} : \pi^{-1}(U) &\longrightarrow \phi(U) \times \R^n \\
               (p, v) &\longmapsto (\phi(p), v_1, \ldots, v_n)
            \end{aligned}
         \]
         Où ici les \( (v_i) \) sont les coordonées du vecteur \( v \in TM_p \) dans la base associée à la carte locale \( \phi \).
      \end{itemize}
      Définissons maintenant l'atlas sur \( TM \), on considère l'atlas \( (U_i, \phi_i) \) de \( M \), alors elles recouvrent \( M \). On pose alors \((\pi^{-1}(U_i), \Phi_i)\) où \( \Phi_i \) est la carte locale définie plus haut. On doit montrer que:
      \begin{enumerate}
         \item Les \( \pi^{-1}(U_i) \) recouvrent \( TM \).
         \item Les \( \Phi_i \) sont des homéomorphismes sur leurs images.
         \item Les applications de changement de cartes sont lisses.
      \end{enumerate}
   \subsection*{Le fibré tangent est un variété de dimension \( 2n \) - Partie 2 {:}}
      \begin{enumerate}
         \item Soit \( (p, v) \) un point de \( TM \), alors si \( (U, \phi) \) est une carte qui contient \( p \), on a que \((p, v) \in \pi^{-1}(U)\), donc on a bien un recouvrement.
         \item Les \( \widetilde{\phi}_i \) sont des bijections sur leurs images, en effet on peut écrire l'inverse pour tout \((x_1, \ldots, x_n, v_1, \ldots, v_n) \in \phi(U) \times \R^n \), on pose:
         \[ 
            \widetilde{\phi}^{-1}(x_1, \ldots, x_n, v_1, \ldots, v_n) = \left(\phi^{-1}(x_1, \ldots, x_n), \sum_{i \leq n} v_i\partialD{}{x_i}\bigg|_{\phi^{-1}(x_1, \ldots, x_n)}\right)
         \]
         Aussi, \( \widetilde{\phi}, \widetilde{\phi}^{-1} \) sont continues par composantes, pour la première composante car \( \phi \) est continue, et pour les autres ?
         \item Les applications de changement de cartes sont lisses, en effet soit \( (U, \phi), (V, \psi) \) deux cartes qui contiennent \( p \), alors \( (\pi^{-1}(U), \Phi), (\pi^{-1}(V), \Psi) \) sont les cartes correspondantes de \( TM \), alors:
         \[ 
            \begin{cases}
               E = \Phi(\pi^{-1}(U) \cap \pi^{-1}(V)) = \phi(U \cap V) \times \R^n\\
               F = \Psi(\pi^{-1}(U) \cap \pi^{-1}(V)) = \psi(U \cap V) \times \R^n
            \end{cases}
         \]
         Et donc on peut calculer l'application de changement de cartes:
         \begin{align*}
            \Psi \circ \Phi^{-1}(x_1, \ldots, x_n, v_1, \ldots, v_n) 
            &= \Psi\left(\phi^{-1}(x_1, \ldots, x_n), \sum_{i \leq n} v_i \partialD{}{x_i}\bigg|_{\phi^{-1}(x_1, \ldots, x_n)} \right)\\ 
            &= \left(\psi \circ \phi^{-1}(x_1, \ldots, x_n), \sum_{i \leq n}\sum_{j \leq n} v_i \partialD{c_j}{x_i}(x_1, \ldots, x_n) \partialD{}{y_j}\biggr|_{\phi^{-1}(x_1, \ldots, x_n)} \right)
         \end{align*} 
      \end{enumerate}
\section*{\uline{Chapitre 9 - Espaces cotangents dans \( M \) {:}}}
   \subsection*{Différentielle abstraite et différentielle usuelle - Partie 1:}
      L'expression de la différentielle d'une fonction \( f : M \longrightarrow \R \) s'identifie à l'expression usuelle d'une différentielle, en effet, si \( v \in TM_p \), alors pour toute fonction lisse \( g : \R \longrightarrow \R \) on a:
      \begin{align*}
         df_p(v)(g) &= \sum_{i=1}^n v_i df_p\left(\partialD{}{x_i}\biggr|_p\right)(g)\\
         &= \sum_{i=1}^n v_i \partialD{g \circ f}{x_i}(p)\\
         &= \sum_{i=1}^n v_i \partialD{g}{t}(f(t))\partialD{f}{x_i}(p)\\
         &= \left(\sum_{i=1}^n v_i \partialD{f}{x_i}(p)\partialD{}{t}\biggr|_{f(p)}\right)(g)
      \end{align*}
      Ceci étant vrai pour toute fonction lisse \( g \), on trouve l'expression suivante:
      \[ 
         df_p(v) = \sum_{i=1}^n v_i \partialD{f}{x_i}(p)\partialD{}{t}\biggr|_{f(p)}
      \]
   \subsection*{Différentielle abstraite et différentielle usuelle - Partie 2:}
      L'expression de la différentielle d'une fonction \( f : M \longrightarrow N \) s'identifie aussi à l'expression usuelle d'une différentielle, en effet, si \( v \in TM_p \), alors pour toute fonction lisse \( g : M \longrightarrow \R \) on a:
      \begin{align*}
         df_p(v)(g) &= \sum_{i=1}^n v_i df_p\left(\partialD{}{x_i}\biggr|_p\right)(g)\\
         &= \sum_{i=1}^n v_i \partialD{g \circ f}{x_i}(p)\\
         &= \sum_{i=1}^n v_i \sum_{j=1}^m \partialD{g}{y_j}(f(p)) \partialD{f_j}{x_i}(p)\\
         &= \left(\sum_{i=1}^n\sum_{j=1}^m v_i \partialD{f_j}{x_i}(p)\partialD{}{y_j}\biggr|_{f(p)}\right)(g)
      \end{align*}
      Où ici \( f_j \) représente les coordonées locales de \( f \) au voisinage de \( f(p) \), ie \( f_j = (\psi \circ f)_j(p) \). Ceci étant vrai pour toute fonction lisse \( g \), on trouve l'expression suivante:
      \[ 
         df_p(v) = \sum_{i=1}^n\sum_{j=1}^m v_i \partialD{f_j}{x_i}(p)\partialD{}{y_j}\biggr|_{f(p)}
      \]
\section*{\uline{Chapitre 10 - Formes différentielles dans \( M \) {:}}}
   \subsection*{Existence de la dérivée extérieure locale - Analyse {:}}
      On procède par analyse-synthèse, et on considère une dérivée extérieure \( d \) qui vérifie les axiomes. Alors pour toute \( k- \)forme \( \omega \in \Omega^k(U)\), on a nécessairement:
      \begin{flalign*}
         d\omega_p &= d\left(\sum_I \omega_I(p) dx^I\right)\\
         &=\sum_I d(\omega_I(p) dx^I)\shorteqnote{(Linéarité)} \\
         &= \sum_I d\omega_I(p) \wedge dx^I + \omega_I(p)d(dx^I)\shorteqnote{(Leibniz)}\\
         &= \sum_I d\omega_I(p) \wedge dx^I
      \end{flalign*}
      La dernière ligne venant du fait que \( d(dx^I) = 0 \), en effet on explicite les indices et par une récurrence assez évidente et \( d^2 = 0 \) on conclut.
   \subsection*{Existence de la dérivée extérieure locale - Synthèse {:}}
      Pour tout \( p \in M \), pour toute \( k- \)forme \( \omega \in \Omega^k(U)\), alors on définit dans les coordonées locales associées à une carte \( (U, \phi) \) l'opérateur suivant:
      \[ 
         d\omega_p = \sum_I d\omega_I(p) \wedge dx^I
      \]
      Alors on doit montrer que cet opérateur est bien une dérivée extérieure.
   
   \subsection*{Généralisation de la différentielle {:}}
   Si \( f \in \Omega^0(M) \), c'est une fonction lisse, et alors \( d \) agit par définition par différentiation des coefficients, l'unique coefficient de la \( 0 \)-forme \( f \), étant elle même, on conclut.
   \subsection*{Linéarité de la dérivée extérieure {:}}
   On se donne \( \omega, \eta \) deux \( k \)-formes sur \( M \) ainsi que \( \lambda \) un scalaire, alors on a:
   \[ 
      d( \omega + \lambda\eta)_p = d\left( \sum_I (f_I + \lambda g_I)(p) dx^I\right) = \sum_I d(f_I + \lambda g_I)(p) \wedge dx^I = d\omega_p + \lambda d\eta_p
   \]
   \subsection*{Formule de Leibniz {:}}
   On se donne \( \omega, \eta \) respectivement une \( k \)-forme et une \( l \)-forme sur \( M \) alors on a:
   \[ 
      \begin{cases}
         \omega_p = \sum_I \omega_I(p)dx^I\\
         \eta_p = \sum_J \eta_J(p)dx^J
      \end{cases} 
   \]
   On a alors:
   \[ 
      (\omega \wedge \eta)_p = \sum_{I, J} (\omega_I\eta_J)(p)dx^I \wedge dx^J
   \]
   Appliquons la dérivée extérieure:
   \[ 
      d(\omega \wedge \eta)_p = \sum_{I, J} d(\omega_I\eta_J)(p) \wedge dx^I \wedge dx^J
   \]
   Les fonctions différentiées sont scalaires, donc on applique la règle du produit:
   \begin{flalign*}
      d(\omega \wedge \eta)_p &= \sum_{I, J} \left(d\omega_I(p)\eta_J(p) + \omega_I(p)d\eta_J(p)\right) \wedge dx^I \wedge dx^J \shorteqnote{(Par définition.)}\\
      &= \sum_{I, J} d\omega_I(p)\eta_J(p)\wedge dx^I \wedge dx^J  + \sum_{I, J} \omega_I(p)d\eta_J(p) \wedge dx^I \wedge dx^J \shorteqnote{(Distributivité du produit extérieur.)}\\
      &= \sum_{I, J} d\omega_I(p) \wedge dx^I \wedge \eta_J(p)dx^J  + \sum_{I, J} d\eta^J(p) \wedge \omega^I(p)dx^I \wedge dx^J \shorteqnote{(Bilinéarité du produit extérieur.)}\\
      &= d(\omega \wedge \eta)_p + (-1)^k\sum_{I, J} \omega_I(p)dx^I \wedge d\eta_J(p) \wedge  dx^J \shorteqnote{$(\star)$}\\
      &= d(\omega \wedge \eta)_p + (-1)^k(\omega \wedge d\eta)_p\\
      &= (d\omega \wedge \eta + (-1)^k(\omega \wedge d\eta))_p
   \end{flalign*}
   Où pour trouver \((\star)\), on identifie le premier terme, et on utilise le fait que \( \eta_J(p), dx_I \) sont respectivement une \( 1- \) forme et une \( k \)-forme et il y a donc \( k \) échanges à faire et donc \( k \) application de l'antisymétrie.
   \subsection*{Propriété fondamentale de la dérivée extérieure {:}}
   On considère une \( k \) forme \( \omega = \sum_I \omega_I dx^I\), alors on a:
   \[ 
      d\omega = \sum_I d\omega_I \wedge dx_I = \sum_I \sum_{i \leq n} \partialD{\omega_I}{x_i}dx^i \wedge dx^I
   \]
   Et donc si on dérive à nouveau on obtient:
   \[ 
      dd\omega =  \sum_I \sum_{i \leq n} \sum_{j \leq n} \frac{\partial^2 \omega_I}{\partial x_j \partial x_i} dx^j \wedge dx^i \wedge dx^I
   \]
   On peut alors montrer que pour \( I \) fixé, on a:
   \[ 
      \sum_{1 \leq i, j\leq n} \frac{\partial^2 \omega_I}{\partial x_j \partial x_i} dx^j \wedge dx^i = 0
   \]
   En effet si \( i = j \) c'est évident, mais si \( i \neq j \), on a que:
   \[ 
      \frac{\partial^2 \omega_I}{\partial x_j \partial x_i} dx^j \wedge dx^i + \frac{\partial^2 \omega_I}{\partial x_i \partial x_j} dx^i \wedge dx^j = 0
   \]
   En effet, \( \omega_I \) est une fonction lisse donc on applique Schwartz et les dérivées croisés sont égales, mais les produit extérieurs \( dx^i \wedge dx^j \) sont opposés. Ceci nous permet d'appairer les termes en \( n \) sommes nulles (il y a \( 2n \)) termes dans la somme).
   \subsection*{Existence de la dérivée extérieure globale{:}}
   Pour tout \( p \in M \), pour toute \( k- \)forme \( \omega \in \Omega^k(M)\), alors on définit dans les coordonées locales associées à une carte \( (U, \phi) \) l'opérateur suivant:
   \[ 
      d\omega_p = \sum_I d\omega_I(p) \wedge dx^I
   \]
   Alors on doit montrer que cet opérateur est bien défini, ie que \( d\omega_p \) ne dépends pas de la carte \( (U, \phi) \) choisie.
   \subsection*{Bonne définition {:}}
   Soit \( \omega \in \Omega^k(M) \), soit \( (U, \phi), (V, \psi) \) deux cartes qui contiennent \( p \), montrons que \( d\omega_p \) est bien définie, on a dans les deux cartes:
   \[ 
      \begin{cases}
         \omega_p = \sum_I a_I(p)dx^I \\
         \omega_p = \sum_I b_I(p)dy^I
      \end{cases} 
   \]
   Mais alors sur l'intersection de ces cartes, il existe une dérivée extérieure comme définie plus haut, et c'est un opérateur local donc on a l'implication suivante sur \( U \cap V \):
   \[ 
      \sum_I a_I(p)dx^I = \sum_I b_I(p)dy^I \implies \sum_I da_I(p) \wedge dx^I = \sum_I db_I(p) \wedge dy^I
   \]
\section*{\uline{Chapitre 11 - Orientation d'une variété à bord {:}}}
   \subsection*{Propriétés du produit intérieur {:}}
   \subsection*{Expression en coordonées {:}}

   \subsection*{Existence du champs de vecteurs sortant {:}}
   Soit $M$ une variété à bords, on souhaite montrer l'existence d'un champs de vecteurs $X : p \rightarrow TM$ \textbf{sortant} sur une variété. Dans chaque carte $(U_\alpha, \phi_\alpha)$, on peut construire une fonction $f_\alpha$ qui soit nulle sur $U_\alpha \cap \partial M$ et strictement positive en dehors du bord. On considère par la suite une partition de l'unité $(\rho_\alpha)$ induite par l'atlas, alors:
      $$
         f = \sum_\alpha \rho_\alpha f_\alpha
      $$
      Alors par construction $f|_{U_\alpha} = f_\alpha$ et donc en tout point de son domaine de définition, cette fonction coïncide avec les $f_\alpha$ et sa $n$-ième dérivée partielle est donc un vecteur entrant. Donc:
      $$
         p \longmapsto (p, -\frac{\partial f}{\partial x_n}\big|_p) \in TM
      $$
      C'est bien un champs de vecteurs sortant.
   \subsection*{Le produit intérieur avec le champs sortant oriente bien le bord {:}}

\section*{\uline{Chapitre 12 - Intégrale locale d'une forme {:}}}

\section*{\uline{Chapitre 13 - Intégrale d'une forme {:}}}

\section*{\uline{Chapitre 14 - Théorème de Stokes {:}}}
