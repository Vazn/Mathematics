\section*{\uline{Chapitre 3 - Variétés {:}}}
   \subsection*{La différentiabilité est bien définie {:}}
   Soit \( f : M \longrightarrow \R \), il s'agit de montrer que pour \( p \in M \) et deux cartes \( (U, \phi), (V, \psi) \) qui le contiennent, alors:
   \[ 
      f \circ \phi \text{ différentiable} \iff  f \circ \psi \text{ différentiable}
   \]
   Alors on a: 
   \[ 
      f \circ \psi^{-1} = f \circ \phi^{-1} \circ \phi \circ \psi^{-1} = f \circ \phi^{-1} \circ c_{\psi, \phi}
   \]
   Où \( c \) est l'application de changement de carte associée. On en déduit donc que si \( f \circ \psi^{-1} \) est différentiable alors \(f \circ \phi^{-1}\). De manière analogue si \( f : M \longrightarrow N \) est différentiable pour deux cartes \( (U, \phi), (V, \psi) \) de \( p, f(p) \), alors par composition, on en déduis de même.
   \subsection*{Existence d'une partition de l'unité - Cas compact{:}}
   Soit \( (U_\alpha, \psi_i) \) l'atlas de \( M \) qui est compacte pour simplifier. Alors pour tout \( p \in M \) on considère une famille \( (f_p)_{p \in M} \) de fonctions bosses lisses chacune supportée dans une carte \( (U_\alpha) \) qui contient \( p \). Alors on a directement que \( f_p(p) > 0 \) donc on peut considère le recouvrement de \( M \) par la famille consistuée des:
   \[ 
      O_p := \left\{ q \in M \; ; \; f_p(q) > 0 \right\} 
   \]
   C'est un recouvrement ouvert de \( M \), par compacité on peut en extraitre un recouvrement fini \( (O_i)_{i \leq k} \) et donc une famille finie de fonctions associées \( (f_i)_{i \leq k} \). Alors on pose la nouvelle famille de fonctions suivante:
   \[ 
      \phi_i = \frac{f_i}{\sum_{i \leq k} f_i }
   \]
   Le dénominateur est bien strictement positif en tout point de \( M \) car il existe une \( f_i \) strictement positive sur un voisinage de celui-ci. Alors, on a par construction:
   \[ 
      \begin{cases}
         \sum_i \phi_i = 1\\
         \text{supp}(\phi_i) = \text{supp}(f_i) \subseteq U_\alpha 
      \end{cases} 
   \]
   Il ne reste qu'à réindexer la famille \( (\phi_i)_{i \leq n} \) par \( \alpha \). Pour ceci on pose:
   \[ 
      \rho_\alpha = \sum_J \phi_j \text{ avec } J := \left\{ i \leq k \; ; \; \text{supp}( \phi_i) \subseteq U_\alpha \right\} 
   \] 
   En d'autres termes on regroupe les \( \phi_i \) selon leur support, avec \( \rho_\alpha = 0\) si l'ensemble d'indice est vide. Alors on a bien une famille \( (\rho_\alpha) \) qui vérifie les propriétés voulues.
\pagebreak
\section*{\uline{Chapitre 4 - Variétés à bord {:}}}
   \subsection*{Invariance par homéomorphisme {:}}
      Soit \(U, V\) deux ouverts de \( \mathbb{H}^n \) et \(f : U \longrightarrow V\) un homéomorphisme, alors:
      \begin{itemize}
         \item Si \( p \in U \) et \( p \in \text{int}\mathbb{H}^n\), alors il existe une boule ouverte \( B(p, r) \) de \( \mathbb{R}^n \) incluse dans \( U \). Mais alors d'aprés le théorème d'invariance du domaine:
         \[ 
            f(B) \text{ est un ouvert de } \R^n 
         \]
         C'est donc un ouvert de \( \R^n \) inclu dans \( \mathbb{H}^n \), il est donc inclu dans l'intérieur de \( \mathbb{H}^n \) et donc \( f(p) \) est intérieur.
         \item Si \( p \in U \) et \( p \in \partial\mathbb{H}^n\), alors par l'absurde supposons que \( f(p) \) soit intérieur, alors \( f^{-1}(f(p)) = p\) serait intérieur car \( f^{-1} \) est un homéomorphisme et préserve les points intérieurs. Absurde. 
      \end{itemize}
   \subsection*{Généralisation de la notion de variété {:}}
      On considère une variété au sens du chapitre \( 3 \), alors chaque homéomorphisme de son atlas est un homéomorphisme sur un ouvert de \( \R^n \). Aussi on pose l'application suivante:
      \begin{align*}
         f: \R^n  &\longrightarrow \R^n\\
         (x_1, \ldots, x_n) &\longrightarrow (x_1, \ldots, x_{n-1}, e^{x_n})
      \end{align*}
      Alors c'est facilement un difféomorphisme sur son image qui est \( \mathbb{H}^n_> \). Donc pour toute carte \( (U, \phi) \), on construit une nouvelle carte \( (U, f \circ \phi) \) et plus précisément, on a:
      \begin{align*}
         f \circ \phi : U &\longrightarrow V \subseteq \R^n\\
         &p \longrightarrow (f \circ \phi)(p)
      \end{align*}
      On obtient bien un nouvel homéomorphisme (sur son image) par composition et par le théorème d'invariance du domaine l'ouvert \( V = (f \circ \phi)(U) \) obtenu est aussi un ouvert \textbf{de l'espace ambiant} \( \R^n \). Aussi cet ouvert est inclu dans le demi-espace par construction de \( f \), il s'écrit donc \( \mathbb{H}^n \cap V \), ie c'est un ouvert de \( \mathbb{H}^n \).\<

      Aussi si on considère l'application de changement de carte \(f \circ \psi \circ \phi^{-1} \circ f^{-1}\), alors c'est un difféomorphisme par composition, on donc bien un structure de variété différentielle au sens du chapitre \( 4 \).
   \subsection*{Le bord d'une variété est une variété différentielle {:}}
      Soit \( M \) une variété variété à bord de dimension \( n \) de bord non-vide et d'atlas \( (U_\alpha, \phi_\alpha) \).
      \begin{itemize}
         \item Si \( p \in \partial M \) et si \( (U, \phi) \) est une carte qui contient \( p \), on a un homéomorphisme:
         \[ 
            \phi : U \longrightarrow \phi(U) \subseteq \mathbb{H}^n 
         \]
         Mais alors il se restreint en un homéomorphisme:
         \[ 
            \phi|_{\partial M} : U \cap \partial M \longrightarrow \phi(U) \cap \partial \mathbb{H}^n
         \]
         Aussi \( \phi(U) \cap \partial \mathbb{H}^n \) est un ouvert de \( \partial \mathbb{H}^n \) donc est homéomorphe à \( \R^{n-1} \). Finalement, on pose:
         \[ 
            \mathcal{A}_{ \partial M} :=  (U_\alpha \cap  \partial M, \phi_\alpha|_{\partial M})
         \]
         Ceci munit bien \( \partial M \) d'une structure de variété topologique de dimenson \( n-1 \).
         \item Si \( p \in U \cap V\), alors l'application de changement de carte au bord est:
         \[ 
            \psi \circ \phi^{-1} : \phi(U \cap V) \cap \partial \mathbb{H}^n \longrightarrow \psi(U \cap V) \cap \partial \mathbb{H}^n
         \]
         C'est la restriction au bord de \( \mathbb{H}^n \) de l'application de changement de carte qui est différentiable sur l'ouvert \( \phi(U \cap V) \). Elle est donc différentiable par définition d'une application différentiable sur une partie quelconque.
      \end{itemize}
   \pagebreak
   \subsection*{L'intérieur d'une variété est une variété différentielle {:}}
      Soit \( M \) une variété à bord de dimension \( n \) et d'atlas \( (U_\alpha, \phi_\alpha) \).
      \begin{itemize}
         \item Si \( p \in \text{int}M \) et si \( (U, \phi) \) est une carte qui contient \( p \), on a un homéomorphisme:
         \[ 
            \phi : U \longrightarrow \phi(U) \subseteq \mathbb{H}^n 
         \]
         Mais alors il se restreint en un homéomorphisme:
         \[ 
            \phi|_{\text{int}M} : U \cap \text{int}M \longrightarrow \phi(U \cap \text{int}M) \subseteq \text{int}\mathbb{H}^n
         \]
         Aussi \( \text{int}\mathbb{H}^n \cong \R^{n} \) donc finalement, on pose:
         \[ 
            \mathcal{A}_{ \text{int}M} :=  (U_\alpha \cap  \text{int}M, \phi_\alpha|_{\text{int}M})
         \]
         Ceci munit bien \( \text{int}M \) d'une structure de variété topologique de dimenson \(n\).
         \item Si \( p \in U \cap V\), alors l'application de changement de carte en l'intérieur est:
         \[ 
            \psi \circ \phi^{-1} : \phi(U \cap V) \cap \text{int}\mathbb{H}^n \longrightarrow \psi(U \cap V) \cap \text{int}\mathbb{H}^n
         \]
         C'est la restriction à un ouvert de l'application de changement de carte qui est différentiable sur l'ouvert \( \phi(U \cap V) \). Elle est donc différentiable.
      \end{itemize}
\pagebreak

\section*{\uline{Chapitre 6 - Espaces tangents dans \( \R^n \) {:}}}
   \subsection*{Propriété des dérivations {:}}
   Si \( f \) est une fonction lisse constante égale à \( c \in \R \), et \( D \) une dérivation au point \( p \in \R^n \), alors \( Df = 0 \), en effet on a:
   \begin{flalign*}
      D(fg) &= D(f)g(p) + f(p)D(g) \iff \\
      D(cg) &= D(f)g(p) + cD(g) \iff \shorteqnote{(Car \( fg = cg \) en tant que fonction.)}\\
      cD(g) &= D(f)g(p) + cD(g) \shorteqnote{(Linéarité de \( D \).)}
   \end{flalign*}
   On conclut de la dernière égalité que \( D(f)g(p) = 0 \) et il suffit alors de choisir \( g \) de telle sorte qu'elle ne s'anulle pas en \( p \) pour conclure.
   \subsection*{\( T\R^n_p \) est un espace vectoriel isomorphe à \( \R^n \) {:}}
   On considère l'application suivante:
   \[ 
      \begin{aligned}
         \Phi : \R^n_p &\longrightarrow T\R^n_p \\
         v &\longmapsto \sum_{i \leq n} v_i \partialD{}{x_i}\biggr|_p
      \end{aligned} 
   \]
   \uline{Linéarité:} Découle directement de la définition ..\+
   \uline{Injectivité:} Supposons que \( v \in \R^n_p \) soit tel que:
   \[ 
      D = \Phi(v) = 0
   \]
   Alors pour toute fonction lisse \( f \in \mathcal{F}(\R^n_p, \R) \), on a \( Df = 0 \). Alors en particulier pour les fonctions coordonées \( (x_i)_{i \leq n} \), on a:
   \[ 
      \sum_{j \leq n}v_j \partialD{x_i}{x_j}\biggr|_p = v_i = 0
   \]
   Donc toutes les composantes de \( v \) sont nulles et \( v = 0 \).\+
   \uline{Surjectivité:}
   {\textbf{\color{red} Pas compris ...}}
\pagebreak
\section*{\uline{Chapitre 7 - Espaces tangents dans \( M \) {:}}}
   \subsection*{Propriété de la différentielle {:}}
      On considère \( f: M \longrightarrow N \) lisse et un point \( p \in M \), et on considère l'application suivante:
      \[ 
         df_p : D \in TM_p \longmapsto D( \cdot  \circ f) \in TN_{f(p)}
      \]
      \begin{itemize}
         \item \textbf{Bien définie:} Si \( D \) est une dérivation en \( p \), et \( g, h \) sont deux fonctions lisses, alors on a: 
         \begin{flalign*}
            df_p(D)(gh) &= D(gh \circ f)\\
            &= D((g \circ f)(h \circ f))\\
            &= D(g \circ f)(h \circ f)(p) + (g \circ f)(p)D(h \circ f) \shorteqnote{(Car \( D \) est une dérivation en \( p \).)}\\
            &= df_p(g)h(f(p)) + g(f(p))df_p(h)
         \end{flalign*}
         Et donc \( df_p(D) \) est bien une dérivation en \( f(p) \).
         \item \textbf{Linéarité:} Découle directement de la définition.
         \item \textbf{Règle de la chaîne:} Soit \( f : M \longrightarrow N \) et \( g : N \longrightarrow L \) deux applications lisses entre des variétés. Soit \( p \in M \), alors par définition:
         \[ 
            d(g \circ f)_p : D \in TM_p \longmapsto D(\cdot \circ g \circ f)) \in TL_{g(f(p))} 
         \]
         Il s'agit de montrer que cette fonction est égale à \(dg_{f(p)} \circ df_p \), les domaines de définition coincident bien, soit \( D \) une dérivation de \( TM_p \), alors on a:
         \[ 
            dg_{f(p)} \circ df_p(D) = dg_{f(p)}(df_p(D)) = dg_{f(p)}(D( \cdot \circ f)) = D(\cdot \circ g \circ f)
         \]
         On a donc égalité des images et donc égalité des fonctions.
      \end{itemize}
   \subsection*{Propriété de structure {:}}
      Finalement montrons que si \( f : M \longrightarrow N \) est un \textbf{difféomorphisme}, alors \( df_p \) en un \textbf{isomorphisme} en tout point. On utilisera le lemme (trivial) suivant:
      \[ 
         d(Id_M)_p = Id_{TM_p} 
      \]
      Alors si \( f \) est un difféormorphisme, en tout point \( p \), on a d'aprés la règle de la chaîne:
      \[ 
         \begin{cases}
            d(f^{-1} \circ f)_p = d(f^{-1})_{f(p)} \circ df_p = Id_{TM_p}\\
            d(f \circ f^{-1})_{f(p)} = df_{p} \circ d(f^{-1})_{f(p)} = Id_{TN_{f(p)}}
         \end{cases}
      \]
      Finalement, ceci montre que \( df_p \) est bijective d'inverse \( d(f^{-1})_{f(p)} \) et linéaire par définition, on a donc bien:
      \[ 
         TM_p \cong TN_{f(p)} 
      \]
      On en déduit directement en utilisant que toute carte \( \phi \) est un difféomorphisme et donc qu'en tout point d'une variété on a:
      \[ 
         TM_p \cong T\R^n_{\phi(p)} \cong \R^n 
      \]
   \subsection*{Changement de représentation des vecteurs tangents {:}}
   Si \( (U, \phi), (V, \psi) \) sont deux cartes qui contiennent un point \( p \), alors si on note \( x_i, y_i \) les coordonées respectives dans la première et la deuxième carte et \( c = \psi \circ \phi^{-1} \) l'application de changement de carte alors pour toute fonction lisse \( g \), on a:   
   \begin{flalign*}
      \partialD{}{x_i}\biggr|_p g &= \partialD{g \circ \phi^{-1}}{x_i}(\phi(p))\\
      &= \partialD{g \circ \psi^{-1} \circ \psi \circ \phi^{-1}}{x_i}(\phi(p))\\
      &= \sum_{j \leq n} \partialD{g \circ \psi^{-1}}{y_j}(\psi(p))\partialD{c_j}{x_i}(\phi(p)) \shorteqnote{(Règle de la chaîne dans \( \R^n \))}\\
      &= \sum_{j \leq n} \partialD{c_j}{x_i}(\phi(p))\partialD{}{y_j}\biggr|_pg
   \end{flalign*}
   Ceci étant vrai pour toute application \( g \), on a donc:
   \[ 
      \partialD{}{x_i}\biggr|_p = \sum_{j \leq n} \partialD{c_j}{x_i}(\phi(p)) \partialD{}{y_j}\biggr|_p
   \]
   \subsection*{Le fibré tangent est un variété topologique de dimension \( 2n \){:}}
      On définit simultanément une \textbf{topologie} sur \( TM \) et une carte. Soit \( (U, \phi) \) une carte de \( M \), alors on définit:
      \[ 
         \begin{aligned}
            \Phi : \pi^{-1}(U) &\longrightarrow \phi(U) \times T\R^n_p \\
            (p, v) &\longmapsto (\phi(p), d\phi_{p}(v))
         \end{aligned}         
      \]
      Alors cette application est \textbf{bijective}, en effet on a:
      \[ 
         \begin{aligned}
            \Phi^{-1} : \phi(U) \times T\R^n_p &\longrightarrow \pi^{-1}(U) \\
            (x_1, \ldots, x_n, v_1, \ldots, v_n) &\longmapsto (\phi^{-1}(x_1, \ldots, x_n), d\phi^{-1}_{x_1, \ldots, x_n}(v_1, \ldots, v_n))
         \end{aligned}         
      \]
      Alors on identifie par la suite \( T\R^n_p \) et \( \R^n \) et on définit une topologie sur \( TM \) comme suit:
      \[ 
         \mathcal{T}_{TM} := \left\{ U \subseteq TM \; ; \; \forall (V, \phi) \in \mathcal{A}_M \; , \; \Phi^{-1}(U \cap \pi^{-1}(V)) \in \mathcal{T}_{ \R^{2n}} \right\}  
      \]
      \textbf{\color{red} J'admet } les faits suivants (les démonstrations sont soit introuvables soit horribles):
      \begin{itemize}
         \item Cette topologie est séparée.
         \item Les \( \pi^{-1}(U_\alpha) \) sont bien un recouvrement ouvert localement fini de \( TM \).
         \item Les cartes \( \Phi_{\alpha} \) sont bien toutes des homéomorphismes sur une partie\footnote[1]{Plus rigoureusement à une partie elle même homéomorphe à une partie de \(\R^{2n}\), car \( T \R^n_p \cong \R^n \).} de \(\R^{2n}\).
      \end{itemize}
      Ceci munit \( TM \) d'une structure de variété topologique de dimension \( 2n \).
   \subsection*{Le fibré tangent est un variété différentielle de dimension \( 2n \){:}}
      Les applications de changement de cartes sont lisses, en effet soit \( (U, \phi), (V, \psi) \) deux cartes qui contiennent \( p \), alors si \( (\pi^{-1}(U), \Phi), (\pi^{-1}(V), \Psi) \) sont deux cartes qui contiennent \( (p, v) \), alors:
         \[ 
            \begin{cases}
               E = \Phi(\pi^{-1}(U) \cap \pi^{-1}(V)) = \phi(U \cap V) \times \R^n\\
               F = \Psi(\pi^{-1}(U) \cap \pi^{-1}(V)) = \psi(U \cap V) \times \R^n
            \end{cases}
         \]
         Et donc on peut calculer l'application de changement de cartes:
         \begin{flalign*}
            \Psi \circ \Phi^{-1}(x_1, \ldots, x_n, v_1, \ldots, v_n) 
            &= \Psi\left(\phi^{-1}(x_1, \ldots, x_n), d\phi^{-1}_{x_1, \ldots, x_n}(v_1, \ldots, v_n)\right) \shorteqnote{(Par définition)}\\ 
            &= \left(\psi \circ \phi^{-1}(x_1, \ldots, x_n), d(\psi \circ \phi^{-1})_{x_1, \ldots, x_n}(v_1, \ldots, v_n)\right) \shorteqnote{(Règle de la chaîne)}\\
            &= \left(\psi \circ \phi^{-1}(x_1, \ldots, x_n), \sum_{i \leq n} \partialD{\psi \circ \phi^{-1}}{x_1}(x_1, \ldots, x_n)v_i\right)
         \end{flalign*} 
         Les \( n \) premières composantes sont lisses car \( \psi  \circ \phi^{-1} \) l'est. Les \( n \) dernières le sont car toutes les dérivées partielles sont lisses et par combinaisons linéaires de fonctions lisses. On remarque néanmoins que si \( M \) était simplement \( \mathcal{C}^k \) alors \( TM \) serait \( \mathcal{C}^{k-1} \) par l'apparition des dérivées partielles.
   \subsection*{Un champs de vecteurs est lisse ssi ses composantes le sont{:}}
      Soit \( X: M \longrightarrow TM \) un champs de vecteur lisse et \( p \in M \), alors pour que le champs de vecteur soit lisse, il faut et il suffit qu'il le soit dans deux cartes bien choisies.\<
      
      On choisit \( (U, \phi) \) n'importe quelle carte qui contient \( p \) et \( (\pi^{-1}(U), \Phi) \) la carte induite par celle-ci (qui contient bien \( X(p) \)). Alors le champs de vecteurs est lisse si et seulement si:
      \[ 
         \Phi \circ X \circ \phi^{-1} \text{ est lisse.}
      \]
      \pagebreak

      Or on a par calcul direct que:
      \begin{flalign*}
         \Phi \circ X \circ \phi^{-1}(x_1, \ldots, x_n) &= \Phi(X(p))\\
         &= \Phi(p, \sum X_i(p)\partialD{}{x_i}\big|_{p}) \shorteqnote{On décompose \( X_p \) dans la base associée à \( \phi \).}\\
         &= (p, X_1(p), \ldots, X_n(p))
      \end{flalign*}
      Cette application est bien lisse si et seulement si les \( n \) dernières composantes le sont, ie si les composantes du champs de vecteurs le sont. La preuve est la même pour les champs de covecteurs, tenseurs ...
   \pagebreak
\section*{\uline{Chapitre 8 - Espaces cotangents dans \( M \) {:}}}
   \subsection*{Différentielle abstraite et différentielle usuelle - Partie 1:}
      L'expression de la différentielle d'une fonction \( f : M \longrightarrow \R \) s'identifie à l'expression usuelle d'une différentielle, en effet, si \( v \in TM_p \), alors pour toute fonction lisse \( g : \R \longrightarrow \R \) on a:
      \begin{align*}
         df_p(v)(g) &= \sum_{i=1}^n v_i df_p\left(\partialD{}{x_i}\biggr|_p\right)(g)\\
         &= \sum_{i=1}^n v_i \partialD{g \circ f}{x_i}(p)\\
         &= \sum_{i=1}^n v_i \partialD{g}{t}(f(t))\partialD{f}{x_i}(p)\\
         &= \left(\sum_{i=1}^n v_i \partialD{f}{x_i}(p)\partialD{}{t}\biggr|_{f(p)}\right)(g)
      \end{align*}
      Ceci étant vrai pour toute fonction lisse \( g \), on trouve l'expression suivante:
      \[ 
         df_p(v) = \sum_{i=1}^n v_i \partialD{f}{x_i}(p)\partialD{}{t}\biggr|_{f(p)}
      \]
   \subsection*{Différentielle abstraite et différentielle usuelle - Partie 2:}
      L'expression de la différentielle d'une fonction \( f : M \longrightarrow N \) s'identifie aussi à l'expression usuelle d'une différentielle, en effet, si \( v \in TM_p \), alors pour toute fonction lisse \( g : M \longrightarrow \R \) on a:
      \begin{align*}
         df_p(v)(g) &= \sum_{i=1}^n v_i df_p\left(\partialD{}{x_i}\biggr|_p\right)(g)\\
         &= \sum_{i=1}^n v_i \partialD{g \circ f}{x_i}(p)\\
         &= \sum_{i=1}^n v_i \sum_{j=1}^m \partialD{g}{y_j}(f(p)) \partialD{f_j}{x_i}(p)\\
         &= \left(\sum_{i=1}^n\sum_{j=1}^m v_i \partialD{f_j}{x_i}(p)\partialD{}{y_j}\biggr|_{f(p)}\right)(g)
      \end{align*}
      Où ici \( f_j \) représente les coordonées locales de \( f \) au voisinage de \( f(p) \), ie \( f_j = (\psi \circ f)_j(p) \). Ceci étant vrai pour toute fonction lisse \( g \), on trouve l'expression suivante:
      \[ 
         df_p(v) = \sum_{i=1}^n\sum_{j=1}^m v_i \partialD{f_j}{x_i}(p)\partialD{}{y_j}\biggr|_{f(p)}
      \]

\pagebreak
\section*{\uline{Chapitre 9 - Formes différentielles dans \( M \) {:}}}
   \subsection*{Le pullback commute avec la différentielle}
      Soit $x$ un fonction lisse sur $N$, alors d’après la règle de la chaîne:
      \begin{align*}
         \forall p \in M \; ; \; (dF^*x)_p &= (d(x \circ F))_p = dx_{F(p)} \circ dF_p = (F^*dx)_p 
      \end{align*}
   \subsection*{Linéarité du pullback}  
      Soit $\lambda \in \mathbb{R}$, $\omega, \eta$ deux $k$-formes, alors pour tout point $p$ de $M$, on a:
      \begin{align*}
         (F^*(\lambda\omega + \eta))_p &= (\lambda\omega + \eta)_{F(p)} \circ (dF_p, \ldots, dF_p)\\ 
         &=  (\lambda\omega_{F(p)} + \eta_{F(p)}) \circ (dF_p, \ldots, dF_p)\\ 
         &= \lambda\omega_{F(p)} \circ (dF_p, \ldots, dF_p) + \eta_{F(p)} \circ (dF_p, \ldots, dF_p)\\ 
         &= (\lambda F^*\omega)_p + (F^*\eta)_p\\
         &= (\lambda F^*\omega + F^*\eta)_p
      \end{align*}
   \subsection*{Le pullback est compatible avec le produit extérieur}  
      Soit $\omega, \eta$ respectivement des $k$ et $l$ formes sur $N$, alors pour tout point $p$ de $M$, on a:
      \begin{align*}
         (F^*(\omega \wedge \eta))_p &= (\omega \wedge \eta)_{F(p)} \circ (dF_{p}, \ldots, dF_{p})\\ 
         &=   (\omega_{F(p)} \wedge \eta_{F(p)}) \circ (dF_{p}, \ldots, dF_{p})
      \end{align*}
      Soit $v_1, \ldots, v_{k+l}$ des vecteurs de $TM_p$, alors on évalue l'expression ci-dessus en:
      $$
         (\omega_{F(p)} \wedge \eta_{F(p)}) (dF_{p}(v_1), \ldots, dF_{p}(v_{k+l}))
      $$
      Puis par définition, la quantité ci-dessus est égale à:
      \begin{align*}
         \frac{1}{k!l!} \sum_{\sigma \in S_{k+l}} \epsilon(\sigma) (\omega_{F(p)} \otimes \eta_{F(p)})(dF_{p}(v_{\sigma(1)}), \ldots, dF_{p}(v_{\sigma(k+l)}))
      \end{align*}
      Puis par évaluation du produit tensoriel:
      $$
         \frac{1}{k!l!} \sum_{\sigma \in S_{k+l}} \epsilon(\sigma) \omega_{F(p)}\left[dF_{p}(v_{\sigma(1)}), \ldots, dF_p(v_{\sigma(k)})\right] \eta_{F(p)}\left[dF_{p}(v_{\sigma(k+1)}), \ldots, dF_p(v_{\sigma(k+l)})\right]
      $$
      Finalement, ceci est bien égal à
      \begin{align*}
         \frac{1}{k!l!} \sum_{\sigma \in S_{k+l}} \epsilon(\sigma) (\omega_{F(p)} \circ (dF_p, \ldots, dF_p)) \otimes (\eta_{F(p)} \circ (dF_p, \ldots, dF_p))(v_{\sigma(1)}, \ldots, v_{\sigma(k+l)})
      \end{align*}
      Qui corresponds bien à l'expression:
      $$
         (\omega_{F(p)} \circ (dF_{p}, \ldots, dF_{p})) \wedge (\eta_{F(p)} \circ (dF_{p}, \ldots, dF_{p})) = F^*\omega \wedge F^*\eta
      $$
      En particulier, dans le cas d'un produit avec une 0-forme, le produit extérieur est une multiplication scalaire et on a:
      $$
         F^*(g\omega) = (F^*g)(F^*\omega) = (g \circ F)F^*\omega
      $$    
   \subsection*{Expression en coordonées du pullback}
      Cette fois on considère une $k$ forme $\omega$, alors en coordonnées locales, on a:
      \begin{align*}
      F^*\omega &= F^*\left[\sum_{I} \omega_{i_1, \ldots, i_n} dx^{i_1} \wedge \ldots \wedge dx^{i_n}\right]\\
      &= \sum_{I} F^*(\omega_{i_1, \ldots, i_n} dx^{i_1} \wedge \ldots \wedge dx^{i_n})\\
      &= \sum_{I} F^*\omega_{i_1, \ldots, i_n} F^*(dx^{i_1})  \wedge \ldots \wedge F^*(dx^{i_n})\\
      &= \sum_{I} F^*\omega_{i_1, \ldots, i_n} d(F^*x^{i_1})  \wedge \ldots \wedge d(F^*x^{i_n})\\
      &= \sum_{I} (\omega_{i_1, \ldots, i_n} \circ F) d(x^{i_1} \circ F)  \wedge \ldots \wedge d(x^{i_n} \circ F)\\
      &= \sum_{I} (\omega_{i_1, \ldots, i_n} \circ F) dF^{i_1}  \wedge \ldots \wedge  dF^{i_n}
      \end{align*}
      Où on a utilisé successivement la linéarité, la compatibilité avec le produit extérieur et la commutation avec la différentielle usuelle du pullback. On a noté $dF^i = d(x^i \circ F)$.
   \subsection*{Pullback d'une forme volume}
      On se donne une forme volume $\omega$ sur $N$ et un difféomorphisme $F : M \longrightarrow N$, alors d'aprés l'expression trouvé précédemment, on a:
      $$
          F^*\omega = (\omega \circ F) dF^{1}  \wedge \ldots \wedge  dF^{n}
      $$
      Or par définition de la différentielle, on peut réecrire ceci en:
      \[ 
         F^*\omega = (\omega \circ F) \left(\sum_{{j_1} \leq n} \partialD{F^{1}}{x_{j_1}}dx^{j_1}  \wedge \ldots \wedge  \sum_{{j_n} \leq n} \partialD{F^{n}}{x_{j_1}}dx^{j_n}\right)
      \]
      Puis par multilinéarité on obtient:
      \[ 
         F^*\omega = (\omega \circ F) \sum_{1 \leq j_1, \ldots, j_p \leq n} \partialD{F^{1}}{x_{j_1}}\ldots\partialD{F^{n}}{x_{j_n}} dx^{j_1}  \wedge \ldots \wedge  dx^{j_n}
      \]
      Or on somme en fait sur tout les indices distincts donc de manière équivalente sur toutes les permutations des indices:
      \[ 
         F^*\omega = (\omega \circ F) \sum_{\sigma \in S_n} \partialD{F^{1}}{x_{\sigma(1)}}\ldots\partialD{F^{n}}{x_{\sigma(n)}} dx^{\sigma(1)}\wedge \ldots \wedge  dx^{\sigma(n)}
      \]
      Finalement, par les propriétés du produit extérieur, on trouve:
      \[ 
         F^*\omega = (\omega \circ F) \left(\sum_{\sigma \in S_n} \epsilon(\sigma)\partialD{F^{1}}{x_{\sigma(1)}}\ldots\partialD{F^{n}}{x_{\sigma(n)}}\right) dx^{1}\wedge \ldots \wedge  dx^{n} = (\omega \circ F)\text{det}(J_F)dx^1 \wedge \ldots \wedge dx^n
      \]
   \subsection*{Une dérivée extérieure est locale {:}}
      Si \( d \) est un opérateur de dérivée extérieure, alors si \( \omega = \eta \) sur un ouvert \( U \), alors si on considère une fonction bosse \( f \) sur \( U \), on a que \( f \omega = f \eta \) sur tout \(M\). En outre, d'aprés la règle de Leibniz:
      \[ 
         \begin{cases}
            d(f \omega)_p = df(p) \wedge \omega(p) + f(p)d\omega(p)\\
            d(f \eta)_p = df(p) \wedge \eta(p) + f(p)d\eta(p)
         \end{cases}
      \]
      Donc on en déduit que:
      \[ 
         df(p) \wedge \omega(p) + f(p)d\omega(p) = df(p) \wedge \eta(p) + f(p)d\eta(p)
      \]
      Alors on peut conclure car si \( p \in U \), on obtient par définition de \( f \) que \( f(p) = 1 \) et \( df(p) = 0 \) et donc:
      \[ 
         \forall p \in U \; ; \; d\omega(p) = d\eta(p)
      \]
   \subsection*{Une dérivée extérieure commute avec le pullback}
      On considère une $k$ forme $\omega$ et \( F : M \longrightarrow N \) une application lisse,  alors en coordonnées locales, on a:
      \begin{align*}
         F^*d\omega &= F^*\left[\sum_{I} d\omega_{i_1, \ldots, i_n} \wedge dx^{i_1} \wedge \ldots \wedge dx^{i_n}\right]\\
         &= \sum_{I} F^*(d\omega_{i_1, \ldots, i_n} \wedge dx^{i_1}  \wedge \ldots \wedge dx^{i_n})\\
         &= \sum_{I} F^*(d\omega_{i_1, \ldots, i_n}) \wedge F^*(dx^{i_1})  \wedge \ldots \wedge F^*(dx^{i_n})\\
         &= \sum_{I} d(F^*\omega_{i_1, \ldots, i_n}) \wedge d(F^*x^{i_1})  \wedge \ldots \wedge d(F^*x^{i_n})\\
         &= \sum_{I} d(\omega_{i_1, \ldots, i_n} \circ F) \wedge d(x^{i_1} \circ F)  \wedge \ldots \wedge d(x^{i_n} \circ F)\\
         &= \sum_{I} d(\omega_{i_1, \ldots, i_n} \circ F) \wedge dF^{i_1}  \wedge \ldots \wedge  dF^{i_n}
      \end{align*}
      D'autre part en dérivant l'expression en coordonées locales du pullback ci-dessous, on obtient bien l'égalité:
      \[ 
         F^*\omega = \sum_{I} \omega_{i_1, \ldots, i_n} \circ F \wedge dF^{i_1}  \wedge \ldots \wedge  dF^{i_n}
      \]
   \subsection*{Forme nécessaire de la dérivée extérieure locale {:}}
      On procède par analyse-synthèse, et on considère une dérivée extérieure \( d \) qui vérifie les axiomes. Alors pour toute \( k- \)forme \( \omega \in \Omega^k(U)\), on a nécessairement:
      \begin{flalign*}
         d\omega_p &= d\left(\sum_I \omega_I(p) dx^I\right)\\
         &=\sum_I d(\omega_I(p) dx^I)\shorteqnote{(Linéarité)} \\
         &= \sum_I d\omega_I(p) \wedge dx^I + \omega_I(p)d(dx^I)\shorteqnote{(Leibniz)}\\
         &= \sum_I d\omega_I(p) \wedge dx^I
      \end{flalign*}
      La dernière ligne venant du fait que \( d^2 = 0 \) . Si une dérivée extérieure existe sur \( \Omega^k(U) \), alors elle est uniquement déterminée par la forme ci-dessus.
   \subsection*{Existence de la dérivée extérieure locale{:}}
      Pour tout \( p \in U \), pour toute \( k- \)forme \( \omega \in \Omega^k(U)\) l'opérateur suivant:
      \[ 
         d\omega_p = \sum_I d\omega_I(p) \wedge dx^I
      \]
      Alors on doit montrer que cet opérateur est bien une dérivée extérieure.
      \begin{itemize}
         \item \textbf{Généralisation:} Si \( f \in \Omega^0(U) \), c'est une fonction lisse, et alors \( d \) agit par définition par différentiation des coefficients, l'unique coefficient de la \( 0 \)-forme \( f \), étant elle même, on conclut.
         \item \textbf{Linéarité:} Si \( \omega, \eta \) sont deux \( k \)-formes sur \( U \) ainsi que \( \lambda \) un scalaire, alors on a:
         \[ 
            d( \omega + \lambda\eta)_p = d\left( \sum_I (f_I + \lambda g_I)(p) dx^I\right) = \sum_I d(f_I + \lambda g_I)(p) \wedge dx^I = d\omega_p + \lambda d\eta_p
         \]
         \pagebreak

         \item \textbf{Leibniz:} Si \( \omega, \eta \) sont respectivement une \( k \)-forme et une \( l \)-forme sur \( U \) alors on a:
         \[ 
            \begin{cases}
               \omega_p = \sum_I \omega_I(p)dx^I\\
               \eta_p = \sum_J \eta_J(p)dx^J
            \end{cases} 
         \]
         On a alors:
         \[ 
            (\omega \wedge \eta)_p = \sum_{I, J} (\omega_I\eta_J)(p)dx^I \wedge dx^J
         \]
         Appliquons la dérivée extérieure:
         \[ 
            d(\omega \wedge \eta)_p = \sum_{I, J} d(\omega_I\eta_J)(p) \wedge dx^I \wedge dx^J
         \]
         Les fonctions différentiées sont scalaires, donc on applique la règle du produit:
         \begin{flalign*}
            d(\omega \wedge \eta)_p &= \sum_{I, J} \left(d\omega_I(p)\eta_J(p) + \omega_I(p)d\eta_J(p)\right) \wedge dx^I \wedge dx^J \shorteqnote{(Par définition.)}\\
            &= \sum_{I, J} d\omega_I(p)\eta_J(p)\wedge dx^I \wedge dx^J  + \sum_{I, J} \omega_I(p)d\eta_J(p) \wedge dx^I \wedge dx^J\\
            &= \sum_{I, J} d\omega_I(p) \wedge dx^I \wedge \eta_J(p)dx^J  + \sum_{I, J} d\eta^J(p) \wedge \omega^I(p)dx^I \wedge dx^J\\
            &= d(\omega \wedge \eta)_p + (-1)^k\sum_{I, J} \omega_I(p)dx^I \wedge d\eta_J(p) \wedge  dx^J \shorteqnote{$(\star)$}\\
            &= d(\omega \wedge \eta)_p + (-1)^k(\omega \wedge d\eta)_p\\
            &= (d\omega \wedge \eta + (-1)^k(\omega \wedge d\eta))_p
         \end{flalign*}
         Où pour trouver \((\star)\), on identifie le premier terme, et on utilise le fait que \( \eta_J(p), dx_I \) sont respectivement une \( 1- \) forme et une \( k \)-forme et il y a donc \( k \) échanges à faire et donc \( k \) application de l'antisymétrie.
         \item \textbf{Propriété fondamentale:} Si on considère une \( k \) forme \( \omega = \sum_I \omega_I dx^I\), alors on a:
         \[ 
            d\omega = \sum_I d\omega_I \wedge dx_I = \sum_I \sum_{i \leq n} \partialD{\omega_I}{x_i}dx^i \wedge dx^I
         \]
         Et donc si on dérive à nouveau on obtient:
         \[ 
            dd\omega =  \sum_I \sum_{i \leq n} \sum_{j \leq n} \frac{\partial^2 \omega_I}{\partial x_j \partial x_i} dx^j \wedge dx^i \wedge dx^I
         \]
         On peut alors montrer que pour \( I \) fixé, on a:
         \[ 
            \sum_{1 \leq i, j\leq n} \frac{\partial^2 \omega_I}{\partial x_j \partial x_i} dx^j \wedge dx^i = 0
         \]
         En effet si \( i = j \) c'est évident, mais si \( i \neq j \), on a que:
         \[ 
            \frac{\partial^2 \omega_I}{\partial x_j \partial x_i} dx^j \wedge dx^i + \frac{\partial^2 \omega_I}{\partial x_i \partial x_j} dx^i \wedge dx^j = 0
         \]
         En effet, \( \omega_I \) est une fonction lisse donc on applique Schwartz et les dérivées croisés sont égales, mais les produit extérieurs \( dx^i \wedge dx^j \) sont opposés. Ceci nous permet d'appairer les termes en \( n \) sommes nulles (il y a \( 2n \)) termes dans la somme).
      \end{itemize}
   On a donc bien montré qu'il existe bien une dérivée extérieure sur les \( k \)-formes définies sur une carte \( U \).
   \pagebreak
   \subsection*{Existence de la dérivée extérieure globale{:}}
   On note \( d|_U \) l'unique dérivée extérieure sur une carte \( (U, \phi) \) trouvée ci-dessus. Pour tout \( p \in M \), pour toute \( k- \)forme \( \omega \in \Omega^k(M)\), alors on définit l'opérateur suivant:
   \[ 
      (d\omega)_p = (d|_U\omega)_p
   \]
   Soit \( \omega \in \Omega^k(M) \), soit \( (U, \phi), (V, \psi) \) deux cartes qui contiennent \( p \), montrons que \( d\omega_p \) est bien définie, ie que:
   \[ 
      (d|_U\omega)_p = (d|_V\omega)_p
   \]
   Sur \( U \cap V \), on a que les deux expressions en coordonées de \( \omega \) sont égales:
   \[ 
      \sum_I \omega_I dx^I = \sum_I \omega'_I dy^I
   \]
   Or, si on applique \( d|_U \), on obtient:
   \[ 
      d|_U\omega = \sum_I d\omega_I \wedge dx^I = \sum_I d\omega'_I \wedge dy^I = d|_V\omega
   \]
   Cette définition ne dépends donc pas du choix de la carte qui contient \( p \). En outre c'est bien une dérivée extérieure car \( d|_U \) en est une, et donc elle est locale.
   \subsection*{Unicité de la dérivée extérieure globale{:}}
\pagebreak
\section*{\uline{Chapitre 10 - Orientation d'une variété{:}}}
   \subsection{Deux classes d'équivalences}
      Si \( M \) est connexe et \( \omega, \omega' \) sont deux formes volumes alors l'espace des formes volumes en un point est de dimension 1, on a donc l'existence d'une fonction lisse \( f \) telle que:
      \[ 
         \forall p \in M \; ; \; \omega_p = f(p) \omega_p' 
      \]
      Aussi \( f \) n'est jamais nulle sinon les formes le seraient. Supposons maintenant par l'absurde:
      \[ 
         \begin{cases}
            \exists p_0 \in M \; ; \; f(p_0) > 0\\
            \exists p_1 \in M \; ; \; f(p_1) < 0
         \end{cases} 
      \]
      Alors on a que \(f(M)\) est un connexe de \( \R \) qui contient des nombres positifs et négatifs, elle s'annule, ce qui est absurde. Donc on a bien que:
      \begin{itemize}
         \item Soit \( \omega = f \omega' \; ; \; f > 0 \)
         \item Soit \( \omega = f\omega' \; ; \; f < 0 \) 
      \end{itemize}
      On a donc bien deux classes possibles sur une variété connexe, celle de \( \omega \) et celle de \( -\omega \). En outre si la variété n'est pas connexe, il suffit de faire ce raisonnement sur chaque composante connexe.
   \subsection{Les changements de cartes ont un jacobien positif}
      On se donne une forme volume \( \omega \) sur \( M \), ainsi que deux cartes \( (U, \phi), (V, \psi) \) positivement orientées qui contiennent un point \( p \in M\), alors on a que la forme \( \omega \) se ramène dans \( \R^n \) via la carte \( \phi \) en:
      \[ 
         \omega(x_1, \ldots, x_n) dx^1 \wedge \ldots \wedge dx^n
      \]
      On applique la formule du pullback d'une forme volume pour le difféomorphisme de changement de carte et on obtient:
      \[ 
         \omega(x_1, \ldots, x_n) dx^1 \wedge \ldots \wedge dx^n = \omega(y_1, \ldots, y_n) \text{det}(\text{Jac}(c)) dy^1 \wedge \ldots \wedge dy^n
      \]
      Or les deux cartes sont orientées positivement par hypothèse, donc nécéssairement:
      \[ 
         \begin{cases}
            \omega(x_1, \ldots, x_n) > 0\\
            \omega(y_1, \ldots, y_n) > 0
         \end{cases}
      \] 
      On en déduit que le jacobien du changement de carte est positif.
   \subsection*{Existence d'un champs de vecteurs sortant}
      On souhait construire un champs de vecteurs $X$ sur $M$ tel que pour tout point du bord $\partial M$, $X(p)$ soit un vecteur sortant. On considère une partition de l'unité subordonnée à l'atlas que l'on note $\rho_\alpha$, alors pour toute carte $U_\alpha$ qui intersecte le bord, on définit un champs de vecteur sortant en coordonnées locales:
      $$
         X_\alpha(p) = -\frac{\partial}{\partial x_n}\bigg|_p
      $$
      Alors on aimerait globalise ce champs de vecteur en un champs de vecteur global par la partition de l'unité, on pose:
      $$
         X = \sum_\alpha \rho_\alpha X_\alpha
      $$
      C'est bien un champs de vecteurs sur $M$, lisse car: 
      \begin{align*}
         X(p) &= \sum_\alpha \rho_\alpha(p) X_\alpha(p) \\
         &= \sum_\alpha \rho_\alpha(p) X_\alpha(p)\\
         &= \sum_\alpha \rho_\alpha(p) \sum_{i \leq n}X_{\alpha, i}(p)\frac{\partial}{\partial x_i}\bigg|_p\\
         &= \sum_{i \leq n}\sum_\alpha \rho_\alpha(p)X_{\alpha, i}(p)\frac{\partial}{\partial x_i}\bigg|_p
      \end{align*}
      C'est une somme finie dont toutes les composantes sont lisses car les champs de vecteurs $\rho_\alpha X_\alpha$ sont lisses. Aussi pour tout point du bord $p$, alors \( p \in U_\alpha \cap \partial M \) et on a:
      $$
         (X(p))_n = \sum_\alpha \rho_\alpha(p) X_{\alpha, n}(p)\frac{\partial}{\partial x_n}\bigg|_p = -\frac{\partial}{\partial x_n}\bigg|_p 
      $$
      Qui est bien un vecteur sortant.\<
   \subsection*{Construction d'une forme volume sur $\partial M$}
      On considère une variété orientable à bord $M$ et on note $\omega$ une forme volume sur $M$, montrons que l'on peut construire une forme volume sur $\partial M$. La construction est la suivante:
      \begin{itemize}
         \item On considère un champs sortant $S$ sur $\partial M$.
         \item On construit le produit intérieur de $\iota_S\omega$, qui est une $n-1$ forme sur $M$.
         \item La restriction\footnote{Plus précisement, le pullback par l'inclusion.}  de $\iota_S\omega$ au bord est alors une forme volume sur $\partial M$.
      \end{itemize}
      On sait déja que $\iota_S\omega$ est une $n-1$ forme lisse, supposons maintenant par l'absurde qu'il existe $p \in \partial M$ tel que cette forme s’annule. Ceci signifie que:
      $$
         \forall (v_1, \ldots, v_{n-1}) \in T\partial M_p \; ; \; \iota_S\omega_p(v_1, \ldots, v_{n-1}) = 0
      $$
      C'est donc en particulier vrai pour la base $\left(\frac{\partial}{\partial x_1}\big|_p, \ldots, \frac{\partial}{\partial x_{n-1}}\big|_p\right)$, mais alors:
      $$
      \iota_S\omega_p\left(\frac{\partial}{\partial x_1}\bigg|_p, \ldots, \frac{\partial}{\partial x_{n-1}}\bigg|_p\right) = \omega_p\left(-\frac{\partial}{\partial x_n}\bigg|_p,\frac{\partial}{\partial x_1}\bigg|_p, \ldots, \frac{\partial}{\partial x_{n-1}}\bigg|_p\right) = 0
      $$
      Et la famille sur laquelle on évalue $\omega_p$ est une base de $TM_p$ donc $\omega$ s'annule en \( p \), absurde.

   \subsection*{Expression de la forme volume induite}
   On se donne une forme volume \( \omega \) sur \( M \) et un champs de vecteurs sortant \( S \) au bord de \( M \), on a alors pour tout point \( p \in \partial M \):
   \begin{align*}
      \forall (v_1, \ldots, v_{n-1}) \in T\partial M_p \; ; \; \iota_S\omega_p(v_1, \ldots, v_{n-1}) = \omega_p(S(p), v_1, \ldots, v_n) = \widetilde{\omega}(p)dx^1 \wedge \ldots \wedge dx^n(S(p), v_1, \ldots, v_n)
   \end{align*}
   Or ceci se ramène à un calcul de déterminant (simple car \( S(p) = - \partialD{}{x^n}\big|_p \)), ie il suffit de calculer:
   \begin{align*}
      dx^1 \wedge \ldots \wedge dx^n(S(p), v_1, \ldots, v_n) &= \begin{vmatrix}
         S(p)_1 & v_{1, 1} & \ldots & v_{n, 1}\\
         \vdots & \vdots & \ddots & \vdots\\
         S(p)_n & v_{1, n} & \ldots & v_{n, n}
         \end{vmatrix}\\
      &= \begin{vmatrix}
         0 & v_{1, 1} & \ldots & v_{n, 1}\\
         \vdots & \vdots & \ddots & \vdots\\
         -1 & v_{1, n} & \ldots & v_{n, n}
         \end{vmatrix}\\
      &= (-1)^n dx^1 \wedge \ldots \wedge dx^{n-1}(v_1, \ldots, v_n)
   \end{align*}
   Ceci étant vrai pour tout \( v_1, \ldots, v_{n-1} \), on trouve bien:
   \[ 
      \iota_S\omega_p = \widetilde{\omega}(p)(-1)^n dx^1 \wedge \ldots \wedge dx^{n-1}
   \]
