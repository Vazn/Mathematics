\section*{\uline{Chapitre 12 - Intégrale d'une forme {:}}}
   \subsection*{Changement de variables {:}}
   Soit \( \omega \in \Omega^n_c( \R^n) \) une \( n-\)forme et \( F : \R^n \longrightarrow \R^n \) un difféormorphisme. Alors d'une part on a d'aprés la formule du changement de variables pour les fonctions:
   \begin{itemize}
      \item D'aprés la formule du changement de variable :
      \[ 
         \int_{\R^n} \omega =  \int_{\R^n} \widetilde{\omega}dx^1 \ldots dx^n = \int_{\R^n} \widetilde{\omega} \circ F |\text{det}(JF)| dx^1 \ldots dx^n
      \]
      \item D'aprés les propriétés du pullback :
      \[ 
         \int_{\R^n} F^*\omega = \int_{\R^n} \widetilde{\omega} \circ F \text{det}(JF) dx^1 \ldots dx^n
      \]
   \end{itemize}
   On a donc bien égalité des deux membres si et seulement si \( |\text{det}(JF)| = \text{det}(JF) \), ie ssi \( \text{det}(JF) > 0 \).
   \subsection*{L'intégrale est bien définie {:}}
   On doit montrer que si \( (U_\alpha, \phi_\alpha) \) et \( (V_\beta, \psi_\beta) \) sont deux atlas orientés de \( M \), et que \( \rho_\alpha, \rho_\beta \) sont deux partitions de l'unité induite par ceux-ci, alors:
   \begin{align}
      \sum_\alpha \int_{U_\alpha} \rho_\alpha \omega =  \sum_\beta \int_{U_\beta} \rho_\beta \omega
   \end{align}
   On construit les deux nouveaux atlas suivants:
   \[ 
      (U_\alpha \cap V_\beta, \phi_\alpha|_{V_\beta}) \; ; \; (U_\alpha \cap V_\beta, \psi_\alpha|_{U_\beta})
   \]
   Ils sont orientés car toutes les cartes choisies sont seulement des restrictions de cartes positivement orientées par hypothèse. Alors on effectue le calcul suivant:
   \begin{flalign*}
      \sum_\alpha \int_{U_\alpha} \rho_\alpha \omega &= \sum_\alpha \int_{U_\alpha} \rho_\alpha \sum_\beta \rho_\beta \omega \shorteqnote{(Car la somme introduite vaut identiquement \( 1 \).)} \\
      &= \sum_\alpha \sum_\beta \int_{U_\alpha} \rho_\alpha\rho_\beta \omega \shorteqnote{(Car les sommes sont localement finies.)}\\
      &= \sum_\alpha \sum_\beta \int_{U_\alpha \cap V_\beta} \rho_\alpha\rho_\beta \omega \shorteqnote{(Car le support de l'intégrande est dans \( U_\alpha \cap V_\beta \).)}
   \end{flalign*}
   On peut effectuer le même calcul avec l'expression de droite de (1) et obtient bien l'égalité et l'intégrale est bien définie.
\pagebreak

\section*{\uline{Chapitre 13 - Théorème de Stokes {:}}}
   \subsection*{Stokes dans \( \mathbb{H}^n \) {:}}
   On considère une \( n-1\) forme \(\omega \in \Omega_c^n(\mathbb{H}^n)\), alors elle est de la forme:
   \[ 
      \omega = \sum_{i = 1}^n \omega_i dx^1 \wedge \ldots \wedge \widehat{dx^i} \wedge \ldots \wedge dx^n
   \]
   Ou le chapeau désigne une ommission. Alors on calcule sa dérivée extérieure et on trouve:
   \begin{align*}
      d\omega &= \sum_{i = 1}^n \sum_{j = 1}^n \partialD{\omega_i}{x_j} dx^j \wedge dx^1 \wedge \ldots \wedge \widehat{dx^i} \wedge \ldots \wedge dx^n\\
      &=  \sum_{i = 1}^n (-1)^{i+1} \partialD{\omega_i}{x_i} dx^1 \wedge \ldots \wedge dx^n
   \end{align*}
   \begin{itemize}
      \item On calcule alors l'intégrale de celle-ci judicieusement gràce à Fubini:
      \begin{flalign*}
         \int_{\mathbb{H}^n} d\omega &= \sum_{i = 1}^n \int_{\mathbb{R}^{n-1} \times \mathbb{R}_+} (-1)^{i+1} \partialD{\omega_i}{x_i} dx^1 \ldots dx^n\\
         &= \sum_{i = 1}^{n-1} \int_{\mathbb{R}^{n-1} \times \mathbb{R}_+} (-1)^{i+1} \partialD{\omega_i}{x_i} dx^1 \ldots dx^n + (-1)^{n+1}\int_{\mathbb{R}^{n-1} \times \mathbb{R}_+} \partialD{\omega_n}{x_n} dx^1 \ldots dx^n\\
         &= \sum_{i = 1}^{n-1} (-1)^{i+1} \int_{\mathbb{R}^{n-2} \times \mathbb{R}_+} \left(\int_\mathbb{R} \partialD{\omega_i}{x_i} dx^i\right) dx^1 \ldots \widehat{dx^i} \ldots dx^n + (-1)^{n+1}\int_{\mathbb{R}^{n-1}} \left(\int_{\mathbb{R}_+} \partialD{\omega_n}{x_n} dx^n\right) dx^1 \ldots dx^{n-1} \\
         &=(-1)^{n} \int_{\mathbb{R}^{n-1}} \omega_n(x_1, \ldots, x_{n-1}, 0) dx^1 \ldots dx^{n-1}
      \end{flalign*}
      Le terme de gauche dans l'avant dernière égalité s'annulle car \( \omega \)  est à support compact, le terme de droite se calcule par application du théorème fondamental de l'analyse.
      \item Puis l'intégrale sur le bord où on note \( \iota : \partial \mathbb{H}^n \hookrightarrow \mathbb{H}^n\) se calcule par:
      \begin{flalign*}
         \int_{\partial\mathbb{H}^n} \iota^*\omega &= \int_{\partial\mathbb{H}^n} \sum_{k = 1}^n  \iota^*\left(\omega_idx^1 \wedge \ldots \wedge \widehat{dx^i} \wedge \ldots \wedge dx^n\right)\\
         &= \int_{\partial\mathbb{H}^n}(\omega_n \circ \iota) dx^1 \wedge \ldots \wedge dx^{n-1} \shorteqnote{(Car \( d(x^n \circ \iota) = 0 \))}\\
         &= (-1)^n \int_{\R^{n-1}} \omega_n(x_1, \ldots, x_{n-1}, 0) dx^1 \ldots dx^{n-1}
      \end{flalign*}
      En effet le difféomorphisme \( F : \mathbb{R}^{n-1} \longrightarrow \partial \mathbb{H}^n  \) préserve ou non les orientation choisies selon la parité de \( n \), donc quand on effectue le changement de variable, un facteur \( (-1)^n \) apparaît.
   \end{itemize}
   On a donc bien démontré le théorème de Stokes dans \( \mathbb{H}^n \).
   \subsection*{Stokes dans \( M \) {:}}
   Soit \( M \) une variété orientée munie de son atlas orienté \( (U_\alpha)_\alpha \) ainsi qu'un partition de l'unité \((\rho_\alpha)_\alpha\) associée. Soit \( \omega \in \Omega^n(M)\) une \( n-1 \) forme. On oriente \( \partial M\) avec l'orientation induite par un champs de vecteurs sortant. Alors en appliquant Stokes dans chaque carte, on a:
   \begin{flalign*}
      \int_{\partial M} \omega = \sum_\alpha \int_{\partial U_\alpha} \rho_\alpha \omega = \sum_\alpha \int_{U_\alpha} d(\rho_\alpha \omega) = \int_{M} d\omega
   \end{flalign*}