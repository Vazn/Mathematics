\chapter*{\chapterstyle{VIII --- Introduction}} % A REFAIRE
\addcontentsline{toc}{section}{Introduction} 
Dans ce chapitre nous étudirons les propriétés des fonctions définies sur l'ensemble des complexes. En particulier nous chercherons à définir une notion de \textbf{différentielle} puis une notion \textbf{d'intégrale} pour ces fonctions et enfin étudier les propriétés des ces deux constructions.\+

On rapelle à tout fins utiles \(\C\) est défini par la structure \((\R^2, +, \times)\) avec une multiplication définie par:
\[
   (a, b)(c, d) = (ac - bd, ad + bc)
\]
On rapelle aussi que l'on a \(\C \cong \R^2\) et en particulier la représentation des nombres complexes en tant que couple nous donne une représentation de la multiplication complexe par le produit matriciel suivant:
\[
   (a + bi)(c + di) = ac - bd + i(ad + bc) \sim (ac - bd, ad + bc) =  \begin{pmatrix}
      a & -b \\
      b & a
   \end{pmatrix} \begin{pmatrix}  c \\ d \end{pmatrix}
\]
\subsection*{\subsecstyle{Applications conformes {:}}}
On appelle \textbf{application conforme} toute application qui préservent les angles. En particulier, si on se donne \(\Gamma\) une courbe paramétrée par \(\gamma\) sur \(I\), et \(f\) un application, alors on dira:

est conforme si et seulement si


\chapter*{\chapterstyle{VIII --- Fonctions Holomorphes}} % A REFAIRE
\addcontentsline{toc}{section}{Fonctions Holomorphes} 
On définit ici la notion de \textbf{différentielle complexe}, qui sera une généralisation directe de la différentielle dans le cas réel, en effet on dira que \(f\) est \textbf{différentiable} en \(a\) si et seulement si le taux d'accroissement suivant, appelée dérivée de \(f\) en \(a\) existe:
\[
   \lim_{z \rightarrow a} \frac{f(z) - f(a)}{z - a} = f'(a)
\]
Où de manière équivalente si il existe une application \(\C\)-linéaire \(A\) telle que dans un voisinage de \(a\) on ait:
\[
   f(z) - f(a) - A(z - a) = o(z - a)
\]
On peut alors définir la différentielle de \(f\) en chaque point où elle existe par:
\[
   df : a \in \C \mapsto f'(a)dz
\]
On dira alors que \(f\) est holomorphe sur un ouvert si elle y est holomorphe en tout point, et qu'elle est \textbf{entière} si elle est holomorphe sur \(\C\).
\subsection*{\subsecstyle{Propriétés opératoires{:}}}
En particulier, vu que la définition de la différentielle est analogue à la différentielle réelle, on a alors (par les mêmes démonstrations) toutes les propriétés opératoires de l'opérateur de différentiation, en particulier:
\begin{itemize}
   \item La différentiation est un opérateur linéaire.
   \item La différentiation d'un produit suit la règle de Leibniz.
   \item La différentiation d'une composée suit la règle réelle de différentiation d'une composée.
   \item Le théorème d'inversion local nous donne la différentielle d'une réciproque.
\end{itemize}

\subsection*{\subsecstyle{Différences avec la différentiation réelle {:}}}
Ici on remarque tout de suite la différence notable qui est que la différentielle, qu'on sait être un champ de formes linéaires, est en fait un champ de forme \(\K\)-linéaires selon le corps dans lequel on différentie, ici on nécéssite que la forme soit \(\C\)-linéaire, ce qui va changer ses propriétés.\<

En particulier, on considère une fonction holomorphe \(f\) et on va calculer sa différentielle en \(z\) selon deux restrictions, une par valeurs \textbf{réelles uniquement}, l'autre par valeurs \textbf{imaginaires uniquement}, on obtient alors les deux taux d'accroissements suivants:
\[
   \lim_{h \underset{\R}{\rightarrow} 0} \frac{f(z + h) - f(z)}{h} \quad \quad \quad \lim_{h \underset{i\R}{\rightarrow} 0} \frac{f(z + h) - f(z)}{h}
\]
On sait qu'à tout fonction complexe, on peut associer une fonction sur \(\R^2\), et on va alors étudier les propriétés de ces taux d'accroissements en identifiant \(f(z) \sim f(x, y) = u(x, y) + iv(x, y)\), qui est bien différentiable au sens réel. En développant les taux d'accroissements ci-dessus en \((u, v)\), on obtient alors les deux égalités suivantes:
\[
   \begin{cases}
      f'(x, y) = \partialD{u}{x}(x, y) + i\partialD{v}{x}(x, y)\\
      f'(x, y) = \partialD{v}{y}(x, y) - i\partialD{u}{y}(x, y)
   \end{cases}
\]
Et enfin en regroupant ces égalités et en identifiant les parties réelles et imaginaires, on obtient alors que tout fonction holomorphe vérifie les \textbf{équations de Cauchy-Riemann}:
\[
   \begin{cases}
      \partialD{u}{x}(x, y) = \partialD{v}{y}(x, y)\\
      \partialD{v}{x}(x, y) = \partialD{u}{y}(x, y)
   \end{cases}
\]
\pagebreak 
\subsection*{\subsecstyle{Equations de Cauchy-Riemann{:}}}
En fait, on peut même montrer que ces équations caractérisent l'holomorphie, en effet pour \(f : \C \rightarrow \C\), on a:
\begin{center}
   \textbf{Si est différentiable \textbf{au sens réel} et vérifie les équations de Cauchy-Riemann, alors elle est holomorphe et réciproquement.}
\end{center}
\uline{Exemple:} Considérons la fonction \(f(z) = \overline{z}\), alors on a \(f(z) \sim f(x, y) = x - iy\) différentiable sur tout son domaine de définition mais on a:
\[
   \partialD{u}{x} = 1 \neq -1 = \partialD{v}{y} 
\]
Donc nécéssairement, la fonction conjugué n'est donc \textbf{pas holomorphe}.
\subsection*{\subsecstyle{Equations de Laplace{:}}}
On appelle \textbf{équations de Laplace} une équation aux dérivées partielles de la forme:
\[
   \partialD{{}^2f}{x_1^2} + \ldots + \partialD{{}^2f}{x_n^2} = 0
\]
Et on appelle alors \textbf{Laplacien} l'opérateur suivant:
\[
   \Delta = \partialD{{}^2}{x_1^2} + \ldots + \partialD{{}^2}{x_n^2}
\]
On appellera alors toute fonction solution des équations de Laplace \textbf{fonction harmonique}, et un des résultats intéressants de l'analyse complexe permet de montrer le résultat suivant:
\begin{center}
   \textbf{Toute fonction analytique telle que ses dérivées partielles secondes soient continues est harmonique.}
\end{center}
\chapter*{\chapterstyle{VIII --- Intégration Complexe}} % A REFAIRE
\addcontentsline{toc}{section}{Intégration Complexe} 
On cherche maintenant à définir une notion \textbf{d'intégrale} pour les fonctions complexe, plus précisément nous allons définir l'intégrale d'une fonction continue \(f\) le long d'une courbe paramétrée par \(\gamma\) sur \(\icc{a}{b}\), qui sera défini par l'intégrale suivante:
\[
   \int_{\gamma} f d\gamma = \int_{a}^{b} f(\gamma(t))\gamma'(t) d t
\]
L'intérprétation géométrique n'est pas évidente, mais peut s'éclaircir si on considère \(f(z) \sim f(x, y)\) et \(\overline{f}(z) = f(x, -y)\) le (champ vectoriel) conjugué de \(f\) apellé \textbf{champ de Pòlya}, alors on a pour des notations usuelles:
\[
   \int_{\gamma} f d\gamma = \int_{\gamma} \dotproduct{\overline{f}(\gamma(t))}{T(t)} d t + i \int_{\gamma} \dotproduct{\overline{f}(\gamma(t))}{N(t)}  d t
\]
En d'autres termes, cette intégrale encapsule \textbf{le travail et le flux du champ de Pòlya}.

\subsection*{\subsecstyle{Théorème fondamental de l'analyse complexe{:}}}
On se donne une fonction continue \(f\) tel que \(f =  F'\) pour une certaine fonction holomorphe \(F\) et on se donne un chemin paramétré par \(\gamma\) sur \(\icc{a}{b}\) et on cherche à calculer l'intégrale le long de ce chemin, alors on trouve:
\begin{align*}
   \int_{\gamma} f d \gamma &:= \int_{a}^{b} F'(\gamma(t))\gamma'(t)d t = \int_{a}^{b} (F \circ \gamma)'(t) d t = F(\gamma(b)) - F(\gamma(a))
\end{align*}
Ce n'est pas surprenant quand on comprends la dualité qui existe entre fonction complexe et champ vectoriel, ce n'est alors qu'un cas particulier du \textbf{théorème du gradient} vu en analyse vectorielle. Et aussi, comme énoncé dans le chapitre en question, c'est aussi un cas particulier du \textbf{théorème de Stokes} pour la forme différentielle exacte \(f(z)dz\).


\chapter*{\chapterstyle{VIII --- Quelques extensions}} % A REFAIRE
\addcontentsline{toc}{section}{Quelques extensions} 


\chapter*{\chapterstyle{VIII --- Grands Théorèmes}} % A REFAIRE
\addcontentsline{toc}{section}{Grands Théorèmes} 